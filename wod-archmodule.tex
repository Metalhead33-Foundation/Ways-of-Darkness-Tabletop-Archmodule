\documentclass[openany,10pt,a4paper]{book}
\usepackage{graphicx}
\usepackage[utf8x]{inputenc}
\usepackage{multirow}
\usepackage{blindtext}
\usepackage{graphicx}
\usepackage{numprint}
\usepackage{listings}
\usepackage{xcolor}
\usepackage{tabularx,booktabs}
\usepackage{amsmath}
\usepackage{mathtools}
\usepackage{float}
\usepackage{caption}
\usepackage{hyperref}
\usepackage{enumitem}
\usepackage{newtxtext,newtxmath}
\usepackage{makecell}
\usepackage{import}
\usepackage[square,sort,comma,numbers]{natbib}
\usepackage[a4paper, margin=1.5cm]{geometry}
\newcolumntype{Y}{>{\centering\arraybackslash}X}
\graphicspath{ {./images/} }
\author{Metalhead33}
\title{Ways of Darkness\\
   \normalsize A module for the World of Artograch RPG system}

\definecolor{mGreen}{rgb}{0,0.6,0}
\definecolor{mGray}{rgb}{0.5,0.5,0.5}
\definecolor{mPurple}{rgb}{0.3,0,0.3}
\definecolor{backgroundColour}{rgb}{0.95,0.95,0.92}

\begin{document}
\newcommand{\BipedalTwoArms}[3]{Being bipedal {#1}, {#3} have two hands and two legs, meaning that they can wield only two one-handed weapons \textit{(or a one-handed weapon and a shield)} or a single two-handed weapon at the same time. Just like with other bipedal {#2} races, reaching zero hitpoints on the arms or legs renders those limbs unusable \textit{(and disables the corresponding item slots)}, while reaching zero hitpoints at the torso or head means death - or unconsciousness, if house rules rule out death.}
\newcommand{\MammalRace}[1]{\BipedalTwoArms{mammals without tails}{mammalian}{{#1}} As mammals, {#1} cannot breathe under water without magic or specifial \textit{(read: modern, therefore nonexistent in this setting)} equipment, with prolonged presence underweater leading to drowning.}
\newcommand{\LizardRace}[1]{\BipedalTwoArms{reptoids with tails}{reptilian}{{#1}} Their tails are not a vital body part - their loss may be very painful \textit{(and cause potentially lethal bleeding, if the wound is not treated fast enough)}, the body part itself isn't vital, making its loss non-lethal. As we're talking about reptiles rather than mammmals, this species breeds by eggs rather than giving live birth; has a critical weakness towards extreme temperatures and cannot be turned into a secondary race specific to mammals. Reptiles are different from amphibians - they are \textbf{not} able to breathe underwater, and thus drown just like mammals. They cannot be infected with Vampirism or Theriantropy, because those are mammal-specific diseases.}
\newcommand{\UndeadRace}[1]{Being undead creatures, {#1} have no need for food or water, thus the vitals for proteon, carbohydrates, fat, calories, alcohol and water are inactive for them.}
\newcommand{\Bonus}[1]{\textcolor{green}{\textbf{+{#1} bonus}}}
\newcommand{\BonusS}[1]{\textcolor{green}{\textbf{+{#1}}}}
\newcommand{\Malus}[1]{\textcolor{red}{\textbf{-{#1} malus}}}
\newcommand{\MalusS}[1]{\textcolor{red}{\textbf{-{#1}}}}
\newcommand{\DimorphismBB}[5]{Male {#1} get a \Bonus{{#2}} to {#3}, while Female {#1} get a \Bonus{{#4}} to {#5}.}
\newcommand{\SoCalled}[1]{\textit{``{#1}''}}
\newcommand{\Parentheses}[1]{\textit{({#1})}}
\newcommand{\FeatIII}[7]{\begin{tabularx}{0.75\textwidth}{c @{\extracolsep{\fill}} cc}
\hline
\multicolumn{3}{|c|}{{#4}} \\ \hline
\multicolumn{1}{|c}{\resizebox{.2\linewidth}{!}{{#5}}} & \multicolumn{1}{c}{\resizebox{.2\linewidth}{!}{{#6}}} & \multicolumn{1}{c|}{\resizebox{.2\linewidth}{!}{{#7}}} \\ \hline
\multicolumn{1}{|c}{{#1}}        & \multicolumn{1}{c}{{#2}}       & \multicolumn{1}{c|}{{#3}}       \\ \hline
\end{tabularx}}

\maketitle
\tableofcontents
\chapter*{Preface}
You are currently reading the documentation of the \textbf{Ways of Darkness module for} the \textbf{World of Artograch RPG system}. The author of this document presents you all the information within this document with the assumption that you have read the \textbf{World of Artograch Ruleset}, and thus are familiar with the rules it presented. Contents of the aforementioned document are expected to be referenced in this document.
\addcontentsline{toc}{chapter}{Preface}
\chapter{Introduction to the Occident}
\includegraphics[width=\textwidth,height=\textheight,keepaspectratio]{Occident_Borders}
\newpage
The \textbf{Occident} is the part of \textbf{Artograch} with perhaps the most dynamic history. While even the Occident shows a similar tendency as the Orient to have large countries dominated by a single race, it is slightly more nuanced in the Occident, which has had a history of territories changing hands. While the Orient has always been characterized by the rule of centralized and homogenous kingdoms, and highly bureaucratic and usually equally homogenous, often quasi-despotic empires, the Occident has always been a place where central authorities held only limited amount of power, with provinces enjoying high autonomy.\newline
The various races of the Occident are divided into two major groups: the Elven races \textit{(Humans, High Elves, Wood Elves, Dark Elves, Orcs)} being descended from Oriental invaders; and the indigenous races \textit{(Goblins, Ogres, Lizardmen, Halflings, Gnomes and Dwarves)} whose population was decimated by the arrival of the earlier. Out of the two, the earlier are clearly the dominant force on the continent, with three out of the four dominant powers in the Occident being dominated by an Elven race - Etrand by Humans, Froturn by High Elves, Dragoc by Wood Elves.\newline
Currently, as of 831 AEKE \textit{(831 years after the establishment of the Kingdom of Etrand)}, the Occident is at a tipping point: the three aforementioned great powers are at a three-way cold war with each other, with tensions being at an all-time high, while the previously unmentioned fourth, Gabyr, is sitting in the shadows watching it all with popcorn in their hands, while slowly expanding their trade empire in the Orient. Unseen for roughly five and a half centuries, Demons have been sighted lurking around, no doubt planting their own secret agents among the authorities of the aforementioned states. It is said that even the Orcs of Brutang are getting quite unruly, while the Empire of Neressa continues its long-practiced isolationist policies and the Principality of Artaburro tries to wiggle between the Kingdoms of Froturn, Dragoc and Etrand.
\chapter{Playable Primary Races}
\section{Humans}
\includegraphics{Average_Humans}\newline
\textbf{Humans} are the dominant race of the \textbf{Kingdom of Etrand} and its vassal, the \textbf{Earldom of Etrancoast}, albeit they are also present in high numbers in the \textbf{Principality of Gabyr}, constituting slightly more than half of its population, albeit they're not politically dominant there. In spite of their short lifespan - less than one century! - and their clear lack of pointy ears, humans are in fact an Elven race, descendants of those Ancestral Elves who crossed over from the Orient to colonize the Occident, making them a sister-race to the Wood Elves and High Elves, who also directly evolved out of those Ancestral Elves.\newline
The average human male is 175 centimetres \textit{(5 feet and 9 inches)} tall, while the average human female is 165 centimetres \textit{(5 feet and 5 inches)} tall, however, humans possess much greater variation amongst themselves in regards to height than other races, adult humans being as tall as 215 centimetres \textit{(84 inches or 7 feet)} or as short as 130 centimetres \textit{(4 feet and 3 inches)} not being unheard of either. With possibly the exception of the Lizardmen with the varied colours of their scale, humans are the most diverse race on Artograch, with a large variety of hair colours \textit{(black, brown, blond, red, auburn)} and eye colours \textit{(brown, blue, hazel, grey, green)} occouring naturally.\newline
\begin{tabular}{|c|c|c|}
\hline
 & \textbf{Min} & \textbf{Max} \\ \hline
\textbf{Strength} & 8 & 18 \\ \hline
\textbf{Endurance} & 8 & 18 \\ \hline
\textbf{Dexterity} & 8 & 18 \\ \hline
\textbf{Intelligence} & 8 & 18 \\ \hline
\textbf{Willpower} & 8 & 18 \\ \hline
\textbf{Charisma} & 8 & 18 \\ \hline
\end{tabular}\newline
\DimorphismBB{Humans}{1}{Strength and Endurance}{1}{Dexterity and Charisma} \MammalRace{humans}\newpage
\section{High Elves}
\includegraphics{High_Elves}\newline
\textbf{High Elves} are the dominant race of the \textbf{Kingdom of Froturn} and have a plurality in the \textbf{Empire of Neressa}, albeit their political influence expands far beyond, to the point that the Humans have adopted their religion over eight centuries ago! They have rather long lifespans, theoretically ten times that of a human \textit{(and they're also typically taller than humans, not to mention their sharp ears)}, ageing at one tenth a human's rate after reaching the age of eighteen. They descendants of those Ancestral Elves who crossed over from the Orient to colonize the Occident, making them a sister-race to the Wood Elves and Humans, who also directly evolved out of those Ancestral Elves. As an added bonus, High Elven women are deemed the most beautiful on Artograch, their fame echoing beyond the continent of Artograch, even to the Orient. The average adult High Elf is about 190 centimetres \textit{(6 feet and 3 inches)} tall, regardless of gender \textit{(albeit men to tend to be slightly taller on average)}.\newline
\begin{tabular}{|c|c|c|}
\hline
 & \textbf{Min} & \textbf{Max} \\ \hline
\textbf{Strength} & 8 & 18 \\ \hline
\textbf{Endurance} & 7 & 17 \\ \hline
\textbf{Dexterity} & 9 & 19 \\ \hline
\textbf{Intelligence} & 8 & 18 \\ \hline
\textbf{Willpower} & 8 & 18 \\ \hline
\textbf{Charisma} & 9 & 19 \\ \hline
\end{tabular}\newline
\DimorphismBB{High Elves}{1}{Strength}{1}{Charisma} \MammalRace{High Elves}\newpage
\section{Wood Elves}
\includegraphics{Wood_Elves}\newline
\textbf{Wood Elves} are the dominant race of the \textbf{Kingdom of Dragoc}. They have rather long lifespans, theoretically ten times that of a human \textit{(and they're also typically taller than humans, not to mention their sharp ears)}, ageing at one tenth a human's rate after reaching the age of eighteen. They descendants of those Ancestral Elves who crossed over from the Orient to colonize the Occident, making them a sister-race to the High Elves and Humans, who also directly evolved out of those Ancestral Elves. As an added bonus, Wood Elven women and men are well-known for being rivals to their High Elven counterparts in regards to beauty. The average adult Wood Elf is about 190 centimetres \textit{(6 feet and 3 inches)} tall, regardless of gender \textit{(albeit men to tend to be slightly taller on average)}.\newline
\begin{tabular}{|c|c|c|}
\hline
 & \textbf{Min} & \textbf{Max} \\ \hline
\textbf{Strength} & 8 & 18 \\ \hline
\textbf{Endurance} & 7 & 17 \\ \hline
\textbf{Dexterity} & 9 & 19 \\ \hline
\textbf{Intelligence} & 8 & 18 \\ \hline
\textbf{Willpower} & 8 & 18 \\ \hline
\textbf{Charisma} & 8 & 18 \\ \hline
\end{tabular}\newline
\DimorphismBB{Wood Elves}{2}{Strength}{2}{Charisma} \MammalRace{Wood Elves}\newpage
\section{Dark Elves}
\includegraphics{Dark_Elves}\newline
\textbf{Dark Elves} are a race without a unified country, living in underground clans in dungeon-cities below the the ground, though also being present in several surface states - namely the Kingdoms of Etrand and Froturn - in high numbers as diaspora. They have rather long lifespans, theoretically ten times that of a human \textit{(and they're also typically taller than humans, not to mention their sharp ears and blue-ish grey skin)}, ageing at one tenth a human's rate after reaching the age of eighteen. They descendants of Wood Elves and High Elves who were banished after experimenting with the dark arts, creating their own exodus, where the environment - and the new religion - has altered their appearence. The average adult Dark Elf is about 180 centimetres \textit{(5 feet and 11 inches)} tall, regardless of gender \textit{(albeit men to tend to be slightly taller on average)}.\newline
\begin{tabular}{|c|c|c|}
\hline
 & \textbf{Min} & \textbf{Max} \\ \hline
\textbf{Strength} & 8 & 18 \\ \hline
\textbf{Endurance} & 7 & 17 \\ \hline
\textbf{Dexterity} & 10 & 20 \\ \hline
\textbf{Intelligence} & 8 & 18 \\ \hline
\textbf{Willpower} & 8 & 18 \\ \hline
\textbf{Charisma} & 8 & 18 \\ \hline
\end{tabular}\newline
\DimorphismBB{Dark Elves}{1}{Strength}{1}{Charisma} \MammalRace{Dark Elves}\newpage
\section{Half-Elves}
\includegraphics{Half_Elves}\newline
\textbf{Half-Elves} are hybrids of Humans and any of the aforementioned Elven races \textit{(High Elves, Wood Elves, Dark Elves)}. To be specific, this race only includes only so-called \SoCalled{clean hybrids} - as in, people with 45-55\% Human blood and 45-55\% Elven blood. People where one side dominates - whether it's the Human side or Elven side - are classified as \SoCalled{unclean hybrids}, and thus get shoehorned into the Human or Elven race. Their lifespan is halfway between that of a Human and Elf, thus around five times that of a human's, ageing at one fifth the rate after reaching the age of eighteen.\newline
\begin{tabular}{|c|c|c|}
\hline
 & \textbf{Min} & \textbf{Max} \\ \hline
\textbf{Strength} & 8 & 18 \\ \hline
\textbf{Endurance} & 8 & 18 \\ \hline
\textbf{Dexterity} & 9 & 19 \\ \hline
\textbf{Intelligence} & 8 & 18 \\ \hline
\textbf{Willpower} & 8 & 18 \\ \hline
\textbf{Charisma} & 8 & 18 \\ \hline
\end{tabular}\newline
\DimorphismBB{Half-Elves}{1}{Strength}{1}{Charisma} \MammalRace{Half-Elves}\newpage
\section{Half-Orcs}
\includegraphics{Half_Orcs}\newline
\textbf{Half-Orcs} are hybrids of Orcs and any of the other Elven races \textit{(Humans, High Elves, Wood Elves, Dark Elves)} - from a gameplay perspective, we don't distinguish between Orc-Human and Orc-Elf hybrids, as Orkish traits tend to be the most salient for both. To be specific, this race only includes only so-called \SoCalled{clean hybrids}- as in, people with 45-55\% Orkish blood and 45-55\% Human/Elven blood. People where one side dominates - whether it's the Orkish side or Human/Elven side - are classified as \SoCalled{unclean hybrids}, and thus get shoehorned into the Orkish or Human/Elven race. Their lifespan is halfway between the parent races, which would mean around 200 years for Human-Orc hybrids, 400 years for Orc-Elf hybrids.\newline
\begin{tabular}{|c|c|c|}
\hline
 & \textbf{Min} & \textbf{Max} \\ \hline
\textbf{Strength} & 9 & 19 \\ \hline
\textbf{Endurance} & 9 & 19 \\ \hline
\textbf{Dexterity} & 8 & 18 \\ \hline
\textbf{Intelligence} & 8 & 18 \\ \hline
\textbf{Willpower} & 8 & 18 \\ \hline
\textbf{Charisma} & 7 & 17 \\ \hline
\end{tabular}\newline
\MammalRace{Half-Orcs}\newpage
\section{Orcs}
\includegraphics{Orks}\newline
\textbf{Orcs} are an oddball among the group of Elven races - just like Humans, they look nothing like Elves, despite being genetically descended from the Ancestral Elves, albeit indirectly: the Orcs are descendants of a group of Wood Elves who were exiled for their war crimes and subsequent rebellion against their own country's authority, first tortured then banished into the frozen wastelands of Brutang with the hopes of them dying from exposure or being killed by the native Goblins and Ogres. Instead, these exiled renegades came to dominate the natives, but for some unknown reason, their children and descendants all became green-skinned beings. Even the Orcs themselves aren't sure today how did their ancestors transition from being Wood Elves to being Orcs - perhaps a curse?\newline
Orcs have a lifespan roughly three times that of humans, which means that they age at one third of a human's rate after reaching the age of eighteen. Orcs have an average height of 210 centimetres, or roughly 6 feet and 11 inches. Orcs are well-known for one of the most muscular races, albeit their green skin, shorter lifespan and ugly faces give people the impression that they have more in common with the Goblins and Ogres than the Wood Elves they supposedly descended from. Their true origins still remain a mystery, as it remains impossible to tell legend from reality.\newline
\begin{tabular}{|c|c|c|}
\hline
 & \textbf{Min} & \textbf{Max} \\ \hline
\textbf{Strength} & 10 & 20 \\ \hline
\textbf{Endurance} & 10 & 20 \\ \hline
\textbf{Dexterity} & 7 & 17 \\ \hline
\textbf{Intelligence} & 8 & 18 \\ \hline
\textbf{Willpower} & 8 & 18 \\ \hline
\textbf{Charisma} & 6 & 16 \\ \hline
\end{tabular}\newline
\MammalRace{Orcs}\newpage
\section{Ogres}
\includegraphics{Ogres}\newline
\textbf{Ogres} are a race indigenous to Artograch. Green-skinned, very muscular, very tall \textit{(average height of 250 centimetres, or 8 feet and 2.4 inches)} and very solitary, Ogres have an average lifespan of roughly twice of a human's. Before the arrival of the Orcs, Ogres had a symbiotic relationship with the Goblins: the thriftly Goblins visited foreign places to bring back foreign goods to enrich the land and the ogres, while the Ogres protected the land with their strong hands and xenophobia, giving the Goblins a good base of operations. This all changed with the arrival of the Orcs, who enslaved the Ogres using them for forced labour until they began fighting for emancipation, winning that fight in the more liberal-minded clans. Despite no longer being collectively enslaved by most Orkish clans anymore, Ogres are still somewhat of outsiders in Orkish societies, preferring to stay out of clan politics and abstain from aspiring for high positions in them.\newline
\begin{tabular}{|c|c|c|}
\hline
 & \textbf{Min} & \textbf{Max} \\ \hline
\textbf{Strength} & 11 & 21 \\ \hline
\textbf{Endurance} & 11 & 21 \\ \hline
\textbf{Dexterity} & 7 & 17 \\ \hline
\textbf{Intelligence} & 7 & 17 \\ \hline
\textbf{Willpower} & 7 & 17 \\ \hline
\textbf{Charisma} & 6 & 16 \\ \hline
\end{tabular}\newline
\MammalRace{Ogres}\newpage
\section{Goblins}
\includegraphics{Goblins}\newline
\textbf{Goblins} are a race indigenous to Artograch. Green-skinned, small \textit{(average of 110 cm, or 3 feet and 7.3 inches)}, agile and undemanding  they can live and thrive in just about any place: from the hot and scorching desert to the arctic wasteland. Thanks to their small size, they do not need big houses. Their small and agile fingers are good to hunting smaller animals and using bows. Their in-born night vision allows them to see in the darkness. They are much quicker and less noisy than the Orcs, which why is why the Orcish tribes of Brutang love employing them as scouts.\newline
As an undemanding race that can thrive just about anywhere, they can be found in nearly all corners of Artograch, although their main domicile is in Brutang, which they share with the Orcs and Ogres. Before the arrival of the Orcs in Brutang, they had a symbiotic relationship with the Ogres: the thriftly Goblins visited foreign places to bring back foreign goods to enrich the land and the ogres, while the Ogres protected the land with their strong hands and xenophobia, giving the Goblins a good base of operations. This all changed with the arrival of the Orcs, with the Goblins of Brutang having to serve new - Orcish - masters. They earned their emancipation far quicker than the Ogres, as they provid their worth as merchants, scouts, diplomats and artisans fairly early on. Outside of Brutang, their tribes are mostly nomadic, making a living of stealing and trading. These goblin tribes often create caravans, wandering with their products, caught monsters and slaves, intending to sell them to someone. These caravans are often followed by scouts on wolfback to provide cover and to look for villages to pillage.\newline
The Goblin race contains three subraces: Night Goblins, Wood Goblins and Hobgoblins, with \textbf{Night Goblins} being the sneaky ones who prefer to dwell in caves and lack claustrophobia and taphophobia of any kind; \textbf{Wood Goblins} who feel most at home when under the open sky and thus have never heard agoraphobia; and \textbf{Hobgoblins}, silghtly taller and sturdier on average than other Goblins, characterized by an affinity for rocky lands and extreme wastelands, getting as close to extremophilia as a humanoid mammal can realistically get. There is no real class distinction or caste system to keep these three subspecies apart, so interbreeding betweem them is very common.\newline
\begin{tabular}{|c|c|c|}
\hline
 & \textbf{Min} & \textbf{Max} \\ \hline
\textbf{Strength} & 8 & 18 \\ \hline
\textbf{Endurance} & 8 & 18 \\ \hline
\textbf{Dexterity} & 8 & 18 \\ \hline
\textbf{Intelligence} & 8 & 18 \\ \hline
\textbf{Willpower} & 8 & 18 \\ \hline
\textbf{Charisma} & 7 & 17 \\ \hline
\end{tabular}\newline
Hobgoblins get a \Bonus{2} bonus to Strength, Night Goblins get a \Bonus{2} to Dexterity, Wood Goblins get a \Bonus{1} to both Strength and Dexterity. \MammalRace{Goblins}\newpage
\section{Halflings}
\includegraphics{Hobbits}\newline
\textbf{Halflings} are a race indigenous to Artograch, with an average lifespan of 150 years, and average height of 130 centimetres or 4 feet and 3.1 inches. Due to the fact that they are rather short-statued humanoid mammals and speak a language related to Dwarven, they are sometimes grouped in with the Dwarves and Gnomes as the \SoCalled{Norlokian races}, even though they are incapable of interbreeding with each other, and thus may not even be biologically related, despite the superficial resemblance.\newline
Halflings are a race of smart, inventive survivors and adventurers, known for their curiousity and boldness. They are easily seduced by riches, but they prefer spending money, for they are a race definitely not made famous by their frugality. Their skin is ruddy, their hair is typically either black, brown or red, and either straight or curly \textit{(never wavy)}, and their eyes are usually brown or black. Halfling men very often grow out their sideburns, but seldom do they also grow a beard or mustache. They prefer dressing for comfort and practicality, and in spite of their love of riches, they typically shun jewellery. They love to enjoy the little things in life, such as smoking their pipes.\newline
\begin{tabular}{|c|c|c|}
\hline
 & \textbf{Min} & \textbf{Max} \\ \hline
\textbf{Strength} 7 8 & 17 \\ \hline
\textbf{Endurance} & 8 & 18 \\ \hline
\textbf{Dexterity} & 9 & 19 \\ \hline
\textbf{Intelligence} & 8 & 18 \\ \hline
\textbf{Willpower} & 8 & 18 \\ \hline
\textbf{Charisma} & 8 & 18 \\ \hline
\end{tabular}\newline
\DimorphismBB{Halflings}{1}{Strength and Endurance}{1}{Dexterity and Charisma} \MammalRace{Halflings}\newpage
\section{Gnomes}
\includegraphics{Gnomes}\newline
\textbf{Gnomes} are a race indigenous to Artograch, with an average lifespan of 300 years, and average height of 100 centimetres or 3 feet and 3.37 inches. Due to the fact that they are rather short-statued humanoid mammals and speak the Dwarven language, they are sometimes grouped in with the Dwarves and Halflings as the \SoCalled{Norlokian races}, even though they are incapable of interbreeding with each other, and thus may not even be biologically related, despite the superficial resemblance.\newline
The Gnomes are a race like no other. While the Dwarves, Humans, Elves and even the self-proclaimed \SoCalled{spiritual} Lizardmen love gossiping about just about everything, from politics to the petty personal matters of anyone who isn’t present, the gnomes are a whole different world. They discuss science and magic, not tratsch. Gnomish genius knows no limits. They construct complex machines, golems, clockwork engines, and even more. In addition to mastering mechanics, gnomes also excel at alchemy. While the Elves - especially the Wood Elves - are unrivalled masters of Herbalism and potion-making, the Gnomes prefer to explore the more obscure yet powerful side of Alchemy: transmutation. The art of turning worthless rock into diamonds and mere paper into gold is a dangerous art mastered only by the bravest of souls.\newline
The gnomes have been living side-by-side with the dwarves for 5000 years, and against all odds, they kept their own culture, as a separate race - which is understandable, given how they are biologically incapable of interbreeding with Dwarves. Although gnomes are widely employed as alchemists, inventors and artillery engineers, they prefer to live amongst their own kind. They are fond of animals, gemstones, jokes and pranks. They like learning things in a personal way, and they always try to build up things by a new way.\newline
\begin{tabular}{|c|c|c|}
\hline
 & \textbf{Min} & \textbf{Max} \\ \hline
\textbf{Strength} & 6 & 16 \\ \hline
\textbf{Endurance} & 6 & 16 \\ \hline
\textbf{Dexterity} & 10 & 20 \\ \hline
\textbf{Intelligence} & 10 & 20 \\ \hline
\textbf{Willpower} & 8 & 18 \\ \hline
\textbf{Charisma} & 8 & 18 \\ \hline
\end{tabular}\newline
\DimorphismBB{Gnomes}{1}{Strength and Endurance}{1}{Dexterity and Charisma} \MammalRace{Gnomes}\newpage
\section{Dwarves}
\includegraphics{Dwarves}\newline
\textbf{Dwarves} are a race indigenous to Artograch, with an average lifespan of 200 years, and average height of 120 centimetres or 3 feet and 11.2 inches. Due to the fact that they are rather short-statued humanoid mammals and speak a language related to Halfling, they are sometimes grouped in with the Halflings and Gnomes as the \SoCalled{Norlokian races}, even though they are incapable of interbreeding with each other, and thus may not even be biologically related, despite the superficial resemblance.\newline
The Dwarves are one of the most easily recognizable races of Artograch. Short but muscular statue, crude humour, mild xenophobia, fondness for beer and women, you will almost instantly recognize the dwarf. Having been a closed people for millennia, dwarves are not very trusting with foreigners \textit{(other than the gnomes whom they don’t really consider foreigners at all)}, but they are generous with the ones who earn their trust. They are also known for their talent at blacksmithing, mining and melee weapons. Dwarven steel is universally considered the best material for making both weapons and armour, and no one can deny the beauty of dwarven-made ornaments that decorate armour and weapons that would be already considered nicely-made even without them. Dwarves’ skin colour can vary from yellowish brown to pale white. A dwarf's hair colour can be blond, red, brown or black. In Dwarven society, if a male does not have a beard, he isn't considered male. Traditionally, the dwarves lived in clans that constantly bickered amongst each other, but historical events six centuries ago changed all of that.\newline
\begin{tabular}{|c|c|c|}
\hline
 & \textbf{Min} & \textbf{Max} \\ \hline
\textbf{Strength} & 8 & 18 \\ \hline
\textbf{Endurance} & 10 & 20 \\ \hline
\textbf{Dexterity} & 8 & 18 \\ \hline
\textbf{Intelligence} & 8 & 18 \\ \hline
\textbf{Willpower} & 8 & 18 \\ \hline
\textbf{Charisma} & 7 & 17 \\ \hline
\end{tabular}\newline
\DimorphismBB{Dwarves}{1}{Strength}{1}{Charisma} \MammalRace{Dwarves}\newpage
\section{Lizardmen}
\includegraphics{Average_Lizardman}\newline
\textbf{Lizardmen} are a cold-blooded reptiles \textit{(varanids to be specific)}, mainly characterized by thick scales, sharp thinking and high level of endurance. They populate mainly swamps and sylvan mountains. They tolerate all sorts of climates and terrains, with the exception of very low temperature: cold is their weak point. Unlike normal lizards, they mainly eat meat, but they never prey on sentiment beings \textit{(like Humans)}, they also like fruits, but they never eat grains/cereals \textit{(nor alcoholic beverage made out of it)}. Weaker lizardmen live at least 900 years, while the stronger could live up to 1300 years. The colour of the scales can vary regardless of gender. Their eyes can be green, yellow or orange. On average, they weight 70 kilograms or 155 pounds and are 175 centimetres or 5 feet and 9 inches tall.\newline
Their limbs are muscular and they can endure physically demand work for long, although not as nimble or agile as elves, they can endure running for longer amount of time than Elves or Orcs, either up to hills or on the flat ground. They are neutral to other races, they are not expansive, they are conservative and tradition-respecting. They do not engage in combat with those who doesn't attack first*, although in battle the enemy is clearly enemy for them, they do not hold them back when fighting someone who wants their head. Their warriors are more fond of agility and speed than raw strength, although they don’t underestimate the last either. They usually use spears or missile weapons, but there are some who prefer swords or heavy weapons \textit{(such as battle axes)}. They also use their strong tails in battle! Due to their thick skin, they do not wear armour, and they wear clothes rarely - most of the time only trousers.\newline
They live in smaller or bigger communities, but it’s not rare for the younger to decide to become an adventurer, travelling warrior or mercenary. There are some of them who are spellcasters, mostly shamans who help the community with their magic and teach them in the ways of the religion. There are also some of them who decide to be mages. There are some who aren’t born to their kind. They are usually sociable, but not always accepted due the way they look.\newline
\begin{tabular}{|c|c|c|}
\hline
 & \textbf{Min} & \textbf{Max} \\ \hline
\textbf{Strength} & 9 & 19 \\ \hline
\textbf{Endurance} & 10 & 20 \\ \hline
\textbf{Dexterity} & 10 & 20 \\ \hline
\textbf{Intelligence} & 9 & 19 \\ \hline
\textbf{Willpower} & 8 & 18 \\ \hline
\textbf{Charisma} & 4 & 14 \\ \hline
\end{tabular}\newline
\LizardRace{Lizardmen}\newpage
\chapter{Non-Playable Primary Races}
\textbf{Non-PLayable races} are races that, while not explicitly forbidden, are not recommended for usage as player characters for various reasons. They will not be playable in any of the video game adaptations either, and they are not recommended for tabletop usage either, other than as NPCs controlled by the game master. The reasons for this vary a lot, but boil down to these so-called \SoCalled{non-playable races} being...
\begin{itemize}
	\item ...unballanced, thus giving players who control them an unfair \Parentheses{dis}advantage.
	\item ...not as fleshed out as the so-called so-called \SoCalled{playable races}
	\item ...not fitting into the traditional formula of characters having attributes, classes and equipment. As a rule of thumb, non-bipedal races are automatically non-playable in the Occident.
\end{itemize}
As previously stated, while playing as characters from these races is not explicitly forbidden per se, it is genuinely not recommended for game masters to let players create characters from these races. These races will also be non-playable in any future video game adaptations.
\section{Nereids}
\textbf{Nereids} - also called \textbf{Mermaids and Mermen} - are the politically dominant race of the Free City of Gabyr, in spite of only constituting 26.72\% of its population.\newline
\begin{tabular}{|c|c|c|}
\hline
 & \textbf{Min} & \textbf{Max} \\ \hline
\textbf{Strength} & 8 & 18 \\ \hline
\textbf{Endurance} & 10 & 20 \\ \hline
\textbf{Dexterity} & 10 & 20 \\ \hline
\textbf{Intelligence} & 10 & 20 \\ \hline
\textbf{Willpower} & 10 & 20 \\ \hline
\textbf{Charisma} & 10 & 20 \\ \hline
\end{tabular}\newline
Nereids look like mammals above the hip, but hash fish bodies under the hips. This means that they have two hands - allowing them to use two one-handed weapons or a single two-handed weapon at a time, or a single one-handed weapon and a shield. They have their torso and their head, but lack legs - instead, they have a large tail, whose loss is just as lethal as the loss of their torso or head. They have the same vitals as other mortal creatures, but they cannot be infected with vampirism or theriantropy.\newpage
\section{Demons}
\section{Angels}
\section{Dragons}
\section{Griffins}
\section{Winged Cobras}
\chapter{Secondary Races}
\textbf{Secondary races} are races that people aren't usually born into, but instead are turned into, usually via some sort of magic. Most secondary races are undead, and their condition is caused by a symbiotic relationship between their own animated corpse and a tiny creature that is responsible for the animation: the only cure to such a condition is a final death.
\section{Liches}
\textbf{Liches} are sentient undead creatures who resemble skeletons, but are in fact a power to be reckoned with, as the ritual of transformation in which liches are made is lethal to anyone who doesn't possess sufficient amount of magical power to have all of their body \Parentheses{save for their bones} burn away, yet live. In exchange for the efforts taken prior to the ritual and the excruciating pain sustained during the ritual, they gain eternal life, increased magical powers and being freed from various other mortal restraints, such as mortal needs and mortal desires.\newline
\begin{tabular}{|c|c|c|}
\hline
 & \textbf{Bonus/Malus} \\ \hline
\textbf{Strength} & \textit{unchanged} \\ \hline
\textbf{Endurance} & \textit{unchanged}  \\ \hline
\textbf{Dexterity} & \textit{unchanged}  \\ \hline
\textbf{Intelligence} & \BonusS{6} \\ \hline
\textbf{Willpower} & \BonusS{6} \\ \hline
\textbf{Charisma} & \MalusS{4} \\ \hline
\end{tabular}\newline
\UndeadRace{Liches}\newpage
\section{Vampires}
\textbf{Vampires} are sentient undead creatures who - unlike Liches - largely retain their original mortal looks, albeit with a few characteristics \Parentheses{paler skin, unnatural eye colours, more prominent, pointy and sharp canine teeth}. Unlike lichdoom - which is a condition caused by a ritual - vampirism is caused by a parasitical organism infecting a living sentient mammal, killing it and reanimating its body, sharing control over the body with the original brains and soul, establishing a symbiotic relationship, just like with theriantropes. The parasite demands blood, and will drive the vampire to utter madness \Parentheses{and eventually bodily decay} if it is not delivered, but in exchange, it grants the vampire eternal life, liberation from mortal needs for food and water, as well as superior physical power. In spite of being undead creatures who are supposed to be clinically dead, those who are turned vampires prematurely will still physically mature into adults capable of sexual reproduction: the parsite inside them makes sure, that as long as they are fed with enough blood at regular intervals, some of their mortal functions will continue to work, such as the ones responsible for growing into an adult, for appreciating the taste of mortal food and beverages, and for sexual reproduction - but woe be to a vampire who neglects feeding, as they will begin to experience both bodily decay and mental degeneration.\newline
So-called \SoCalled{pure-blooded vampires} \Parentheses{the offsprings of two vampires} and \SoCalled{half-vampires} \Parentheses{the offsprings of a mortal and vampire} are immune to the harsh rays of the sun from birth, while to regular vampires, the scorching touch of the sun is lethal for roughly three centuries, during which they gradually build up an immunity and gain resistance to it.\newline
The parasite that causes vampirism is one very prone to mutation, which explains the high diversity of vampire clans with different powers unique to them.\newline
\begin{tabular}{|c|c|c|}
\hline
 & \textbf{Bonus/Malus} \\ \hline
\textbf{Strength} & \BonusS{4} \\ \hline
\textbf{Endurance} & \BonusS{4}  \\ \hline
\textbf{Dexterity} & \textit{unchanged}  \\ \hline
\textbf{Intelligence} & \textit{unchanged} \\ \hline
\textbf{Willpower} & \textit{unchanged} \\ \hline
\textbf{Charisma} & \textit{unchanged} \\ \hline
\end{tabular}\newline
\UndeadRace{Vampires} Instead, they activate a need for Blood, their own unique vital shared with Theriantropes.\newpage
\section{Theriantropes}
\textbf{Theriantropes} are sentient undead creatures who - unlike Liches - largely retain their original mortal looks, albeit they take on certain animal motiffs, such as the smell of a wild animal and character tics associated with said animal. The condition of theriantropy shares a lot with Vampirism, such as the fact that it is caused by a parasite that kills its host and reanimates the host's corpse as an undead creature, allowing them retain free will, but still influencing them, giving them powers in exchange for blood. On every full moon, young theriantropes go through an involuntary transformation that makes them lose their free will, turning them into berserkers driven by lust for blood - they also go through a physical transformation, from their original form into a bipedal animal \Parentheses{wolf, bear, boar, rat, lion, tiger, etc.}. As theriantropes grow older and stronger, they gradually learn to control their condition, transforming at will, and retaining their free will even when transformed.\newline
Just like how vampirism comes in various forms, theriantropy does to come in many forms, with werewolves \Parentheses{also known as lycantropes} being far by the most common type of theriantrope. Other kinds of theriantropes include werebears, wereboars, wererats, werelions and weretigers. The differences between these were-animals is largely cosmetic, with the common denominator being the fact that they are undead beings stronger than regular mortals, go through involuntary transformations when young, and obstain the smell and character tics of the associated animal.\newline
Just like vampires, theriantropes are a special kind of undead that - through their symbiosis with the parasite - can still get to mingle with mortals, appreciate the taste of mortal food and beverages, reproduce sexually and grow into adulthood when turned prematurely when fed with sufficient amount of blood at regular intervals. However, lack of blood causes involuntary transformations and loss of control first, then the decaying of their body.\newline
\begin{tabular}{|c|c|c|}
\hline
 & \textbf{Bonus/Malus} \\ \hline
\textbf{Strength} & \BonusS{4} \\ \hline
\textbf{Endurance} & \BonusS{4}  \\ \hline
\textbf{Dexterity} & \textit{unchanged}  \\ \hline
\textbf{Intelligence} & \textit{unchanged} \\ \hline
\textbf{Willpower} & \textit{unchanged} \\ \hline
\textbf{Charisma} & \textit{unchanged} \\ \hline
\end{tabular}\newline
\UndeadRace{Theriantropes} Instead, they activate a need for Blood, their own unique vital shared with Vampires.\newpage
\chapter{Classes}
\section{Fighter Type Classes}
\subsection{The \textit{(Occidental)} Mercenary}
\subsection{The Headhunter}
\subsection{The Footman}
\subsection{The Sharpshooter}
\section{Rogue Type Classes}
\subsection{The Thief}
\subsection{The Assassin}
\subsection{The Smuggler}
\subsection{The Scout}
\section{Magician Type Classes}
\subsection{The Mage}
\subsection{The Battlemage}
\subsection{The Spellthief}
\subsection{The Warlock}
\section{Cleric Type Classes}
\subsection{The Priest}
\subsection{The Inquisitor}
\subsection{The Knight}
\subsection{The Monk}
\section{Druid Type Classes}
\subsection{The Shaman}
\subsection{The Ranger}
\subsection{The Bard}
\subsection{The Herbalist}
\chapter{Feats}
\section{Combat Feats}
\subsection{Armour Conditioning}
\begin{table}[!ht]
\centering
\FeatIII{Light}{Medium}{Heavy}{Armour Conditioning}{{\import{images/Feats/}{images/Feats/Armour1.pdf_tex}}}{{\import{images/Feats/}{images/Feats/Armour2.pdf_tex}}}{{\import{images/Feats/}{images/Feats/Armour3.pdf_tex}}}
\end{table}
\textbf{Requirements:}
\begin{itemize}
	\item \textbf{Light:} Not Monk.
	\item \textbf{Medium:} Fighter-type class, Scout, Battlemage, Priest, Inquisitor or Knight.
	\item \textbf{Heavy:} Fighter-type class, Battlemage or Knight.
\end{itemize}
\textbf{Effects:}
\begin{itemize}
	\item \textbf{Untrained:} The character feels uncomfortable in armour, and a \Malus{50\%} to movement speed when wearing any sort of armour. When any calculations are made that involve Dexterity, wearing any sort of armour counts as a \Malus{2} to Dexterity. Magic-using characters also cannot cast spells when wearing any sort of armour.
	\item \textbf{Light:} The character feels comfortable with Light Armour \Parentheses{Gambeson and Leather Armour}, uncomfortable in anything heavier, thus incurring a \Malus{50\%} to movement speed when wearing any armour other than Gambeson and Leather Armour. When any calculations are made that involve Dexterity, wearing Medium or Heavy armour counts as a \Malus{2} to Dexterity. Magic-using characters also cannot cast spells when wearing Medium or Heavy Armour.
	\item \textbf{Medium:} The character feels comfortable with Medium Armour \Parentheses{Chainmail and Scalemail}, uncomfortable in anything heavier, thus incurring a \Malus{50\%} to movement speed when wearing Heavy Armour. When any calculations are made that involve Dexterity, wearing Heavy armour counts as a \Malus{2} to Dexterity. Magic-using characters also cannot cast spells when wearing Heavy Armour.
	\item \textbf{Heavy:} The character feels comfortable with Heavy Armour \Parentheses{Breastplates and Platemail}, incurring no malluses when wearing them. Magic-using characters \textbf{can} cast spells in any type of armour.
\end{itemize}\newpage
\subsection{Swords}
\begin{table}[!ht]
\centering
\FeatIII{Basic}{Advanced}{Expert}{Swords}{{\import{images/Feats/}{images/Feats/Blade1.pdf_tex}}}{{\import{images/Feats/}{images/Feats/Blade2.pdf_tex}}}{{\import{images/Feats/}{images/Feats/Blade3.pdf_tex}}}
\end{table}
\textbf{Requirements:}
\begin{itemize}
	\item \textbf{Basic:} Cannot be Cleric-type or Druid-type class, except Knight, Ranger or Bard.
	\item \textbf{Advanced:} Fighter or Rogue-type class, or Battlemage, Spellthief, Knight, Ranger or Bard.
	\item \textbf{Expert:} Fighter or Rogue-type class, or Battlemage, Spellthief, Knight, Ranger or Bard.
\end{itemize}
\textbf{Effects:}
\begin{itemize}
	\item \textbf{Untrained:} The character always deals minimal damage with swords, no need to roll the dice. They also get a \Malus{2} to Dexterity when it's being counted when using swords.
	\item \textbf{Basic:} The character suffers no bonuses or maluses when using swords.
	\item \textbf{Advanced:} The character reiceves a \Bonus{25\%} to all damage done by swords, and a \Bonus{25\%} to chance to hit or cause critical damage.
	\item \textbf{Expert:} The character reiceves a \Bonus{50\%} to all damage done by swords, and a \Bonus{50\%} to chance to hit or cause critical damage.
\end{itemize}\newpage
\subsection{Bludgeoners}
\begin{table}[!ht]
\centering
\FeatIII{Basic}{Advanced}{Expert}{Bludgeoners}{{\import{images/Feats/}{images/Feats/Blunt1.pdf_tex}}}{{\import{images/Feats/}{images/Feats/Blunt2.pdf_tex}}}{{\import{images/Feats/}{images/Feats/Blunt3.pdf_tex}}}
\end{table}
\textbf{Requirements:}
\begin{itemize}
	\item \textbf{Basic:} Any class.
	\item \textbf{Advanced:} Any class.
	\item \textbf{Expert:} Fighter, Rogue or Cleric type class, or Battlemage, Spellthief, Ranger or Bard.
\end{itemize}
\textbf{Effects:}
\begin{itemize}
	\item \textbf{Untrained:} The character always deals minimal damage with so-called \SoCalled{striking weapons} \Parentheses{clubs/cudgels, maces, hammers, axes}, no need to roll the dice. They also get a \Malus{2} to Dexterity when it's being counted when using bludgeoners.
	\item \textbf{Basic:} The character suffers no bonuses or maluses when using so-called \SoCalled{striking weapons} \Parentheses{clubs/cudgels, maces, hammers, axes}.
	\item \textbf{Advanced:} The character reiceves a \Bonus{25\%} to all damage done by \SoCalled{striking weapons} \Parentheses{clubs/cudgels, maces, hammers, axes}, and a \Bonus{25\%} to chance to hit or cause critical damage.
	\item \textbf{Expert:} The character reiceves a \Bonus{50\%} to all damage done by \SoCalled{striking weapons} \Parentheses{clubs/cudgels, maces, hammers, axes}, and a \Bonus{50\%} to chance to hit or cause critical damage.
\end{itemize}\newpage
\subsection{Polearms}
\begin{table}[!ht]
\centering
\FeatIII{Basic}{Advanced}{Expert}{Polearms}{{\import{images/Feats/}{images/Feats/Stick1.pdf_tex}}}{{\import{images/Feats/}{images/Feats/Stick2.pdf_tex}}}{{\import{images/Feats/}{images/Feats/Stick3.pdf_tex}}}
\end{table}
\textbf{Requirements:}
\begin{itemize}
	\item \textbf{Basic:} Any class.
	\item \textbf{Advanced:} Any class.
	\item \textbf{Expert:} Fighter or Rogue type class, or Battlemage, Spellthief, Knight, Ranger or Bard.
\end{itemize}
\textbf{Effects:}
\begin{itemize}
	\item \textbf{Untrained:} The character always deals minimal damage with so-called \SoCalled{polearms} \Parentheses{quarterstaffs, spears, pikes, halberds, pollaxes, glaives, voulges and bills}, no need to roll the dice. They also get a \Malus{2} to Dexterity when it's being counted when using polearms.
	\item \textbf{Basic:} The character suffers no bonuses or maluses when using so-called \SoCalled{polearms} \Parentheses{quarterstaffs, spears, pikes, halberds, pollaxes, glaives, voulges and bills}.
	\item \textbf{Advanced:} The character reiceves a \Bonus{25\%} to all damage done by \SoCalled{polearms} \Parentheses{quarterstaffs, spears, pikes, halberds, pollaxes, glaives, voulges and bills}, and a \Bonus{25\%} to chance to hit or cause critical damage.
	\item \textbf{Expert:} The character reiceves a \Bonus{50\%} to all damage done by \SoCalled{polearms} \Parentheses{quarterstaffs, spears, pikes, halberds, pollaxes, glaives, voulges and bills}, and a \Bonus{50\%} to chance to hit or cause critical damage.
\end{itemize}\newpage
\subsection{Bows}
\begin{table}[!ht]
\centering
\FeatIII{Basic}{Advanced}{Expert}{Bows}{{\import{images/Feats/}{images/Feats/Ranged1.pdf_tex}}}{{\import{images/Feats/}{images/Feats/Ranged2.pdf_tex}}}{{\import{images/Feats/}{images/Feats/Ranged3.pdf_tex}}}
\end{table}
\textbf{Requirements:}
\begin{itemize}
	\item \textbf{Basic:} Cannot be Cleric-type or Druid-type class, except Ranger or Bard.
	\item \textbf{Advanced:} Fighter or Rogue-type class, or Battlemage, Spellthief, Ranger or Bard.
	\item \textbf{Expert:} Sharpshooter class.
\end{itemize}
\textbf{Effects:}
\begin{itemize}
	\item \textbf{Untrained:} The character always deals minimal damage with bows, no need to roll the dice. They also get a \Malus{2} to Dexterity when it's being counted when using bows. In video game adaptations, this should halve the effective range of all bows.
	\item \textbf{Basic:} The character suffers no bonuses or maluses when using bows.
	\item \textbf{Advanced:} The character reiceves a \Bonus{25\%} to all damage done by bows, and a \Bonus{25\%} to chance to hit or cause critical damage. In video game adaptations, this should also increase range.
	\item \textbf{Expert:} The character reiceves a \Bonus{50\%} to all damage done by bows, and a \Bonus{50\%} to chance to hit or cause critical damage. In video game adaptations, this should also increase range.
\end{itemize}\newpage
\subsection{Crossbows}
\begin{table}[!ht]
\centering
\FeatIII{Basic}{Advanced}{Expert}{Crossbows}{{\import{images/Feats/}{images/Feats/Crossbow1.pdf_tex}}}{{\import{images/Feats/}{images/Feats/Crossbow2.pdf_tex}}}{{\import{images/Feats/}{images/Feats/Crossbow3.pdf_tex}}}
\end{table}
\textbf{Requirements:}
\begin{itemize}
	\item \textbf{Basic:} Cannot be Cleric-type or Druid-type class, except Ranger or Bard.
	\item \textbf{Advanced:} Fighter or Rogue-type class, or Battlemage, Spellthief, Ranger or Bard.
	\item \textbf{Expert:} Sharpshooter class or Rogue-type class.
\end{itemize}
\textbf{Effects:}
\begin{itemize}
	\item \textbf{Untrained:} The character needs to skip a turn before using a crossbow to reload every single time they want to shoot. They also get a \Malus{2} to Dexterity when it's being counted when using crossbows. In video game adaptations, this should halve the effective range of all bows.
	\item \textbf{Basic:} The character suffers no bonuses or maluses when using crossbows, and can reload fast enough to not to skip a turn doing it.
	\item \textbf{Advanced:} The character reiceves a \Bonus{25\%} to all damage done by crossbows, and a \Bonus{25\%} to chance to hit or cause critical damage. In video game adaptations, this should also increase range.
	\item \textbf{Expert:} The character reiceves a \Bonus{50\%} to all damage done by crossbows, and a \Bonus{50\%} to chance to hit or cause critical damage. In video game adaptations, this should also increase range.
\end{itemize}\newpage
\subsection{Firearms}
\begin{table}[!ht]
\centering
\FeatIII{Basic}{Advanced}{Expert}{Firearms}{{\import{images/Feats/}{images/Feats/Gun1.pdf_tex}}}{{\import{images/Feats/}{images/Feats/Gun2.pdf_tex}}}{{\import{images/Feats/}{images/Feats/Gun3.pdf_tex}}}
\end{table}
\textbf{Requirements:}
\begin{itemize}
	\item \textbf{Basic:} Cannot be Cleric-type or Druid-type class, except Ranger or Bard, unless character's race is Gnome or Dwarf.
	\item \textbf{Advanced:} Cannot be Cleric-type or Druid-type class, except Ranger or Bard, unless character's race is Gnome or Dwarf.
	\item \textbf{Expert:} Sharpshooter or Mercenary class, Magician-type class, or any class if character's race is Gnome or Dwarf.
\end{itemize}
\textbf{Effects:}
\begin{itemize}
	\item \textbf{Untrained:} The character doesn't know how to use firearms, and even when shown, will easily forget it. Acquired knowledge lasts less than a day, and even so, the character takes at least two turns to reload a gun before being able to use it. In video game adaptations, this should also halve the gun's effective range.
	\item \textbf{Basic:} The character got the hang of using firearms, and takes just one turn to reload a gun before using it. 
	\item \textbf{Advanced:} The character reiceves a \Bonus{25\%} to all damage done by guns, and a \Bonus{25\%} to chance to hit or cause critical damage. In video game adaptations, this should also increase range.
	\item \textbf{Expert:} The character reiceves a \Bonus{50\%} to all damage done by guns, and a \Bonus{50\%} to chance to hit or cause critical damage. In video game adaptations, this should also increase range.
\end{itemize}\newpage
\section{Magic Feats}
\subsection{Photomancy}
\begin{table}[!ht]
\centering
\FeatIII{Basic}{Advanced}{Expert}{Photomancy}{{\import{images/Feats/}{images/Feats/LightMagic1.pdf_tex}}}{{\import{images/Feats/}{images/Feats/LightMagic2.pdf_tex}}}{{\import{images/Feats/}{images/Feats/LightMagic3.pdf_tex}}}
\end{table}
\textbf{Requirements:}
\begin{itemize}
	\item \textbf{Basic:} Any Spellcaster class \Parentheses{any class that is Magician-type, Cleric-type or Druid-type}.
	\item \textbf{Advanced:} Any Spellcaster class \Parentheses{any class that is Magician-type, Cleric-type or Druid-type}.
	\item \textbf{Expert:} Any Spellcaster class \Parentheses{any class that is Magician-type, Cleric-type or Druid-type}.
\end{itemize}
\textbf{Effects:}
\begin{itemize}
	\item \textbf{Untrained:} The character cannot learn any Light Magic spells.
	\item \textbf{Basic:} The character can learn Basic-level Light Magic spells, and also gains a bonus Basic-level Light Magic spell of their chosing.
	\item \textbf{Advanced:} The character can learn Advanced-level Light Magic spells, and gets a \Bonus{10\%} to the effectiveness of their Basic-tier Light Magic spells. The character also gains a bonus Advanced-level Light Magic spell of their chosing.
	\item \textbf{Expert:} The character can learn Expert-level Light Magic spells, and gets a \Bonus{30\%} to the effectiveness of their Basic-tier Light Magic spells and \Bonus{15\%} to the effectiveness of their Advanced-tier Light Magic spells. The character also gains a bonus Expert-level Light Magic spell of their chosing.
\end{itemize}\newpage
\subsection{Sciomancy}
\begin{table}[!ht]
\centering
\FeatIII{Basic}{Advanced}{Expert}{Sciomancy}{{\import{images/Feats/}{images/Feats/DarkMagic1.pdf_tex}}}{{\import{images/Feats/}{images/Feats/DarkMagic2.pdf_tex}}}{{\import{images/Feats/}{images/Feats/DarkMagic3.pdf_tex}}}
\end{table}
\textbf{Requirements:}
\begin{itemize}
	\item \textbf{Basic:} Any Spellcaster class \Parentheses{any class that is Magician-type, Cleric-type or Druid-type}.
	\item \textbf{Advanced:} Any Spellcaster class \Parentheses{any class that is Magician-type, Cleric-type or Druid-type}.
	\item \textbf{Expert:} Any Spellcaster class \Parentheses{any class that is Magician-type, Cleric-type or Druid-type}.
\end{itemize}
\textbf{Effects:}
\begin{itemize}
	\item \textbf{Untrained:} The character cannot learn any Dark Magic spells.
	\item \textbf{Basic:} The character can learn Basic-level Dark Magic spells, and also gains a bonus Basic-level Dark Magic spell of their chosing.
	\item \textbf{Advanced:} The character can learn Advanced-level Dark Magic spells, and gets a \Bonus{10\%} to the effectiveness of their Basic-tier Dark Magic spells. The character also gains a bonus Advanced-level Dark Magic spell of their chosing.
	\item \textbf{Expert:} The character can learn Expert-level Dark Magic spells, and gets a \Bonus{30\%} to the effectiveness of their Basic-tier Dark Magic spells and \Bonus{15\%} to the effectiveness of their Advanced-tier Dark Magic spells. The character also gains a bonus Expert-level Dark Magic spell of their chosing.
\end{itemize}\newpage
\subsection{Necromancy}
\begin{table}[!ht]
\centering
\FeatIII{Basic}{Advanced}{Expert}{Necromancy}{{\import{images/Feats/}{images/Feats/DeathMagic1.pdf_tex}}}{{\import{images/Feats/}{images/Feats/DeathMagic2.pdf_tex}}}{{\import{images/Feats/}{images/Feats/DeathMagic3.pdf_tex}}}
\end{table}
\textbf{Requirements:}
\begin{itemize}
	\item \textbf{Basic:} The character already having Basic Sciomancy.
	\item \textbf{Advanced:} None
	\item \textbf{Expert:} The character already having Advanced Sciomancy.
\end{itemize}
\textbf{Effects:}
\begin{itemize}
	\item \textbf{Untrained:} The character cannot revive deceased people and animals as Undead.
	\item \textbf{Basic:} The character can revive dead people and animals as Skeletons and Zombies. The character may control only 5 undead creatures at once.
	\item \textbf{Advanced:} The character can entrap the recently deceased as Ghosts/Wraiths/Wights in their service. The character may control 15 undead creatures at once.
	\item \textbf{Expert:} The character can revive the dead in pristine condition, albeit without their souls. They can also control an unlimited amount of undead creatures at once, potentially an army!
\end{itemize}\newpage
\subsection{Olethromancy}
\begin{table}[!ht]
\centering
\FeatIII{Basic}{Advanced}{Expert}{Olethromancy}{{\import{images/Feats/}{images/Feats/DestructionMagic1.pdf_tex}}}{{\import{images/Feats/}{images/Feats/DestructionMagic2.pdf_tex}}}{{\import{images/Feats/}{images/Feats/DestructionMagic3.pdf_tex}}}
\end{table}
\textbf{Requirements:}
\begin{itemize}
	\item \textbf{Basic:} Any Magician-type class.
	\item \textbf{Advanced:} Any Magician-type class.
	\item \textbf{Expert:} Any Magician-type class.
\end{itemize}
\textbf{Effects:}
\begin{itemize}
	\item \textbf{Untrained:} The character cannot cast Destruction Magic spells, unless he/she has the feat necessary for the corresponding element. The majority of Destruction Magic spells also belong to a school of Elemental Magic \Parentheses{Fire Magic, Water Magic, Air Maqgic or Earth Magic}, thus, a spell that belongs to both schools requires only one of the two feats. Only a small minority of Destruction Magic spells don't also belong to a school of Elemental Magic.
	\item \textbf{Basic:} The character can learn Basic-level Destruction Magic spells, and also gains a bonus Basic-level Destruction Magic spell of their chosing.
	\item \textbf{Advanced:} The character can learn Advanced-level Destruction Magic spells, and gets a \Bonus{10\%} to the effectiveness of their Basic-Destruction Destruction Magic spells. The character also gains a bonus Advanced-level Destruction Magic spell of their chosing. For Destruction Magic spells that also belong to one of the schools of Elemental Magic, this bonus also stacks up with the feat relevant to the corresponding element.
	\item \textbf{Expert:} The character can learn Expert-level Destruction Magic spells, and gets a \Bonus{30\%} to the effectiveness of their Basic-tier Destruction Magic spells and \Bonus{15\%} to the effectiveness of their Advanced-tier Destruction Magic spells. The character also gains a bonus Expert-level Destruction Magic spell of their chosing. For Destruction Magic spells that also belong to one of the schools of Elemental Magic, this bonus also stacks up with the feat relevant to the corresponding element.
\end{itemize}\newpage
\subsection{Pyromancy}
\begin{table}[!ht]
\centering
\FeatIII{Basic}{Advanced}{Expert}{Pyromancy}{{\import{images/Feats/}{images/Feats/FireMagic1.pdf_tex}}}{{\import{images/Feats/}{images/Feats/FireMagic2.pdf_tex}}}{{\import{images/Feats/}{images/Feats/FireMagic3.pdf_tex}}}
\end{table}
\textbf{Requirements:}
\begin{itemize}
	\item \textbf{Basic:} Any Spellcaster class \Parentheses{any class that is Magician-type, Cleric-type or Druid-type}.
	\item \textbf{Advanced:} Any Spellcaster class \Parentheses{any class that is Magician-type, Cleric-type or Druid-type}.
	\item \textbf{Expert:} Any Spellcaster class \Parentheses{any class that is Magician-type, Cleric-type or Druid-type}.
\end{itemize}
\textbf{Effects:}
\begin{itemize}
	\item \textbf{Untrained:} The character cannot learn any Fire Magic spells.
	\item \textbf{Basic:} The character can learn Basic-level Fire Magic spells, and also gains a bonus Basic-level Fire Magic spell of their chosing.
	\item \textbf{Advanced:} The character can learn Advanced-level Fire Magic spells, and gets a \Bonus{10\%} to the effectiveness of their Basic-tier Fire Magic spells. The character also gains a bonus Advanced-level Fire Magic spell of their chosing.
	\item \textbf{Expert:} The character can learn Expert-level Fire Magic spells, and gets a \Bonus{30\%} to the effectiveness of their Basic-tier Fire Magic spells and \Bonus{15\%} to the effectiveness of their Advanced-tier Fire Magic spells. The character also gains a bonus Expert-level Fire Magic spell of their chosing.
\end{itemize}\newpage
\subsection{Hydromancy}
\begin{table}[!ht]
\centering
\FeatIII{Basic}{Advanced}{Expert}{Hydromancy}{{\import{images/Feats/}{images/Feats/WaterMagic1.pdf_tex}}}{{\import{images/Feats/}{images/Feats/WaterMagic2.pdf_tex}}}{{\import{images/Feats/}{images/Feats/WaterMagic3.pdf_tex}}}
\end{table}
\textbf{Requirements:}
\begin{itemize}
	\item \textbf{Basic:} Any Spellcaster class \Parentheses{any class that is Magician-type, Cleric-type or Druid-type}.
	\item \textbf{Advanced:} Any Spellcaster class \Parentheses{any class that is Magician-type, Cleric-type or Druid-type}.
	\item \textbf{Expert:} Any Spellcaster class \Parentheses{any class that is Magician-type, Cleric-type or Druid-type}.
\end{itemize}
\textbf{Effects:}
\begin{itemize}
	\item \textbf{Untrained:} The character cannot learn any Water Magic spells.
	\item \textbf{Basic:} The character can learn Basic-level Water Magic spells, and also gains a bonus Basic-level Water Magic spell of their chosing.
	\item \textbf{Advanced:} The character can learn Advanced-level Water Magic spells, and gets a \Bonus{10\%} to the effectiveness of their Basic-tier Water Magic spells. The character also gains a bonus Advanced-level Water Magic spell of their chosing.
	\item \textbf{Expert:} The character can learn Expert-level Water Magic spells, and gets a \Bonus{30\%} to the effectiveness of their Basic-tier Water Magic spells and \Bonus{15\%} to the effectiveness of their Advanced-tier Water Magic spells. The character also gains a bonus Expert-level Water Magic spell of their chosing.
\end{itemize}\newpage
\subsection{Geomancy}
\begin{table}[!ht]
\centering
\FeatIII{Basic}{Advanced}{Expert}{Geomancy}{{\import{images/Feats/}{images/Feats/EarthMagic1.pdf_tex}}}{{\import{images/Feats/}{images/Feats/EarthMagic2.pdf_tex}}}{{\import{images/Feats/}{images/Feats/EarthMagic3.pdf_tex}}}
\end{table}
\textbf{Requirements:}
\begin{itemize}
	\item \textbf{Basic:} Any Spellcaster class \Parentheses{any class that is Magician-type, Cleric-type or Druid-type}.
	\item \textbf{Advanced:} Any Spellcaster class \Parentheses{any class that is Magician-type, Cleric-type or Druid-type}.
	\item \textbf{Expert:} Any Spellcaster class \Parentheses{any class that is Magician-type, Cleric-type or Druid-type}.
\end{itemize}
\textbf{Effects:}
\begin{itemize}
	\item \textbf{Untrained:} The character cannot learn any Earth Magic spells.
	\item \textbf{Basic:} The character can learn Basic-level Earth Magic spells, and also gains a bonus Basic-level Earth Magic spell of their chosing.
	\item \textbf{Advanced:} The character can learn Advanced-level Earth Magic spells, and gets a \Bonus{10\%} to the effectiveness of their Basic-tier Earth Magic spells. The character also gains a bonus Advanced-level Earth Magic spell of their chosing.
	\item \textbf{Expert:} The character can learn Expert-level Earth Magic spells, and gets a \Bonus{30\%} to the effectiveness of their Basic-tier Earth Magic spells and \Bonus{15\%} to the effectiveness of their Advanced-tier Earth Magic spells. The character also gains a bonus Expert-level Earth Magic spell of their chosing.
\end{itemize}\newpage
\subsection{Aeromancy}
\begin{table}[!ht]
\centering
\FeatIII{Basic}{Advanced}{Expert}{Aeromancy}{{\import{images/Feats/}{images/Feats/AirMagic1.pdf_tex}}}{{\import{images/Feats/}{images/Feats/AirMagic2.pdf_tex}}}{{\import{images/Feats/}{images/Feats/AirMagic3.pdf_tex}}}
\end{table}
\textbf{Requirements:}
\begin{itemize}
	\item \textbf{Basic:} Any Spellcaster class \Parentheses{any class that is Magician-type, Cleric-type or Druid-type}.
	\item \textbf{Advanced:} Any Spellcaster class \Parentheses{any class that is Magician-type, Cleric-type or Druid-type}.
	\item \textbf{Expert:} Any Spellcaster class \Parentheses{any class that is Magician-type, Cleric-type or Druid-type}.
\end{itemize}
\textbf{Effects:}
\begin{itemize}
	\item \textbf{Untrained:} The character cannot learn any Air Magic spells.
	\item \textbf{Basic:} The character can learn Basic-level Air Magic spells, and also gains a bonus Basic-level Air Magic spell of their chosing.
	\item \textbf{Advanced:} The character can learn Advanced-level Air Magic spells, and gets a \Bonus{10\%} to the effectiveness of their Basic-tier Air Magic spells. The character also gains a bonus Advanced-level Air Magic spell of their chosing.
	\item \textbf{Expert:} The character can learn Expert-level Air Magic spells, and gets a \Bonus{30\%} to the effectiveness of their Basic-tier Air Magic spells and \Bonus{15\%} to the effectiveness of their Advanced-tier Air Magic spells. The character also gains a bonus Expert-level Air Magic spell of their chosing.
\end{itemize}\newpage
\chapter{Weapons n' Armour}
\chapter{Magicks}
\end{document}
