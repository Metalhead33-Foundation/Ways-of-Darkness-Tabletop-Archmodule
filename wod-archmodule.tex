\documentclass[openany,10pt,a4paper]{book}
\usepackage{import}
\usepackage[T1]{fontenc}
\usepackage{tgbonum}
\usepackage{graphicx}
\usepackage[utf8x]{inputenc}
\usepackage{multirow}
\usepackage{blindtext}
\usepackage{graphicx}
\usepackage{numprint}
\usepackage{listings}
\usepackage{xcolor}
\usepackage{tabularx,booktabs}
\usepackage{amsmath}
\usepackage{mathtools}
\usepackage{float}
\usepackage{caption}
\usepackage{enumitem}
\usepackage{tcolorbox}
\usepackage{newtxtext,newtxmath}
\usepackage[flushleft]{threeparttable}
\usepackage{adjustbox}
\usepackage[square,sort,comma,numbers]{natbib}
\usepackage[a4paper, margin=1.5cm]{geometry}
%\setcounter{tocdepth}{2}
\newcolumntype{Y}{>{\centering\arraybackslash}X}
\graphicspath{ {./images/} }
\author{Metalhead33}
\title{Ways of Darkness\\
   \normalsize A module for the World of Artograch RPG system}
\definecolor{mGreen}{rgb}{0,0.6,0}
\definecolor{mGray}{rgb}{0.5,0.5,0.5}
\definecolor{mPurple}{rgb}{0.3,0,0.3}
\definecolor{backgroundColour}{rgb}{0.95,0.95,0.92}
\usepackage{hyperref}
\hypersetup{
    colorlinks=true,
    linkcolor=blue,
    filecolor=magenta,      
    urlcolor=cyan,
}

\begin{document}
\renewcommand\tabularxcolumn[1]{m{#1}}% for vertical centering text in X column
\newcommand{\BipedalTwoArms}[3]{Being bipedal {#1}, {#3} have two hands and two legs, meaning that they can wield only two one-handed weapons \textit{(or a one-handed weapon and a shield)} or a single two-handed weapon at the same time. Just like with other bipedal {#2} races, reaching zero hitpoints on the arms or legs renders those limbs unusable \textit{(and disables the corresponding item slots)}, while reaching zero hitpoints at the torso or head means death - or unconsciousness, if house rules rule out death.}
\newcommand{\MammalRace}[1]{\BipedalTwoArms{mammals without tails}{mammalian}{{#1}} As mammals, {#1} cannot breathe under water without magic or specifial \textit{(read: modern, therefore nonexistent in this setting)} equipment, with prolonged presence underweater leading to drowning.}
\newcommand{\LizardRace}[1]{\BipedalTwoArms{reptoids with tails}{reptilian}{{#1}} Their tails are not a vital body part - their loss may be very painful \textit{(and cause potentially lethal bleeding, if the wound is not treated fast enough)}, the body part itself isn't vital, making its loss non-lethal. As we're talking about reptiles rather than mammmals, this species breeds by eggs rather than giving live birth; has a critical weakness towards extreme temperatures and cannot be turned into a secondary race specific to mammals. Reptiles are different from amphibians - they are \textbf{not} able to breathe underwater, and thus drown just like mammals. They cannot be infected with Vampirism or Theriantropy, because those are mammal-specific diseases.}
\newcommand{\UndeadRace}[1]{Being undead creatures, {#1} have no need for food or water, thus the vitals for proteon, carbohydrates, fat, calories, alcohol and water are inactive for them.}
\newcommand{\Bonus}[1]{\textcolor{green}{\textbf{+{#1} bonus}}}
\newcommand{\BonusS}[1]{\textcolor{green}{\textbf{+{#1}}}}
\newcommand{\Malus}[1]{\textcolor{red}{\textbf{-{#1} malus}}}
\newcommand{\MalusS}[1]{\textcolor{red}{\textbf{-{#1}}}}
\newcommand{\MalusP}[1]{\textcolor{red}{\textbf{+{#1} malus}}}
\newcommand{\MalusPS}[1]{\textcolor{red}{\textbf{+{#1}}}}
\newcommand{\DimorphismBB}[5]{Male {#1} get a \Bonus{{#2}} to {#3}, while Female {#1} get a \Bonus{{#4}} to {#5}.}
\newcommand{\SoCalled}[1]{\textit{``{#1}''}}
\newcommand{\Parentheses}[1]{\textit{({#1})}}
\newcommand{\FeatIIHeader}[2]{
\multicolumn{1}{|c}{{#1}}        & \multicolumn{1}{c}{{#2}}  \\ \hline \hline
}
\newcommand{\FeatIIContent}[2]{
\multicolumn{1}{|c}{\resizebox{.2\linewidth}{!}{{#1}}} & \multicolumn{1}{c}{\resizebox{.2\linewidth}{!}{{#2}}} \\ \hline
}
\newcommand{\FeatIIIHeader}[3]{
\multicolumn{1}{|c}{{#1}}        & \multicolumn{1}{c}{{#2}}       & \multicolumn{1}{c|}{{#3}}       \\ \hline \hline
}
\newcommand{\FeatIIIContent}[3]{
\multicolumn{1}{|c}{\resizebox{.2\linewidth}{!}{{#1}}} & \multicolumn{1}{c}{\resizebox{.2\linewidth}{!}{{#2}}} & \multicolumn{1}{c|}{\resizebox{.2\linewidth}{!}{{#3}}} \\ \hline
}
\newcommand{\FeatIVHeader}[4]{
\multicolumn{1}{c}{{#1}}        & \multicolumn{1}{c}{{#2}}       & \multicolumn{1}{c}{{#3}}       & \multicolumn{1}{c|}{{#4}}       \\ \hline \hline
}
\newcommand{\FeatIVContent}[4]{
\multirow{1}{*}{{#1}} & \multicolumn{1}{c}{\resizebox{.1\linewidth}{!}{{#2}}} & \multicolumn{1}{c}{\resizebox{.1\linewidth}{!}{{#3}}} & \multicolumn{1}{c|}{\resizebox{.1\linewidth}{!}{{#4}}} \\ \hline
}
\newcommand{\LangFeatContent}[4]{
#1 & \adjustbox{valign=M,min size={.075\linewidth}{.075\linewidth}}{{#2}}     & \adjustbox{valign=M,min size={.075\linewidth}{.075\linewidth}}{{#3}}         & \adjustbox{valign=M,min size={.075\linewidth}{.075\linewidth}}{{#4}}      \\ \hline
}
\newcommand{\FeatIRow}[2]{
{#1} \\ \hline
{\resizebox{.2\linewidth}{!}{{#2}}} \\ \hline
}
\newcommand{\FeatIIRow}[4]{
\FeatIIContent{{#3}}{{#4}}
\FeatIIHeader{{#1}}{{#2}}
}
\newcommand{\FeatIIIRow}[6]{
\FeatIIIContent{{#4}}{{#5}}{{#6}}
\FeatIIIHeader{{#1}}{{#2}}{{#3}}
}
\newcommand{\FeatI}[2]{\begin{tabular}{|c|}
\FeatIRow{{#1}}{{#2}}
\end{tabular}}
\newcommand{\FeatII}[5]{\begin{tabularx}{0.75\textwidth}{c @{\extracolsep{\fill}} c}
\hline
\multicolumn{2}{|c|}{{#3}} \\ \hline
\FeatIIRow{{#1}}{{#2}}{{#4}}{{#5}}
\end{tabularx}}
\newcommand{\MoreInfo}[1]{\begin{tcolorbox}[title=Notice]
More \textit{(up-to-date)} information at: \url{https://ways-of-darkness.sonck.nl/#1}
\end{tcolorbox}}
\newcommand{\FeatIII}[7]{\begin{tabularx}{0.75\textwidth}{c @{\extracolsep{\fill}} cc}
\hline
\multicolumn{3}{|c|}{{#4}} \\ \hline
\FeatIIIRow{{#1}}{{#2}}{{#3}}{{#5}}{{#6}}{{#7}}
\end{tabularx}}
\newcommand{\ReligionHeader}[0]{
\hline
\textbf{Virtue} & \textbf{Effect}               & \textbf{Contradicts with}      \\ \hline
}
%\newcommand{\ReligionRow}[5]{
%\textbf{{#1}}  & {#3} & \multirow{2}{*}{{#5}} \\
%{#2}   & {#4} &                       \\ \hline
%}
\newcommand{\ReligionRow}[5]{
\subsection{{#1}}

{#2}

\begin{itemize}
\item \textbf{Effect:} {#3}
\item \textbf{Divine Favour:} {#4}
\item \textbf{Contradicts with:} {#5}
\end{itemize}
}

\maketitle
\chapter*{Versions}
\input{latex.make/generated/versions.tex}
\tableofcontents
\input{chapters.d.tex/index.tex}
\end{document}
