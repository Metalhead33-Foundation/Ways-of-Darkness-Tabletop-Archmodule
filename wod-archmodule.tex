\documentclass[openany,10pt,a4paper]{book}
\usepackage{graphicx}
\usepackage[utf8x]{inputenc}
\usepackage{multirow}
\usepackage{blindtext}
\usepackage{graphicx}
\usepackage{numprint}
\usepackage{listings}
\usepackage{xcolor}
\usepackage{tabularx,booktabs}
\usepackage{amsmath}
\usepackage{mathtools}
\usepackage{float}
\usepackage{caption}
\usepackage{enumitem}
\usepackage{newtxtext,newtxmath}
\usepackage{makecell}
\usepackage{import}
\usepackage[square,sort,comma,numbers]{natbib}
\usepackage[a4paper, margin=1.5cm]{geometry}
%\setcounter{tocdepth}{2}
\newcolumntype{Y}{>{\centering\arraybackslash}X}
\graphicspath{ {./images/} }
\author{Metalhead33}
\title{Ways of Darkness\\
   \normalsize A module for the World of Artograch RPG system}
\definecolor{mGreen}{rgb}{0,0.6,0}
\definecolor{mGray}{rgb}{0.5,0.5,0.5}
\definecolor{mPurple}{rgb}{0.3,0,0.3}
\definecolor{backgroundColour}{rgb}{0.95,0.95,0.92}
\usepackage{hyperref}
\hypersetup{
    colorlinks=true,
    linkcolor=blue,
    filecolor=magenta,      
    urlcolor=cyan,
}

\begin{document}
\newcommand{\BipedalTwoArms}[3]{Being bipedal {#1}, {#3} have two hands and two legs, meaning that they can wield only two one-handed weapons \textit{(or a one-handed weapon and a shield)} or a single two-handed weapon at the same time. Just like with other bipedal {#2} races, reaching zero hitpoints on the arms or legs renders those limbs unusable \textit{(and disables the corresponding item slots)}, while reaching zero hitpoints at the torso or head means death - or unconsciousness, if house rules rule out death.}
\newcommand{\MammalRace}[1]{\BipedalTwoArms{mammals without tails}{mammalian}{{#1}} As mammals, {#1} cannot breathe under water without magic or specifial \textit{(read: modern, therefore nonexistent in this setting)} equipment, with prolonged presence underweater leading to drowning.}
\newcommand{\LizardRace}[1]{\BipedalTwoArms{reptoids with tails}{reptilian}{{#1}} Their tails are not a vital body part - their loss may be very painful \textit{(and cause potentially lethal bleeding, if the wound is not treated fast enough)}, the body part itself isn't vital, making its loss non-lethal. As we're talking about reptiles rather than mammmals, this species breeds by eggs rather than giving live birth; has a critical weakness towards extreme temperatures and cannot be turned into a secondary race specific to mammals. Reptiles are different from amphibians - they are \textbf{not} able to breathe underwater, and thus drown just like mammals. They cannot be infected with Vampirism or Theriantropy, because those are mammal-specific diseases.}
\newcommand{\UndeadRace}[1]{Being undead creatures, {#1} have no need for food or water, thus the vitals for proteon, carbohydrates, fat, calories, alcohol and water are inactive for them.}
\newcommand{\Bonus}[1]{\textcolor{green}{\textbf{+{#1} bonus}}}
\newcommand{\BonusS}[1]{\textcolor{green}{\textbf{+{#1}}}}
\newcommand{\Malus}[1]{\textcolor{red}{\textbf{-{#1} malus}}}
\newcommand{\MalusS}[1]{\textcolor{red}{\textbf{-{#1}}}}
\newcommand{\MalusP}[1]{\textcolor{red}{\textbf{+{#1} malus}}}
\newcommand{\MalusPS}[1]{\textcolor{red}{\textbf{+{#1}}}}
\newcommand{\DimorphismBB}[5]{Male {#1} get a \Bonus{{#2}} to {#3}, while Female {#1} get a \Bonus{{#4}} to {#5}.}
\newcommand{\SoCalled}[1]{\textit{``{#1}''}}
\newcommand{\Parentheses}[1]{\textit{({#1})}}
\newcommand{\FeatI}[2]{\begin{tabular}{|c|}
\hline
{#1} \\ \hline
{\resizebox{.2\linewidth}{!}{{#2}}} \\ \hline
\end{tabular}}
\newcommand{\FeatII}[5]{\begin{tabularx}{0.75\textwidth}{c @{\extracolsep{\fill}} c}
\hline
\multicolumn{2}{|c|}{{#3}} \\ \hline
\multicolumn{1}{|c}{\resizebox{.2\linewidth}{!}{{#4}}} & \multicolumn{1}{c}{\resizebox{.2\linewidth}{!}{{#5}}} \\ \hline
\multicolumn{1}{|c}{{#1}}        & \multicolumn{1}{c}{{#2}}  \\ \hline
\end{tabularx}}
\newcommand{\FeatIII}[7]{\begin{tabularx}{0.75\textwidth}{c @{\extracolsep{\fill}} cc}
\hline
\multicolumn{3}{|c|}{{#4}} \\ \hline
\multicolumn{1}{|c}{\resizebox{.2\linewidth}{!}{{#5}}} & \multicolumn{1}{c}{\resizebox{.2\linewidth}{!}{{#6}}} & \multicolumn{1}{c|}{\resizebox{.2\linewidth}{!}{{#7}}} \\ \hline
\multicolumn{1}{|c}{{#1}}        & \multicolumn{1}{c}{{#2}}       & \multicolumn{1}{c|}{{#3}}       \\ \hline
\end{tabularx}}

\maketitle
\tableofcontents
\chapter*{Preface}
You are currently reading the documentation of the \textbf{Ways of Darkness module for} the \textbf{World of Artograch RPG system}. The author of this document presents you all the information within this document with the assumption that you have read the \textbf{World of Artograch Ruleset}, and thus are familiar with the rules it presented. Contents of the aforementioned document are expected to be referenced in this document.
\addcontentsline{toc}{chapter}{Preface}
\chapter{Introduction to the Occident}
\includegraphics[width=\textwidth,height=\textheight,keepaspectratio]{Occident_Borders}
\newpage
The \textbf{Occident} is the part of \textbf{Artograch} with perhaps the most dynamic history. While even the Occident shows a similar tendency as the Orient to have large countries dominated by a single race, it is slightly more nuanced in the Occident, which has had a history of territories changing hands. While the Orient has always been characterized by the rule of centralized and homogenous kingdoms, and highly bureaucratic and usually equally homogenous, often quasi-despotic empires, the Occident has always been a place where central authorities held only limited amount of power, with provinces enjoying high autonomy.\newline
The various races of the Occident are divided into two major groups: the Elven races \textit{(Humans, High Elves, Wood Elves, Dark Elves, Orcs)} being descended from Oriental invaders; and the indigenous races \textit{(Goblins, Ogres, Lizardmen, Halflings, Gnomes and Dwarves)} whose population was decimated by the arrival of the earlier. Out of the two, the earlier are clearly the dominant force on the continent, with three out of the four dominant powers in the Occident being dominated by an Elven race - Etrand by Humans, Froturn by High Elves, Dragoc by Wood Elves.\newline
Currently, as of 831 AEKE \textit{(831 years after the establishment of the Kingdom of Etrand)}, the Occident is at a tipping point: the three aforementioned great powers are at a three-way cold war with each other, with tensions being at an all-time high, while the previously unmentioned fourth, Gabyr, is sitting in the shadows watching it all with popcorn in their hands, while slowly expanding their trade empire in the Orient. Unseen for roughly five and a half centuries, Demons have been sighted lurking around, no doubt planting their own secret agents among the authorities of the aforementioned states. It is said that even the Orcs of Brutang are getting quite unruly, while the Empire of Neressa continues its long-practiced isolationist policies and the Principality of Artaburro tries to wiggle between the Kingdoms of Froturn, Dragoc and Etrand.
\chapter{Playable Primary Races}
\section{Humans}
More \textit{(up-to-date)} information at: \url{https://ways-of-darkness.sonck.nl/Humans}\newline
\includegraphics{Average_Humans}\newline
\textbf{Humans} are the dominant race of the \textbf{Kingdom of Etrand} and its vassal, the \textbf{Earldom of Etrancoast}, albeit they are also present in high numbers in the \textbf{Principality of Gabyr}, constituting slightly more than half of its population, albeit they're not politically dominant there. In spite of their short lifespan - less than one century! - and their clear lack of pointy ears, humans are in fact an Elven race, descendants of those Ancestral Elves who crossed over from the Orient to colonize the Occident, making them a sister-race to the Wood Elves and High Elves, who also directly evolved out of those Ancestral Elves.\newline
The average human male is 175 centimetres \textit{(5 feet and 9 inches)} tall, while the average human female is 165 centimetres \textit{(5 feet and 5 inches)} tall, however, humans possess much greater variation amongst themselves in regards to height than other races, adult humans being as tall as 215 centimetres \textit{(84 inches or 7 feet)} or as short as 130 centimetres \textit{(4 feet and 3 inches)} not being unheard of either. With possibly the exception of the Lizardmen with the varied colours of their scale, humans are the most diverse race on Artograch, with a large variety of hair colours \textit{(black, brown, blond, red, auburn)} and eye colours \textit{(brown, blue, hazel, grey, green)} occouring naturally.\newline
\begin{tabular}{|c|c|c|}
\hline
 & \textbf{Min} & \textbf{Max} \\ \hline
\textbf{Strength} & 8 & 18 \\ \hline
\textbf{Endurance} & 8 & 18 \\ \hline
\textbf{Dexterity} & 8 & 18 \\ \hline
\textbf{Intelligence} & 8 & 18 \\ \hline
\textbf{Willpower} & 8 & 18 \\ \hline
\textbf{Charisma} & 8 & 18 \\ \hline
\end{tabular}\newline
\DimorphismBB{Humans}{1}{Strength and Endurance}{1}{Dexterity and Charisma} \MammalRace{humans}\newpage
\section{High Elves}
More \textit{(up-to-date)} information at: \url{https://ways-of-darkness.sonck.nl/High_Elves}\newline
\includegraphics{High_Elves}\newline
\textbf{High Elves} are the dominant race of the \textbf{Kingdom of Froturn} and have a plurality in the \textbf{Empire of Neressa}, albeit their political influence expands far beyond, to the point that the Humans have adopted their religion over eight centuries ago! They have rather long lifespans, theoretically ten times that of a human \textit{(and they're also typically taller than humans, not to mention their sharp ears)}, ageing at one tenth a human's rate after reaching the age of eighteen. They descendants of those Ancestral Elves who crossed over from the Orient to colonize the Occident, making them a sister-race to the Wood Elves and Humans, who also directly evolved out of those Ancestral Elves. As an added bonus, High Elven women are deemed the most beautiful on Artograch, their fame echoing beyond the continent of Artograch, even to the Orient. The average adult High Elf is about 190 centimetres \textit{(6 feet and 3 inches)} tall, regardless of gender \textit{(albeit men to tend to be slightly taller on average)}.\newline
\begin{tabular}{|c|c|c|}
\hline
 & \textbf{Min} & \textbf{Max} \\ \hline
\textbf{Strength} & 8 & 18 \\ \hline
\textbf{Endurance} & 7 & 17 \\ \hline
\textbf{Dexterity} & 9 & 19 \\ \hline
\textbf{Intelligence} & 8 & 18 \\ \hline
\textbf{Willpower} & 8 & 18 \\ \hline
\textbf{Charisma} & 9 & 19 \\ \hline
\end{tabular}\newline
\DimorphismBB{High Elves}{1}{Strength}{1}{Charisma} \MammalRace{High Elves}\newpage
\section{Wood Elves}
More \textit{(up-to-date)} information at: \url{https://ways-of-darkness.sonck.nl/Wood_Elves}\newline
\includegraphics{Wood_Elves}\newline
\textbf{Wood Elves} are the dominant race of the \textbf{Kingdom of Dragoc}. They have rather long lifespans, theoretically ten times that of a human \textit{(and they're also typically taller than humans, not to mention their sharp ears)}, ageing at one tenth a human's rate after reaching the age of eighteen. They descendants of those Ancestral Elves who crossed over from the Orient to colonize the Occident, making them a sister-race to the High Elves and Humans, who also directly evolved out of those Ancestral Elves. As an added bonus, Wood Elven women and men are well-known for being rivals to their High Elven counterparts in regards to beauty. The average adult Wood Elf is about 190 centimetres \textit{(6 feet and 3 inches)} tall, regardless of gender \textit{(albeit men to tend to be slightly taller on average)}.\newline
\begin{tabular}{|c|c|c|}
\hline
 & \textbf{Min} & \textbf{Max} \\ \hline
\textbf{Strength} & 8 & 18 \\ \hline
\textbf{Endurance} & 7 & 17 \\ \hline
\textbf{Dexterity} & 9 & 19 \\ \hline
\textbf{Intelligence} & 8 & 18 \\ \hline
\textbf{Willpower} & 8 & 18 \\ \hline
\textbf{Charisma} & 8 & 18 \\ \hline
\end{tabular}\newline
\DimorphismBB{Wood Elves}{2}{Strength}{2}{Charisma} \MammalRace{Wood Elves}\newpage
\section{Dark Elves}
More \textit{(up-to-date)} information at: \url{https://ways-of-darkness.sonck.nl/Dark_Elves}\newline
\includegraphics{Dark_Elves}\newline
\textbf{Dark Elves} are a race without a unified country, living in underground clans in dungeon-cities below the the ground, though also being present in several surface states - namely the Kingdoms of Etrand and Froturn - in high numbers as diaspora. They have rather long lifespans, theoretically ten times that of a human \textit{(and they're also typically taller than humans, not to mention their sharp ears and blue-ish grey skin)}, ageing at one tenth a human's rate after reaching the age of eighteen. They descendants of Wood Elves and High Elves who were banished after experimenting with the dark arts, creating their own exodus, where the environment - and the new religion - has altered their appearence. The average adult Dark Elf is about 180 centimetres \textit{(5 feet and 11 inches)} tall, regardless of gender \textit{(albeit men to tend to be slightly taller on average)}.\newline
\begin{tabular}{|c|c|c|}
\hline
 & \textbf{Min} & \textbf{Max} \\ \hline
\textbf{Strength} & 8 & 18 \\ \hline
\textbf{Endurance} & 7 & 17 \\ \hline
\textbf{Dexterity} & 10 & 20 \\ \hline
\textbf{Intelligence} & 8 & 18 \\ \hline
\textbf{Willpower} & 8 & 18 \\ \hline
\textbf{Charisma} & 8 & 18 \\ \hline
\end{tabular}\newline
\DimorphismBB{Dark Elves}{1}{Strength}{1}{Charisma} \MammalRace{Dark Elves}\newpage
\section{Half-Elves}
More \textit{(up-to-date)} information at: \url{https://ways-of-darkness.sonck.nl/Half-Elves}\newline
\includegraphics{Half_Elves}\newline
\textbf{Half-Elves} are hybrids of Humans and any of the aforementioned Elven races \textit{(High Elves, Wood Elves, Dark Elves)}. To be specific, this race only includes only so-called \SoCalled{clean hybrids} - as in, people with 45-55\% Human blood and 45-55\% Elven blood. People where one side dominates - whether it's the Human side or Elven side - are classified as \SoCalled{unclean hybrids}, and thus get shoehorned into the Human or Elven race. Their lifespan is halfway between that of a Human and Elf, thus around five times that of a human's, ageing at one fifth the rate after reaching the age of eighteen.\newline
\begin{tabular}{|c|c|c|}
\hline
 & \textbf{Min} & \textbf{Max} \\ \hline
\textbf{Strength} & 8 & 18 \\ \hline
\textbf{Endurance} & 8 & 18 \\ \hline
\textbf{Dexterity} & 9 & 19 \\ \hline
\textbf{Intelligence} & 8 & 18 \\ \hline
\textbf{Willpower} & 8 & 18 \\ \hline
\textbf{Charisma} & 8 & 18 \\ \hline
\end{tabular}\newline
\DimorphismBB{Half-Elves}{1}{Strength}{1}{Charisma} \MammalRace{Half-Elves}\newpage
\section{Half-Orcs}
More \textit{(up-to-date)} information at: \url{https://ways-of-darkness.sonck.nl/Half-Orcs}\newline
\includegraphics{Half_Orcs}\newline
\textbf{Half-Orcs} are hybrids of Orcs and any of the other Elven races \textit{(Humans, High Elves, Wood Elves, Dark Elves)} - from a gameplay perspective, we don't distinguish between Orc-Human and Orc-Elf hybrids, as Orkish traits tend to be the most salient for both. To be specific, this race only includes only so-called \SoCalled{clean hybrids}- as in, people with 45-55\% Orkish blood and 45-55\% Human/Elven blood. People where one side dominates - whether it's the Orkish side or Human/Elven side - are classified as \SoCalled{unclean hybrids}, and thus get shoehorned into the Orkish or Human/Elven race. Their lifespan is halfway between the parent races, which would mean around 200 years for Human-Orc hybrids, 400 years for Orc-Elf hybrids.\newline
\begin{tabular}{|c|c|c|}
\hline
 & \textbf{Min} & \textbf{Max} \\ \hline
\textbf{Strength} & 9 & 19 \\ \hline
\textbf{Endurance} & 9 & 19 \\ \hline
\textbf{Dexterity} & 8 & 18 \\ \hline
\textbf{Intelligence} & 8 & 18 \\ \hline
\textbf{Willpower} & 8 & 18 \\ \hline
\textbf{Charisma} & 7 & 17 \\ \hline
\end{tabular}\newline
\MammalRace{Half-Orcs}\newpage
\section{Orcs}
More \textit{(up-to-date)} information at: \url{https://ways-of-darkness.sonck.nl/Orcs}\newline
\includegraphics{Orks}\newline
\textbf{Orcs} are an oddball among the group of Elven races - just like Humans, they look nothing like Elves, despite being genetically descended from the Ancestral Elves, albeit indirectly: the Orcs are descendants of a group of Wood Elves who were exiled for their war crimes and subsequent rebellion against their own country's authority, first tortured then banished into the frozen wastelands of Brutang with the hopes of them dying from exposure or being killed by the native Goblins and Ogres. Instead, these exiled renegades came to dominate the natives, but for some unknown reason, their children and descendants all became green-skinned beings. Even the Orcs themselves aren't sure today how did their ancestors transition from being Wood Elves to being Orcs - perhaps a curse?\newline
Orcs have a lifespan roughly three times that of humans, which means that they age at one third of a human's rate after reaching the age of eighteen. Orcs have an average height of 210 centimetres, or roughly 6 feet and 11 inches. Orcs are well-known for one of the most muscular races, albeit their green skin, shorter lifespan and ugly faces give people the impression that they have more in common with the Goblins and Ogres than the Wood Elves they supposedly descended from. Their true origins still remain a mystery, as it remains impossible to tell legend from reality.\newline
\begin{tabular}{|c|c|c|}
\hline
 & \textbf{Min} & \textbf{Max} \\ \hline
\textbf{Strength} & 10 & 20 \\ \hline
\textbf{Endurance} & 10 & 20 \\ \hline
\textbf{Dexterity} & 7 & 17 \\ \hline
\textbf{Intelligence} & 8 & 18 \\ \hline
\textbf{Willpower} & 8 & 18 \\ \hline
\textbf{Charisma} & 6 & 16 \\ \hline
\end{tabular}\newline
\MammalRace{Orcs}\newpage
\section{Ogres}
More \textit{(up-to-date)} information at: \url{https://ways-of-darkness.sonck.nl/Ogres}\newline
\includegraphics{Ogres}\newline
\textbf{Ogres} are a race indigenous to Artograch. Green-skinned, very muscular, very tall \textit{(average height of 250 centimetres, or 8 feet and 2.4 inches)} and very solitary, Ogres have an average lifespan of roughly twice of a human's. Before the arrival of the Orcs, Ogres had a symbiotic relationship with the Goblins: the thriftly Goblins visited foreign places to bring back foreign goods to enrich the land and the ogres, while the Ogres protected the land with their strong hands and xenophobia, giving the Goblins a good base of operations. This all changed with the arrival of the Orcs, who enslaved the Ogres using them for forced labour until they began fighting for emancipation, winning that fight in the more liberal-minded clans. Despite no longer being collectively enslaved by most Orkish clans anymore, Ogres are still somewhat of outsiders in Orkish societies, preferring to stay out of clan politics and abstain from aspiring for high positions in them.\newline
\begin{tabular}{|c|c|c|}
\hline
 & \textbf{Min} & \textbf{Max} \\ \hline
\textbf{Strength} & 11 & 21 \\ \hline
\textbf{Endurance} & 11 & 21 \\ \hline
\textbf{Dexterity} & 7 & 17 \\ \hline
\textbf{Intelligence} & 7 & 17 \\ \hline
\textbf{Willpower} & 7 & 17 \\ \hline
\textbf{Charisma} & 6 & 16 \\ \hline
\end{tabular}\newline
\MammalRace{Ogres}\newpage
\section{Goblins}
More \textit{(up-to-date)} information at: \url{https://ways-of-darkness.sonck.nl/Goblins}\newline
\includegraphics{Goblins}\newline
\textbf{Goblins} are a race indigenous to Artograch. Green-skinned, small \textit{(average of 110 cm, or 3 feet and 7.3 inches)}, agile and undemanding  they can live and thrive in just about any place: from the hot and scorching desert to the arctic wasteland. Thanks to their small size, they do not need big houses. Their small and agile fingers are good to hunting smaller animals and using bows. Their in-born night vision allows them to see in the darkness. They are much quicker and less noisy than the Orcs, which why is why the Orcish tribes of Brutang love employing them as scouts.\newline
As an undemanding race that can thrive just about anywhere, they can be found in nearly all corners of Artograch, although their main domicile is in Brutang, which they share with the Orcs and Ogres. Before the arrival of the Orcs in Brutang, they had a symbiotic relationship with the Ogres: the thriftly Goblins visited foreign places to bring back foreign goods to enrich the land and the ogres, while the Ogres protected the land with their strong hands and xenophobia, giving the Goblins a good base of operations. This all changed with the arrival of the Orcs, with the Goblins of Brutang having to serve new - Orcish - masters. They earned their emancipation far quicker than the Ogres, as they provid their worth as merchants, scouts, diplomats and artisans fairly early on. Outside of Brutang, their tribes are mostly nomadic, making a living of stealing and trading. These goblin tribes often create caravans, wandering with their products, caught monsters and slaves, intending to sell them to someone. These caravans are often followed by scouts on wolfback to provide cover and to look for villages to pillage.\newline
The Goblin race contains three subraces: Night Goblins, Wood Goblins and Hobgoblins, with \textbf{Night Goblins} being the sneaky ones who prefer to dwell in caves and lack claustrophobia and taphophobia of any kind; \textbf{Wood Goblins} who feel most at home when under the open sky and thus have never heard agoraphobia; and \textbf{Hobgoblins}, silghtly taller and sturdier on average than other Goblins, characterized by an affinity for rocky lands and extreme wastelands, getting as close to extremophilia as a humanoid mammal can realistically get. There is no real class distinction or caste system to keep these three subspecies apart, so interbreeding betweem them is very common.\newline
\begin{tabular}{|c|c|c|}
\hline
 & \textbf{Min} & \textbf{Max} \\ \hline
\textbf{Strength} & 8 & 18 \\ \hline
\textbf{Endurance} & 8 & 18 \\ \hline
\textbf{Dexterity} & 8 & 18 \\ \hline
\textbf{Intelligence} & 8 & 18 \\ \hline
\textbf{Willpower} & 8 & 18 \\ \hline
\textbf{Charisma} & 7 & 17 \\ \hline
\end{tabular}\newline
Hobgoblins get a \Bonus{2} bonus to Strength, Night Goblins get a \Bonus{2} to Dexterity, Wood Goblins get a \Bonus{1} to both Strength and Dexterity. \MammalRace{Goblins}\newpage
\section{Halflings}
More \textit{(up-to-date)} information at: \url{https://ways-of-darkness.sonck.nl/Halflings}\newline
\includegraphics{Hobbits}\newline
\textbf{Halflings} are a race indigenous to Artograch, with an average lifespan of 150 years, and average height of 130 centimetres or 4 feet and 3.1 inches. Due to the fact that they are rather short-statued humanoid mammals and speak a language related to Dwarven, they are sometimes grouped in with the Dwarves and Gnomes as the \SoCalled{Norlokian races}, even though they are incapable of interbreeding with each other, and thus may not even be biologically related, despite the superficial resemblance.\newline
Halflings are a race of smart, inventive survivors and adventurers, known for their curiousity and boldness. They are easily seduced by riches, but they prefer spending money, for they are a race definitely not made famous by their frugality. Their skin is ruddy, their hair is typically either black, brown or red, and either straight or curly \textit{(never wavy)}, and their eyes are usually brown or black. Halfling men very often grow out their sideburns, but seldom do they also grow a beard or mustache. They prefer dressing for comfort and practicality, and in spite of their love of riches, they typically shun jewellery. They love to enjoy the little things in life, such as smoking their pipes.\newline
\begin{tabular}{|c|c|c|}
\hline
 & \textbf{Min} & \textbf{Max} \\ \hline
\textbf{Strength} 7 8 & 17 \\ \hline
\textbf{Endurance} & 8 & 18 \\ \hline
\textbf{Dexterity} & 9 & 19 \\ \hline
\textbf{Intelligence} & 8 & 18 \\ \hline
\textbf{Willpower} & 8 & 18 \\ \hline
\textbf{Charisma} & 8 & 18 \\ \hline
\end{tabular}\newline
\DimorphismBB{Halflings}{1}{Strength and Endurance}{1}{Dexterity and Charisma} \MammalRace{Halflings}\newpage
\section{Gnomes}
More \textit{(up-to-date)} information at: \url{https://ways-of-darkness.sonck.nl/Gnomes}\newline
\includegraphics{Gnomes}\newline
\textbf{Gnomes} are a race indigenous to Artograch, with an average lifespan of 300 years, and average height of 100 centimetres or 3 feet and 3.37 inches. Due to the fact that they are rather short-statued humanoid mammals and speak the Dwarven language, they are sometimes grouped in with the Dwarves and Halflings as the \SoCalled{Norlokian races}, even though they are incapable of interbreeding with each other, and thus may not even be biologically related, despite the superficial resemblance.\newline
The Gnomes are a race like no other. While the Dwarves, Humans, Elves and even the self-proclaimed \SoCalled{spiritual} Lizardmen love gossiping about just about everything, from politics to the petty personal matters of anyone who isn’t present, the gnomes are a whole different world. They discuss science and magic, not tratsch. Gnomish genius knows no limits. They construct complex machines, golems, clockwork engines, and even more. In addition to mastering mechanics, gnomes also excel at alchemy. While the Elves - especially the Wood Elves - are unrivalled masters of Herbalism and potion-making, the Gnomes prefer to explore the more obscure yet powerful side of Alchemy: transmutation. The art of turning worthless rock into diamonds and mere paper into gold is a dangerous art mastered only by the bravest of souls.\newline
The gnomes have been living side-by-side with the dwarves for 5000 years, and against all odds, they kept their own culture, as a separate race - which is understandable, given how they are biologically incapable of interbreeding with Dwarves. Although gnomes are widely employed as alchemists, inventors and artillery engineers, they prefer to live amongst their own kind. They are fond of animals, gemstones, jokes and pranks. They like learning things in a personal way, and they always try to build up things by a new way.\newline
\begin{tabular}{|c|c|c|}
\hline
 & \textbf{Min} & \textbf{Max} \\ \hline
\textbf{Strength} & 6 & 16 \\ \hline
\textbf{Endurance} & 6 & 16 \\ \hline
\textbf{Dexterity} & 10 & 20 \\ \hline
\textbf{Intelligence} & 10 & 20 \\ \hline
\textbf{Willpower} & 8 & 18 \\ \hline
\textbf{Charisma} & 8 & 18 \\ \hline
\end{tabular}\newline
\DimorphismBB{Gnomes}{1}{Strength and Endurance}{1}{Dexterity and Charisma} \MammalRace{Gnomes}\newpage
\section{Dwarves}
More \textit{(up-to-date)} information at: \url{https://ways-of-darkness.sonck.nl/Dwarves}\newline
\includegraphics{Dwarves}\newline
\textbf{Dwarves} are a race indigenous to Artograch, with an average lifespan of 200 years, and average height of 120 centimetres or 3 feet and 11.2 inches. Due to the fact that they are rather short-statued humanoid mammals and speak a language related to Halfling, they are sometimes grouped in with the Halflings and Gnomes as the \SoCalled{Norlokian races}, even though they are incapable of interbreeding with each other, and thus may not even be biologically related, despite the superficial resemblance.\newline
The Dwarves are one of the most easily recognizable races of Artograch. Short but muscular statue, crude humour, mild xenophobia, fondness for beer and women, you will almost instantly recognize the dwarf. Having been a closed people for millennia, dwarves are not very trusting with foreigners \textit{(other than the gnomes whom they don’t really consider foreigners at all)}, but they are generous with the ones who earn their trust. They are also known for their talent at blacksmithing, mining and melee weapons. Dwarven steel is universally considered the best material for making both weapons and armour, and no one can deny the beauty of dwarven-made ornaments that decorate armour and weapons that would be already considered nicely-made even without them. Dwarves’ skin colour can vary from yellowish brown to pale white. A dwarf's hair colour can be blond, red, brown or black. In Dwarven society, if a male does not have a beard, he isn't considered male. Traditionally, the dwarves lived in clans that constantly bickered amongst each other, but historical events six centuries ago changed all of that.\newline
\begin{tabular}{|c|c|c|}
\hline
 & \textbf{Min} & \textbf{Max} \\ \hline
\textbf{Strength} & 8 & 18 \\ \hline
\textbf{Endurance} & 10 & 20 \\ \hline
\textbf{Dexterity} & 8 & 18 \\ \hline
\textbf{Intelligence} & 8 & 18 \\ \hline
\textbf{Willpower} & 8 & 18 \\ \hline
\textbf{Charisma} & 7 & 17 \\ \hline
\end{tabular}\newline
\DimorphismBB{Dwarves}{1}{Strength}{1}{Charisma} \MammalRace{Dwarves}\newpage
\section{Lizardmen}
More \textit{(up-to-date)} information at: \url{https://ways-of-darkness.sonck.nl/Lizardmen}\newline
\includegraphics{Average_Lizardman}\newline
\textbf{Lizardmen} are a cold-blooded reptiles \textit{(varanids to be specific)}, mainly characterized by thick scales, sharp thinking and high level of endurance. They populate mainly swamps and sylvan mountains. They tolerate all sorts of climates and terrains, with the exception of very low temperature: cold is their weak point. Unlike normal lizards, they mainly eat meat, but they never prey on sentiment beings \textit{(like Humans)}, they also like fruits, but they never eat grains/cereals \textit{(nor alcoholic beverage made out of it)}. Weaker lizardmen live at least 900 years, while the stronger could live up to 1300 years. The colour of the scales can vary regardless of gender. Their eyes can be green, yellow or orange. On average, they weight 70 kilograms or 155 pounds and are 175 centimetres or 5 feet and 9 inches tall.\newline
Their limbs are muscular and they can endure physically demand work for long, although not as nimble or agile as elves, they can endure running for longer amount of time than Elves or Orcs, either up to hills or on the flat ground. They are neutral to other races, they are not expansive, they are conservative and tradition-respecting. They do not engage in combat with those who doesn't attack first*, although in battle the enemy is clearly enemy for them, they do not hold them back when fighting someone who wants their head. Their warriors are more fond of agility and speed than raw strength, although they don’t underestimate the last either. They usually use spears or missile weapons, but there are some who prefer swords or heavy weapons \textit{(such as battle axes)}. They also use their strong tails in battle! Due to their thick skin, they do not wear armour, and they wear clothes rarely - most of the time only trousers.\newline
They live in smaller or bigger communities, but it’s not rare for the younger to decide to become an adventurer, travelling warrior or mercenary. There are some of them who are spellcasters, mostly shamans who help the community with their magic and teach them in the ways of the religion. There are also some of them who decide to be mages. There are some who aren’t born to their kind. They are usually sociable, but not always accepted due the way they look.\newline
\begin{tabular}{|c|c|c|}
\hline
 & \textbf{Min} & \textbf{Max} \\ \hline
\textbf{Strength} & 9 & 19 \\ \hline
\textbf{Endurance} & 10 & 20 \\ \hline
\textbf{Dexterity} & 10 & 20 \\ \hline
\textbf{Intelligence} & 9 & 19 \\ \hline
\textbf{Willpower} & 8 & 18 \\ \hline
\textbf{Charisma} & 4 & 14 \\ \hline
\end{tabular}\newline
\LizardRace{Lizardmen}\newpage
\chapter{Non-Playable Primary Races}
\textbf{Non-PLayable races} are races that, while not explicitly forbidden, are not recommended for usage as player characters for various reasons. They will not be playable in any of the video game adaptations either, and they are not recommended for tabletop usage either, other than as NPCs controlled by the game master. The reasons for this vary a lot, but boil down to these so-called \SoCalled{non-playable races} being...
\begin{itemize}
	\item ...unballanced, thus giving players who control them an unfair \Parentheses{dis}advantage.
	\item ...not as fleshed out as the so-called so-called \SoCalled{playable races}
	\item ...not fitting into the traditional formula of characters having attributes, classes and equipment. As a rule of thumb, non-bipedal races are automatically non-playable in the Occident.
\end{itemize}
As previously stated, while playing as characters from these races is not explicitly forbidden per se, it is genuinely not recommended for game masters to let players create characters from these races. These races will also be non-playable in any future video game adaptations.
\section{Nereids}
More \textit{(up-to-date)} information at: \url{https://ways-of-darkness.sonck.nl/Nereids}\newline
\textbf{Nereids} - also called \textbf{Mermaids and Mermen} - are the politically dominant race of the Free City of Gabyr, in spite of only constituting 26.72\% of its population.\newline
\begin{tabular}{|c|c|c|}
\hline
 & \textbf{Min} & \textbf{Max} \\ \hline
\textbf{Strength} & 8 & 18 \\ \hline
\textbf{Endurance} & 10 & 20 \\ \hline
\textbf{Dexterity} & 10 & 20 \\ \hline
\textbf{Intelligence} & 10 & 20 \\ \hline
\textbf{Willpower} & 10 & 20 \\ \hline
\textbf{Charisma} & 10 & 20 \\ \hline
\end{tabular}\newline
Nereids look like mammals above the hip, but hash fish bodies under the hips. This means that they have two hands - allowing them to use two one-handed weapons or a single two-handed weapon at a time, or a single one-handed weapon and a shield. They have their torso and their head, but lack legs - instead, they have a large tail, whose loss is just as lethal as the loss of their torso or head. They have the same vitals as other mortal creatures, but they cannot be infected with vampirism or theriantropy.\newpage
\section{Demons}
More \textit{(up-to-date)} information at: \url{https://ways-of-darkness.sonck.nl/Demons}\newline
\section{Angels}
More \textit{(up-to-date)} information at: \url{https://ways-of-darkness.sonck.nl/Angels}\newline
\section{Dragons}
More \textit{(up-to-date)} information at: \url{https://ways-of-darkness.sonck.nl/Dragons}\newline
\section{Griffins}
More \textit{(up-to-date)} information at: \url{https://ways-of-darkness.sonck.nl/Griffins}\newline
\section{Winged Cobras}
More \textit{(up-to-date)} information at: \url{https://ways-of-darkness.sonck.nl/Winged_Cobras}\newline
\chapter{Secondary Races}
\textbf{Secondary races} are races that people aren't usually born into, but instead are turned into, usually via some sort of magic. Most secondary races are undead, and their condition is caused by a symbiotic relationship between their own animated corpse and a tiny creature that is responsible for the animation: the only cure to such a condition is a final death.
\section{Liches}
More \textit{(up-to-date)} information at: \url{https://ways-of-darkness.sonck.nl/Liches}\newline
\textbf{Liches} are sentient undead creatures who resemble skeletons, but are in fact a power to be reckoned with, as the ritual of transformation in which liches are made is lethal to anyone who doesn't possess sufficient amount of magical power to have all of their body \Parentheses{save for their bones} burn away, yet live. In exchange for the efforts taken prior to the ritual and the excruciating pain sustained during the ritual, they gain eternal life, increased magical powers and being freed from various other mortal restraints, such as mortal needs and mortal desires.\newline
\begin{tabular}{|c|c|c|}
\hline
 & \textbf{Bonus/Malus} \\ \hline
\textbf{Strength} & \textit{unchanged} \\ \hline
\textbf{Endurance} & \textit{unchanged}  \\ \hline
\textbf{Dexterity} & \textit{unchanged}  \\ \hline
\textbf{Intelligence} & \BonusS{6} \\ \hline
\textbf{Willpower} & \BonusS{6} \\ \hline
\textbf{Charisma} & \MalusS{4} \\ \hline
\end{tabular}\newline
\UndeadRace{Liches}\newpage
\section{Vampires}
More \textit{(up-to-date)} information at: \url{https://ways-of-darkness.sonck.nl/Vampirism}\newline
\textbf{Vampires} are sentient undead creatures who - unlike Liches - largely retain their original mortal looks, albeit with a few characteristics \Parentheses{paler skin, unnatural eye colours, more prominent, pointy and sharp canine teeth}. Unlike lichdoom - which is a condition caused by a ritual - vampirism is caused by a parasitical organism infecting a living sentient mammal, killing it and reanimating its body, sharing control over the body with the original brains and soul, establishing a symbiotic relationship, just like with theriantropes. The parasite demands blood, and will drive the vampire to utter madness \Parentheses{and eventually bodily decay} if it is not delivered, but in exchange, it grants the vampire eternal life, liberation from mortal needs for food and water, as well as superior physical power. In spite of being undead creatures who are supposed to be clinically dead, those who are turned vampires prematurely will still physically mature into adults capable of sexual reproduction: the parsite inside them makes sure, that as long as they are fed with enough blood at regular intervals, some of their mortal functions will continue to work, such as the ones responsible for growing into an adult, for appreciating the taste of mortal food and beverages, and for sexual reproduction - but woe be to a vampire who neglects feeding, as they will begin to experience both bodily decay and mental degeneration.\newline
So-called \SoCalled{pure-blooded vampires} \Parentheses{the offsprings of two vampires} and \SoCalled{half-vampires} \Parentheses{the offsprings of a mortal and vampire} are immune to the harsh rays of the sun from birth, while to regular vampires, the scorching touch of the sun is lethal for roughly three centuries, during which they gradually build up an immunity and gain resistance to it.\newline
The parasite that causes vampirism is one very prone to mutation, which explains the high diversity of vampire clans with different powers unique to them.\newline
\begin{tabular}{|c|c|c|}
\hline
 & \textbf{Bonus/Malus} \\ \hline
\textbf{Strength} & \BonusS{4} \\ \hline
\textbf{Endurance} & \BonusS{4}  \\ \hline
\textbf{Dexterity} & \textit{unchanged}  \\ \hline
\textbf{Intelligence} & \textit{unchanged} \\ \hline
\textbf{Willpower} & \textit{unchanged} \\ \hline
\textbf{Charisma} & \textit{unchanged} \\ \hline
\end{tabular}\newline
\UndeadRace{Vampires} Instead, they activate a need for Blood, their own unique vital shared with Theriantropes.\newpage
\section{Theriantropes}
More \textit{(up-to-date)} information at: \url{https://ways-of-darkness.sonck.nl/Theriantropy}\newline
\textbf{Theriantropes} are sentient undead creatures who - unlike Liches - largely retain their original mortal looks, albeit they take on certain animal motiffs, such as the smell of a wild animal and character tics associated with said animal. The condition of theriantropy shares a lot with Vampirism, such as the fact that it is caused by a parasite that kills its host and reanimates the host's corpse as an undead creature, allowing them retain free will, but still influencing them, giving them powers in exchange for blood. On every full moon, young theriantropes go through an involuntary transformation that makes them lose their free will, turning them into berserkers driven by lust for blood - they also go through a physical transformation, from their original form into a bipedal animal \Parentheses{wolf, bear, boar, rat, lion, tiger, etc.}. As theriantropes grow older and stronger, they gradually learn to control their condition, transforming at will, and retaining their free will even when transformed.\newline
Just like how vampirism comes in various forms, theriantropy does to come in many forms, with werewolves \Parentheses{also known as lycantropes} being far by the most common type of theriantrope. Other kinds of theriantropes include werebears, wereboars, wererats, werelions and weretigers. The differences between these were-animals is largely cosmetic, with the common denominator being the fact that they are undead beings stronger than regular mortals, go through involuntary transformations when young, and obstain the smell and character tics of the associated animal.\newline
Just like vampires, theriantropes are a special kind of undead that - through their symbiosis with the parasite - can still get to mingle with mortals, appreciate the taste of mortal food and beverages, reproduce sexually and grow into adulthood when turned prematurely when fed with sufficient amount of blood at regular intervals. However, lack of blood causes involuntary transformations and loss of control first, then the decaying of their body.\newline
\begin{tabular}{|c|c|c|}
\hline
 & \textbf{Bonus/Malus} \\ \hline
\textbf{Strength} & \BonusS{4} \\ \hline
\textbf{Endurance} & \BonusS{4}  \\ \hline
\textbf{Dexterity} & \textit{unchanged}  \\ \hline
\textbf{Intelligence} & \textit{unchanged} \\ \hline
\textbf{Willpower} & \textit{unchanged} \\ \hline
\textbf{Charisma} & \textit{unchanged} \\ \hline
\end{tabular}\newline
\UndeadRace{Theriantropes} Instead, they activate a need for Blood, their own unique vital shared with Vampires.\newpage
\chapter{Feats}
\section{Combat Ability Feats}
\subsection{Flurry}
\begin{table}[!ht]
\centering
\FeatIII{Basic}{Advanced}{Expert}{Flurry}{{\import{images/Feats/}{images/Feats/Flurry1.pdf_tex}}}{{\import{images/Feats/}{images/Feats/Flurry2.pdf_tex}}}{{\import{images/Feats/}{images/Feats/Flurry3.pdf_tex}}}
\end{table}
\textbf{Requirements:}
\begin{itemize}
	\item \textbf{Basic:} None, unless your background forbids it.
	\item \textbf{Advanced:} None.
	\item \textbf{Expert:} None.
\end{itemize}
\textbf{Effects:}
\begin{itemize}
	\item \textbf{Untrained:} The character cannot use the combat ability \textit{"Quick Attack"}.
	\item \textbf{Basic:} The character has unlocked the combat ability \textit{"Quick Attack"}. As stated in the basic ruleset, the Quick Attack move allows the character to attack twice in melee during a turn, but at the cost of a \MalusPS{30\%} increase to chance of miss and a \Malus{30\%} to all damage done in melee in case of landing a hit.
	\item \textbf{Advanced:} The aforementioned maluses are reduced to \MalusPS{15\%} and \MalusS{15\%} respectively.
	\item \textbf{Expert:} The aforementioned maluses are gone, nullified.
\end{itemize}\newpage
\subsection{Overswing}
\begin{table}[!ht]
\centering
\FeatIII{Basic}{Advanced}{Expert}{Overswing}{{\import{images/Feats/}{images/Feats/Overdraw1.pdf_tex}}}{{\import{images/Feats/}{images/Feats/Overdraw2.pdf_tex}}}{{\import{images/Feats/}{images/Feats/Overdraw3.pdf_tex}}}
\end{table}
\textbf{Requirements:}
\begin{itemize}
	\item \textbf{Basic:} None, unless your background forbids it.
	\item \textbf{Advanced:} None.
	\item \textbf{Expert:} None.
\end{itemize}
\textbf{Effects:}
\begin{itemize}
	\item \textbf{Untrained:} The character cannot use the combat ability \textit{"Power Attack"}.
	\item \textbf{Basic:} The character has unlocked the combat ability \textit{"Power Attack"}. As stated in the basic ruleset, the Power Attack sacrifices accuracy for strength by getting a \MalusPS{30\%} increase to chance of miss, but also a \Bonus{25\%} to all damage done in melee.
	\item \textbf{Advanced:} The aforementioned malus is reduced to \MalusPS{20\%}, while the bonus is increased to \BonusS{30\%}.
	\item \textbf{Expert:} The aforementioned malus is reduced to \MalusPS{10\%}, while the bonus is increased to \BonusS{35\%}.
\end{itemize}\newpage
\subsection{Chevauchée}
\begin{table}[!ht]
\centering
\FeatIII{Basic}{Advanced}{Expert}{Chevauchée}{{\import{images/Feats/}{images/Feats/Chevauchee1.pdf_tex}}}{{\import{images/Feats/}{images/Feats/Chevauchee2.pdf_tex}}}{{\import{images/Feats/}{images/Feats/Chevauchee3.pdf_tex}}}
\end{table}
\textbf{Requirements:}
\begin{itemize}
	\item \textbf{Basic:} Basic Riding.
	\item \textbf{Advanced:} None.
	\item \textbf{Expert:} None.
\end{itemize}
\textbf{Effects:}
\begin{itemize}
	\item \textbf{Untrained:} The character cannot use the combat ability \textit{"Mounted Attack"}, and is thus forced to rely on Regular Attacks even while mounted.
	\item \textbf{Basic:} The character has unlocked the combat ability \textit{"Mounted Attack"}. As stated in the basic ruleset, the Mounted Attack is only possible while the attacker is mounted and is doing the attack move while moving. This kind of attack comes with a \Bonus{40\%} to damage at the cost of an extra \MalusPS{15\%} chance to miss the target
	\item \textbf{Advanced:} The aforementioned malus is reduced to \MalusPS{10\%}, while the bonus is increased to \BonusS{50\%}.
	\item \textbf{Expert:} The aforementioned malus is reduced is nullified, while the bonus is increased to \BonusS{60\%}.
\end{itemize}\newpage
\subsection{Mounted Archery}
\begin{table}[!ht]
\centering
\FeatIII{Basic}{Advanced}{Expert}{Mounted Archery}{{\import{images/Feats/}{images/Feats/HorseArchery1.pdf_tex}}}{{\import{images/Feats/}{images/Feats/HorseArchery2.pdf_tex}}}{{\import{images/Feats/}{images/Feats/HorseArchery3.pdf_tex}}}
\end{table}
\textbf{Requirements:}
\begin{itemize}
	\item \textbf{Basic:} Basic Riding.
	\item \textbf{Advanced:} None.
	\item \textbf{Expert:} None.
\end{itemize}
\textbf{Effects:}
\begin{itemize}
	\item \textbf{Untrained:} The character suffers a \Malus{4} to Dexterity while trying to use Ranged Weapons on horseback, at least when moving.
	\item \textbf{Basic:} The character suffers a \Malus{2} to Dexterity while trying to use Ranged Weapons on horseback, at least when moving.
	\item \textbf{Advanced:} The character suffers a \Malus{1} to Dexterity while trying to use Ranged Weapons on horseback, at least when moving.
	\item \textbf{Expert:} The character suffers no malus to Dexterity while trying to use Ranged Weapons on horseback, at all, shooting just as well as he/she would if stationary.
\end{itemize}\newpage
\section{Combat Equipment Proficiency Feats}
\subsection{Armour Conditioning}
\begin{table}[!ht]
\centering
\FeatIII{Light}{Medium}{Heavy}{Armour Conditioning}{{\import{images/Feats/}{images/Feats/Armour1.pdf_tex}}}{{\import{images/Feats/}{images/Feats/Armour2.pdf_tex}}}{{\import{images/Feats/}{images/Feats/Armour3.pdf_tex}}}
\end{table}
\textbf{Requirements:}
\begin{itemize}
	\item \textbf{Light:} None, unless your background forbids it.
	\item \textbf{Medium:} None.
	\item \textbf{Heavy:} None.
\end{itemize}
\textbf{Effects:}
\begin{itemize}
	\item \textbf{Untrained:} The character feels uncomfortable in armour, and a \Malus{50\%} to movement speed when wearing any sort of armour. When any calculations are made that involve Dexterity, wearing any sort of armour counts as a \Malus{2} to Dexterity. Magic-using characters also cannot cast spells when wearing any sort of armour.
	\item \textbf{Light:} The character feels comfortable with Light Armour \Parentheses{Gambeson and Leather Armour}, uncomfortable in anything heavier, thus incurring a \Malus{50\%} to movement speed when wearing any armour other than Gambeson and Leather Armour. When any calculations are made that involve Dexterity, wearing Medium or Heavy armour counts as a \Malus{2} to Dexterity. Magic-using characters also cannot cast spells when wearing Medium or Heavy Armour.
	\item \textbf{Medium:} The character feels comfortable with Medium Armour \Parentheses{Chainmail and Scalemail}, uncomfortable in anything heavier, thus incurring a \Malus{50\%} to movement speed when wearing Heavy Armour. When any calculations are made that involve Dexterity, wearing Heavy armour counts as a \Malus{2} to Dexterity. Magic-using characters also cannot cast spells when wearing Heavy Armour.
	\item \textbf{Heavy:} The character feels comfortable with Heavy Armour \Parentheses{Breastplates and Platemail}, incurring no malluses when wearing them. Magic-using characters \textbf{can} cast spells in any type of armour.
\end{itemize}\newpage
\subsection{Swords}
\begin{table}[!ht]
\centering
\FeatIII{Basic}{Advanced}{Expert}{Swords}{{\import{images/Feats/}{images/Feats/Blade1.pdf_tex}}}{{\import{images/Feats/}{images/Feats/Blade2.pdf_tex}}}{{\import{images/Feats/}{images/Feats/Blade3.pdf_tex}}}
\end{table}
\textbf{Requirements:}
\begin{itemize}
	\item \textbf{Basic:} None, unless your background forbids it.
	\item \textbf{Advanced:} None.
	\item \textbf{Expert:} None.
\end{itemize}
\textbf{Effects:}
\begin{itemize}
	\item \textbf{Untrained:} The character always deals minimal damage with swords, no need to roll the dice. They also get a \Malus{2} to Dexterity when it's being counted when using swords.
	\item \textbf{Basic:} The character suffers no bonuses or maluses when using swords.
	\item \textbf{Advanced:} The character reiceves a \Bonus{25\%} to all damage done by swords, and a \Bonus{25\%} to chance to hit or cause critical damage.
	\item \textbf{Expert:} The character reiceves a \Bonus{50\%} to all damage done by swords, and a \Bonus{50\%} to chance to hit or cause critical damage.
\end{itemize}\newpage
\subsection{Bludgeoners}
\begin{table}[!ht]
\centering
\FeatIII{Basic}{Advanced}{Expert}{Bludgeoners}{{\import{images/Feats/}{images/Feats/Blunt1.pdf_tex}}}{{\import{images/Feats/}{images/Feats/Blunt2.pdf_tex}}}{{\import{images/Feats/}{images/Feats/Blunt3.pdf_tex}}}
\end{table}
\textbf{Requirements:}
\begin{itemize}
	\item \textbf{Basic:} None, unless your background forbids it.
	\item \textbf{Advanced:} None.
	\item \textbf{Expert:} None.
\end{itemize}
\textbf{Effects:}
\begin{itemize}
	\item \textbf{Untrained:} The character always deals minimal damage with so-called \SoCalled{striking weapons} \Parentheses{clubs/cudgels, maces, hammers, axes}, no need to roll the dice. They also get a \Malus{2} to Dexterity when it's being counted when using bludgeoners.
	\item \textbf{Basic:} The character suffers no bonuses or maluses when using so-called \SoCalled{striking weapons} \Parentheses{clubs/cudgels, maces, hammers, axes}.
	\item \textbf{Advanced:} The character reiceves a \Bonus{25\%} to all damage done by \SoCalled{striking weapons} \Parentheses{clubs/cudgels, maces, hammers, axes}, and a \Bonus{25\%} to chance to hit or cause critical damage.
	\item \textbf{Expert:} The character reiceves a \Bonus{50\%} to all damage done by \SoCalled{striking weapons} \Parentheses{clubs/cudgels, maces, hammers, axes}, and a \Bonus{50\%} to chance to hit or cause critical damage.
\end{itemize}\newpage
\subsection{Polearms}
\begin{table}[!ht]
\centering
\FeatIII{Basic}{Advanced}{Expert}{Polearms}{{\import{images/Feats/}{images/Feats/Stick1.pdf_tex}}}{{\import{images/Feats/}{images/Feats/Stick2.pdf_tex}}}{{\import{images/Feats/}{images/Feats/Stick3.pdf_tex}}}
\end{table}
\textbf{Requirements:}
\begin{itemize}
	\item \textbf{Basic:} None, unless your background forbids it.
	\item \textbf{Advanced:} None.
	\item \textbf{Expert:} None.
\end{itemize}
\textbf{Effects:}
\begin{itemize}
	\item \textbf{Untrained:} The character always deals minimal damage with so-called \SoCalled{polearms} \Parentheses{quarterstaffs, spears, pikes, halberds, pollaxes, glaives, voulges and bills}, no need to roll the dice. They also get a \Malus{2} to Dexterity when it's being counted when using polearms.
	\item \textbf{Basic:} The character suffers no bonuses or maluses when using so-called \SoCalled{polearms} \Parentheses{quarterstaffs, spears, pikes, halberds, pollaxes, glaives, voulges and bills}.
	\item \textbf{Advanced:} The character reiceves a \Bonus{25\%} to all damage done by \SoCalled{polearms} \Parentheses{quarterstaffs, spears, pikes, halberds, pollaxes, glaives, voulges and bills}, and a \Bonus{25\%} to chance to hit or cause critical damage.
	\item \textbf{Expert:} The character reiceves a \Bonus{50\%} to all damage done by \SoCalled{polearms} \Parentheses{quarterstaffs, spears, pikes, halberds, pollaxes, glaives, voulges and bills}, and a \Bonus{50\%} to chance to hit or cause critical damage.
\end{itemize}\newpage
\subsection{Bows}
\begin{table}[!ht]
\centering
\FeatIII{Basic}{Advanced}{Expert}{Bows}{{\import{images/Feats/}{images/Feats/Ranged1.pdf_tex}}}{{\import{images/Feats/}{images/Feats/Ranged2.pdf_tex}}}{{\import{images/Feats/}{images/Feats/Ranged3.pdf_tex}}}
\end{table}
\textbf{Requirements:}
\begin{itemize}
	\item \textbf{Basic:} None, unless your background forbids it.
	\item \textbf{Advanced:} None.
	\item \textbf{Expert:} None.
\end{itemize}
\textbf{Effects:}
\begin{itemize}
	\item \textbf{Untrained:} The character always deals minimal damage with bows, no need to roll the dice. They also get a \Malus{4} to Dexterity when it's being counted when using bows. In video game adaptations, this should halve the effective range of all bows.
	\item \textbf{Basic:} The character suffers no bonuses or maluses when using bows.
	\item \textbf{Advanced:} The character reiceves a \Bonus{25\%} to all damage done by bows, and a \Bonus{25\%} to chance to hit or cause critical damage. In video game adaptations, this should also increase range.
	\item \textbf{Expert:} The character reiceves a \Bonus{50\%} to all damage done by bows, and a \Bonus{50\%} to chance to hit or cause critical damage. In video game adaptations, this should also increase range.
\end{itemize}\newpage
\subsection{Crossbows}
\begin{table}[!ht]
\centering
\FeatIII{Basic}{Advanced}{Expert}{Crossbows}{{\import{images/Feats/}{images/Feats/Crossbow1.pdf_tex}}}{{\import{images/Feats/}{images/Feats/Crossbow2.pdf_tex}}}{{\import{images/Feats/}{images/Feats/Crossbow3.pdf_tex}}}
\end{table}
\textbf{Requirements:}
\begin{itemize}
	\item \textbf{Basic:} None, unless your background forbids it.
	\item \textbf{Advanced:} None.
	\item \textbf{Expert:} None.
\end{itemize}
\textbf{Effects:}
\begin{itemize}
	\item \textbf{Untrained:} The character needs to skip a turn before using a crossbow to reload every single time they want to shoot. They also get a \Malus{4} to Dexterity when it's being counted when using crossbows. In video game adaptations, this should halve the effective range of all bows.
	\item \textbf{Basic:} The character suffers no bonuses or maluses when using crossbows, and can reload fast enough to not to skip a turn doing it.
	\item \textbf{Advanced:} The character reiceves a \Bonus{25\%} to all damage done by crossbows, and a \Bonus{25\%} to chance to hit or cause critical damage. In video game adaptations, this should also increase range.
	\item \textbf{Expert:} The character reiceves a \Bonus{50\%} to all damage done by crossbows, and a \Bonus{50\%} to chance to hit or cause critical damage. In video game adaptations, this should also increase range.
\end{itemize}\newpage
\subsection{Firearms}
\begin{table}[!ht]
\centering
\FeatIII{Basic}{Advanced}{Expert}{Firearms}{{\import{images/Feats/}{images/Feats/Gun1.pdf_tex}}}{{\import{images/Feats/}{images/Feats/Gun2.pdf_tex}}}{{\import{images/Feats/}{images/Feats/Gun3.pdf_tex}}}
\end{table}
\textbf{Requirements:}
\begin{itemize}
	\item \textbf{Basic:} None, unless your background forbids it.
	\item \textbf{Advanced:} None.
	\item \textbf{Expert:} None.
\end{itemize}
\textbf{Effects:}
\begin{itemize}
	\item \textbf{Untrained:} The character doesn't know how to use firearms, and even when shown, will easily forget it. Acquired knowledge lasts less than a day, and even so, the character takes at least two turns to reload a gun before being able to use it. In video game adaptations, this should also halve the gun's effective range.
	\item \textbf{Basic:} The character got the hang of using firearms, and takes just one turn to reload a gun before using it. 
	\item \textbf{Advanced:} The character reiceves a \Bonus{25\%} to all damage done by guns, and a \Bonus{25\%} to chance to hit or cause critical damage. In video game adaptations, this should also increase range.
	\item \textbf{Expert:} The character reiceves a \Bonus{50\%} to all damage done by guns, and a \Bonus{50\%} to chance to hit or cause critical damage. In video game adaptations, this should also increase range.
\end{itemize}\newpage
\subsection{Exotic Weapons}
\begin{table}[!ht]
\centering
\FeatIII{Basic}{Advanced}{Expert}{Exotic Weapons}{{\import{images/Feats/}{images/Feats/Exotic1.pdf_tex}}}{{\import{images/Feats/}{images/Feats/Exotic2.pdf_tex}}}{{\import{images/Feats/}{images/Feats/Exotic3.pdf_tex}}}
\end{table}
Exotic weapons are weapons that don't fit into any of the aforementioned categories. In fact, most characters probably wouldn't recognize any of these as weapons first, or would be largely ignorant about its attributes, whether we're talking about an existing weapon or a character's own creation.
\textbf{Requirements:}
\begin{itemize}
	\item \textbf{Basic:} None, unless your background forbids it.
	\item \textbf{Advanced:} None.
	\item \textbf{Expert:} None.
\end{itemize}
\textbf{Effects:}
\begin{itemize}
	\item \textbf{Untrained:} The character likely doesn't even recognize the item as a weapon to begin with - but even if he/she does, the character always has a 50\% chance of inflicting damage to themselves instead of the enemy while using the weapon - and when the character doesn't inflict damage on themselves, there is a 50\% chance of missing the enemy still.
	\item \textbf{Basic:} The character can recognize exotic weapons, and knows a bit about them. The character doesn't inflict damage on themselves when using exotic weapons, but still has a minimum of 25\% chance of missing at each attack.
	\item \textbf{Advanced:} The character receives no bonuses or malluses when using exotic weapons. There is no minimum chance of missing the enemy anymore, so chances of missing are purely dictated by the user's dexterity, the target's dexterity and various other factors.
	\item \textbf{Expert:} The character reiceves a \Bonus{50\%} to all damage done by exotic weapons.
\end{itemize}\newpage
\subsection{Martial Arts}
\begin{table}[!ht]
\centering
\FeatIII{Basic}{Advanced}{Expert}{Martial Arts}{{\import{images/Feats/}{images/Feats/Unarmed1.pdf_tex}}}{{\import{images/Feats/}{images/Feats/Unarmed2.pdf_tex}}}{{\import{images/Feats/}{images/Feats/Unarmed3.pdf_tex}}}
\end{table}
\textbf{Requirements:}
\begin{itemize}
	\item \textbf{Basic:} None, unless your background forbids it.
	\item \textbf{Advanced:} None.
	\item \textbf{Expert:} None.
\end{itemize}
\textbf{Effects:}
\begin{itemize}
	\item \textbf{Untrained:} The character doesn't know any martial arts at all, only basic moves \Parentheses{such as punching, slapping, kicking, tripping, biting, headbutting, etc.} - in unarmed combat, the character is forced to rely purely on their physical strength, which isn't necessarily a bad thing, if their Strength attribute is high.
	\item \textbf{Basic:} The character knows some martial arts at a beginner level - in other words the character's dexterity counts towards their unarmed damage - if the character's dexterity is higher than their strength, strength is ignored, and dexterity is used in calculations instead. Additionally, the character gains a \Bonus{10\%} to all unarmed melee damage.
	\item \textbf{Advanced:} The character is rather good at martial arts - in other words, when the character does unarmed damage, their actual strength is substituted with 75\% of a combination of their Strength and Dexterity. Additionally, the character gains a \Bonus{20\%} to all unarmed melee damage.
	\item \textbf{Expert:} The character is a master at martial arts - in other words, when the character does unarmed damage, during the calculation, their actual strength is substituted with a sum of their Strength and Dexterity. Additionally, the character gains a \Bonus{30\%} to all unarmed melee damage.
\end{itemize}\newpage
\section{Logistic Feats}
\subsection{Sneaking}
\begin{table}[!ht]
\centering
\FeatIII{Basic}{Advanced}{Expert}{Sneaking}{{\import{images/Feats/}{images/Feats/Stealth1.pdf_tex}}}{{\import{images/Feats/}{images/Feats/Stealth2.pdf_tex}}}{{\import{images/Feats/}{images/Feats/Stealth3.pdf_tex}}}
\end{table}
\textbf{Requirements:}
\begin{itemize}
	\item \textbf{Basic:} None, unless your background forbids it.
	\item \textbf{Advanced:} None.
	\item \textbf{Expert:} None.
\end{itemize}
\textbf{Effects:}
\begin{itemize}
	\item \textbf{Untrained:} The character cannot really sneak around, and is thus easily noticed.
	\item \textbf{Basic:} The character can move silently, and can sneak around less vigilant enemies. The character gains a \Bonus{25\%} to any damage to any stealth attacks.
	\item \textbf{Advanced:} The character can move silently and manipulate their environment without making much noise. The character gains a \Bonus{50\%} to any damage to any stealth attacks.
	\item \textbf{Expert:} The character can particularly move in the shadows completely unseen and unheard. The character gains a \Bonus{100\%} to any damage to any stealth attacks.
\end{itemize}\newpage
\subsection{Swimming}
\begin{table}[!ht]
\centering
\FeatIII{Basic}{Advanced}{Expert}{Swimming}{{\import{images/Feats/}{images/Feats/Swimming1.pdf_tex}}}{{\import{images/Feats/}{images/Feats/Swimming2.pdf_tex}}}{{\import{images/Feats/}{images/Feats/Swimming3.pdf_tex}}}
\end{table}
\textbf{Requirements:}
\begin{itemize}
	\item \textbf{Basic:} None, unless your background forbids it.
	\item \textbf{Advanced:} None.
	\item \textbf{Expert:} None.
\end{itemize}
\textbf{Effects:}
\begin{itemize}
	\item \textbf{Untrained:} The character can't swim in any kind of water that is deeper than the height of their body below the neck. If they get into water that is any deeper, they'll struggle and drown. They don't even know anything about holding their breath underwater.
	\item \textbf{Basic:} The character can swim in and float above water that's slightly deeper than their own height, knows how to hold their breath underwater, and knows the basics about swimming techniques. They are rather slow and clumsy swimmers, but at least they aren't at the risk of drowning in shallow waters.
	\item \textbf{Advanced:} The character is a competent swimmer. They are competent enough to take some equipment with themselves. The character can also swim through a slower-flowing river.
	\item \textbf{Expert:} The character can easily swim like a dolphin for hours, taking their equipment with them. However, they are still incapable of swimming in metal armour.
\end{itemize}\newpage
\subsection{Riding}
\begin{table}[!ht]
\centering
\FeatI{Basic Riding}{{\import{images/Feats/}{images/Feats/Riding.pdf_tex}}}
\end{table}
\textbf{Requirements:}
\begin{itemize}
	\item \textbf{Basic:} None, unless your background forbids it.
\end{itemize}
\textbf{Effects:}
\begin{itemize}
	\item \textbf{Untrained:} The character has probably never ridden a mount before, and even if they did, they quickly forget all the lessons learned. They don't really understand how to communicate with their mount. When they say "Yah!" to the animal, it won't move one inch. 
	\item \textbf{Basic:} A character with basic riding skills has a broad understanding of how to mount most mounts and get them to move in the direction the character wants. Some animals are harder to control than others, but as long as the animal in question is already tame to begin with, the character should have no problems mounting it and riding it after some familiarization - of course, mounted combat is a different matter altogether, and a different skill.
\end{itemize}\newpage
\subsection{Horsemanship}
\begin{table}[!ht]
\centering
\FeatII{Basic}{Advanced}{Horsemanship}{{\import{images/Feats/}{images/Feats/Horsemanship1.pdf_tex}}}{{\import{images/Feats/}{images/Feats/Horsemanship2.pdf_tex}}}
\end{table}
\textbf{Requirements:}
\begin{itemize}
	\item \textbf{Advanced:} The character must possess Basic Riding.
	\item \textbf{Expert:} None, as long as Advanced Horsemanship is already possessed.
\end{itemize}
\textbf{Effects:}
\begin{itemize}
	\item \textbf{Untrained:} Assuming the character already has Basic Riding, they can ride horses and other horse-like animals, but aren't specialized at riding them, and can ride them just about as well as they can ride any other tame animals large and strong enough to bear with their weight. Otherwise, refer to the description of Untrained Riding.
	\item \textbf{Advanced:} The character has grown specialized in riding horses and other horse-like creatures, thus knows more about them than the average rider, is qualified to compete in races. They also familiar themselves with their horse much faster.
	\item \textbf{Expert:} The character controls their horse instinctively, forming a bond with their mount rather quickly.
\end{itemize}
\subsection{Cameleering}
\begin{table}[!ht]
\centering
\FeatII{Basic}{Advanced}{Cameleering}{{\import{images/Feats/}{images/Feats/Camel1.pdf_tex}}}{{\import{images/Feats/}{images/Feats/Camel2.pdf_tex}}}
\end{table}
\textbf{Requirements:}
\begin{itemize}
	\item \textbf{Advanced:} The character must possess Basic Riding.
	\item \textbf{Expert:} None, as long as Advanced Cameleering is already possessed.
\end{itemize}
\textbf{Effects:}
\begin{itemize}
	\item \textbf{Untrained:} Assuming the character already has Basic Riding, they can ride camels and other camel-like animals, but aren't specialized at riding them, and can ride them just about as well as they can ride any other tame animals large and strong enough to bear with their weight. Otherwise, refer to the description of Untrained Riding.
	\item \textbf{Advanced:} The character has grown specialized in riding camels and other camel-like creatures, thus knows more about them than the average rider, is qualified to compete in races. They also familiar themselves with their camel much faster.
	\item \textbf{Expert:} The character controls their camel instinctively, forming a bond with their mount rather quickly.
\end{itemize}\newpage
\subsection{Raptoring}
\begin{table}[!ht]
\centering
\FeatII{Basic}{Advanced}{Raptoring}{{\import{images/Feats/}{images/Feats/Raptor1.pdf_tex}}}{{\import{images/Feats/}{images/Feats/Raptor2.pdf_tex}}}
\end{table}
\textbf{Requirements:}
\begin{itemize}
	\item \textbf{Advanced:} The character must possess Basic Riding.
	\item \textbf{Expert:} None, as long as Advanced Raptoring is already possessed.
\end{itemize}
\textbf{Effects:}
\begin{itemize}
	\item \textbf{Untrained:} Even the character already knows the Basics of Riding, they still can't control - or at the very least, struggle heavily with - controlling a mount that also happens to be a carnivore, such as basilisks.
	\item \textbf{Advanced:} The character can control a carnivorous mount somewhat effectively.
	\item \textbf{Expert:} The character controls their carnivorous mount instinctively, forming a bond with their mount rather quickly.
\end{itemize}\newpage
\subsection{Forgery}
\begin{table}[!ht]
\centering
\FeatIII{Basic}{Advanced}{Expert}{Forgery}{{\import{images/Feats/}{images/Feats/Forgery1.pdf_tex}}}{{\import{images/Feats/}{images/Feats/Forgery2.pdf_tex}}}{{\import{images/Feats/}{images/Feats/Forgery3.pdf_tex}}}
\end{table}
\textbf{Requirements:}
\begin{itemize}
	\item \textbf{Basic:} None, unless your background forbids it.
	\item \textbf{Advanced:} None.
	\item \textbf{Expert:} None.
\end{itemize}
\textbf{Effects:}
\begin{itemize}
	\item \textbf{Untrained:} The character cannot do any forgery.
	\item \textbf{Basic:} The character can make forgeries that will fool idiots and maybe some ordinary people.
	\item \textbf{Advanced:} The character can make forgeries that can fool most people, save for those with a good eye for authentic merchandise.
	\item \textbf{Expert:} The character can make forgeries that will fool just about everyone, save for true experts.
\end{itemize}\newpage
\subsection{Arithmetics}
\begin{table}[!ht]
\centering
\FeatIII{Basic}{Advanced}{Expert}{Arithmetics}{{\import{images/Feats/}{images/Feats/Arithmetic1.pdf_tex}}}{{\import{images/Feats/}{images/Feats/Arithmetic2.pdf_tex}}}{{\import{images/Feats/}{images/Feats/Arithmetic3.pdf_tex}}}
\end{table}
\textbf{Requirements:}
\begin{itemize}
	\item \textbf{Basic:} None.
	\item \textbf{Advanced:} None.
	\item \textbf{Expert:} None.
\end{itemize}
\textbf{Effects:}
\begin{itemize}
	\item \textbf{Untrained:} The character can count up to maybe ten, and can literally put two and two together, but not much beyond that.
	\item \textbf{Basic:} The character is acquited with mathematics well enough to count up to hundreds, add, negate, multiply and divide with relative ease.
	\item \textbf{Advanced:} The character isn't intimidated by big numbers, can perform just about all mathematical operations that are can be broken down to primitive functions \textit{(addition, negation, multiplication, division, modulo, even possibly integer exponentiaton)}, albeit possibly rather slowly. With the aid of a lookup table, even trigonometry is possible. Nevertheless, complex equations still confuse the character. The character is also good with boolean algebra.
	\item \textbf{Expert:} The character can not only perform calculations by hand at a reasonable speed, but is also good at comprehending more complex equations - albeit, trigonometry is still going to be slow and difficult without the aid of a lookup table. The character might also know about some of the more obscure sides of mathematics, such as complex numbers and differential algebra \textit{(derivation and integration)}.
\end{itemize}\newpage
\subsection{Seamanship}
\begin{table}[!ht]
\centering
\FeatIII{Basic}{Advanced}{Expert}{Seamanship}{{\import{images/Feats/}{images/Feats/Semenship1.pdf_tex}}}{{\import{images/Feats/}{images/Feats/Semenship2.pdf_tex}}}{{\import{images/Feats/}{images/Feats/Semenship3.pdf_tex}}}
\end{table}
\textbf{Requirements:}
\begin{itemize}
	\item \textbf{Basic:} None, unless your background forbids it.
	\item \textbf{Advanced:} None.
	\item \textbf{Expert:} None.
\end{itemize}
\textbf{Effects:}
\begin{itemize}
	\item \textbf{Untrained:} The character has no idea how to operate a ship, and lacks all the skill needed to be a competent sailor, let alone a captain.
	\item \textbf{Basic:} The character knows the basics of how to handle oneself on a ship, knows the enough of a typical ship's anatomy to know what's up when things happen, and knows something about rudementary navigation without a compass - they'll still likely to get lost in the open sea if misfortune falls upon the crew and fate puts the character in charge of a ship. The character makes an okay sailor, but probably a terrible captain. \textit{(Up to the GM to enforce)}
	\item \textbf{Advanced:} The character knows how to operate most vessels, knows a thing or two about meteorology and weather forecasting, and in case a crisis puts them in charge of a ship, they also know about watchkeeping \textit{(the assignment of sailors to specific roles on a ship to operate it continuously)}. As a sailor, the character knows what he/she is doing, but as a captain - average at best. \textit{(Up to the GM to enforce)}
	\item \textbf{Expert:} The character not only knows the open sea like the palm of their own hands, but can intuit their direction, the distance of nearest land, the time of the next storm, and various other things that come handy when sailing. The character knows just about everything that can be known about operating and commanding a single vessel, though not necessarily a whole fleet. Thus, the character makes a fine captain, but not necessarily an admiral - that depends more on their general leadership skills. \textit{(Up to the GM to enforce)}
\end{itemize}\newpage
\subsection{Tracking}
\begin{table}[!ht]
\centering
\FeatIII{Basic}{Advanced}{Expert}{Tracking}{{\import{images/Feats/}{images/Feats/Tracking1.pdf_tex}}}{{\import{images/Feats/}{images/Feats/Tracking2.pdf_tex}}}{{\import{images/Feats/}{images/Feats/Tracking3.pdf_tex}}}
\end{table}
\textbf{Requirements:}
\begin{itemize}
	\item \textbf{Basic:} None, unless your background forbids it.
	\item \textbf{Advanced:} None.
	\item \textbf{Expert:} None.
\end{itemize}
\textbf{Effects:}
\begin{itemize}
	\item \textbf{Untrained:} The character doesn't know how to follow animal tracks, easily loses sight of tracks, etc.
	\item \textbf{Basic:} The character knows how to follow tracks no older than one day.
	\item \textbf{Advanced:} The character can learn much more from the tracks.
	\item \textbf{Expert:} The character makes a good detective.
\end{itemize}\newpage
\subsection{Naturalism}
\begin{table}[!ht]
\centering
\FeatIII{Basic}{Advanced}{Expert}{Naturalism}{{\import{images/Feats/}{images/Feats/Nature1.pdf_tex}}}{{\import{images/Feats/}{images/Feats/Nature2.pdf_tex}}}{{\import{images/Feats/}{images/Feats/Nature3.pdf_tex}}}
\end{table}
\textbf{Requirements:}
\begin{itemize}
	\item \textbf{Basic:} None, unless your background forbids it.
	\item \textbf{Advanced:} None.
	\item \textbf{Expert:} None.
\end{itemize}
\textbf{Effects:}
\begin{itemize}
	\item \textbf{Untrained:} The character is utterly lost outside of the bounds of civilization. They can't even tell edible berries apart from poisonous ones! And they'll likely freeze to death on cold nights.
	\item \textbf{Basic:} The character can tell edible plants apart from inedible ones, knows how to use a flintstone to make fire, is informed about the habits of wild animals, etc.
	\item \textbf{Advanced:} The character knows how to survive in harsher environments, like high mountains and semidesert environments. They also know that grasshoppers are edible.
	\item \textbf{Expert:} The character knows how to survive and food nutrition even in the roots of the most extreme environments \Parentheses{sandy deserts and literal ice fields}, finding water and safe shelter alike. The only real challenge would be a land cursed by magic.
\end{itemize}\newpage
\section{Magic Feats}
\subsection{Magica Profana}
\begin{table}[!ht]
\centering
\FeatI{Magica Profana}{{\import{images/Feats/}{images/Feats/Wizardry.pdf_tex}}}
\end{table}
\textbf{Requirements:}
\begin{itemize}
	\item \textbf{Trained:} Character must not have the feat \textbf{Magica Divinitatis} or \textbf{Magica Naturae} in order to take up this feat. Additionally, in a serious and realistic tabletop setting - where no one can just learn magic overnight - it should be up to the GM's discression to decide whether characters who didn't already have this feat in the first place should be allowed to take it up during an adventure.
\end{itemize}
\textbf{Effects:}
\begin{itemize}
	\item \textbf{Untrained:} The character cannot use Arcane Magic.
	\item \textbf{Trained:} The character can use Arcane Magic, and can learn spells and other magic-related feats. The character also gains the Basic Spells \textit{(Telekinessis, Lighting, Energy Bolt)}.
\end{itemize}\newpage
\subsection{Magica Divinitatis}
\begin{table}[!ht]
\centering
\FeatI{Magica Divinitatis}{{\import{images/Feats/}{images/Feats/Clericalism.pdf_tex}}}
\end{table}
\textbf{Requirements:}
\begin{itemize}
	\item \textbf{Trained:} Character must not have the feat \textbf{Magica Profana} or \textbf{Magica Naturae} in order to take up this feat. Additionally, in a serious and realistic tabletop setting - where no one can just learn magic overnight - it should be up to the GM's discression to decide whether characters who didn't already have this feat in the first place should be allowed to take it up during an adventure \textit{(and even so, only religious characters should be allowed to take this up)}.
\end{itemize}
\textbf{Effects:}
\begin{itemize}
	\item \textbf{Untrained:} The character cannot use Clerical Magic.
	\item \textbf{Trained:} The character can use Clerical Magic, and can learn spells and other magic-related feats - albeit he/she is limited by what his/her religion allows. The character also gains the Basic Spells \textit{(Telekinessis, Lighting, Energy Bolt)}.
\end{itemize}\newpage
\subsection{Magica Naturae}
\begin{table}[!ht]
\centering
\FeatI{Magica Naturae}{{\import{images/Feats/}{images/Feats/Druidism.pdf_tex}}}
\end{table}
\textbf{Requirements:}
\begin{itemize}
	\item \textbf{Trained:} Character must not have the feat \textbf{Magica Divinitatis} or \textbf{Magica Divinitatis} in order to take up this feat. Additionally, in a serious and realistic tabletop setting - where no one can just learn magic overnight - it should be up to the GM's discression to decide whether characters who didn't already have this feat in the first place should be allowed to take it up during an adventure.
\end{itemize}
\textbf{Effects:}
\begin{itemize}
	\item \textbf{Untrained:} The character cannot use Nature Magic.
	\item \textbf{Trained:} The character can use Nature Magic, and can learn spells and other magic-related feats. The character also gains the Basic Spells \textit{(Telekinessis, Lighting, Energy Bolt)}.
\end{itemize}\newpage
\subsection{Alchemy}
\subsection{Enchantment}
\subsection{Photomancy}
\begin{table}[!ht]
\centering
\FeatIII{Basic}{Advanced}{Expert}{Photomancy}{{\import{images/Feats/}{images/Feats/LightMagic1.pdf_tex}}}{{\import{images/Feats/}{images/Feats/LightMagic2.pdf_tex}}}{{\import{images/Feats/}{images/Feats/LightMagic3.pdf_tex}}}
\end{table}
\textbf{Requirements:}
\begin{itemize}
	\item \textbf{Basic:} The character already possessing the feat \textbf{Magica Profana} \textit{(enables Arcane Magic)}, \textbf{Magica Divinitatis} \textit{(enables Clerical Magic)}, or \textbf{Magica Naturae} \textit{(enables Nature Magic)}. For users of Clerical Magic, the ability to take up this feat depends on their religion.
	\item \textbf{Advanced:} None \textit{(provided they already have this feat in Basic)}.
	\item \textbf{Expert:} None \textit{(provided they already have this feat in Advanced)}.
\end{itemize}
\textbf{Effects:}
\begin{itemize}
	\item \textbf{Untrained:} The character cannot learn any Light Magic spells.
	\item \textbf{Basic:} The character can learn Basic-level Light Magic spells, and also gains a bonus Basic-level Light Magic spell of their chosing.
	\item \textbf{Advanced:} The character can learn Advanced-level Light Magic spells, and gets a \Bonus{10\%} to the effectiveness of their Basic-tier Light Magic spells. The character also gains a bonus Advanced-level Light Magic spell of their chosing.
	\item \textbf{Expert:} The character can learn Expert-level Light Magic spells, and gets a \Bonus{30\%} to the effectiveness of their Basic-tier Light Magic spells and \Bonus{15\%} to the effectiveness of their Advanced-tier Light Magic spells. The character also gains a bonus Expert-level Light Magic spell of their chosing.
\end{itemize}\newpage
\subsection{Sciomancy}
\begin{table}[!ht]
\centering
\FeatIII{Basic}{Advanced}{Expert}{Sciomancy}{{\import{images/Feats/}{images/Feats/DarkMagic1.pdf_tex}}}{{\import{images/Feats/}{images/Feats/DarkMagic2.pdf_tex}}}{{\import{images/Feats/}{images/Feats/DarkMagic3.pdf_tex}}}
\end{table}
\textbf{Requirements:}
\begin{itemize}
	\item \textbf{Basic:} The character already possessing the feat \textbf{Magica Profana} \textit{(enables Arcane Magic)}, \textbf{Magica Divinitatis} \textit{(enables Clerical Magic)}, or \textbf{Magica Naturae} \textit{(enables Nature Magic)}. For users of Clerical Magic, the ability to take up this feat depends on their religion.
	\item \textbf{Advanced:} None \textit{(provided they already have this feat in Basic)}.
	\item \textbf{Expert:} None \textit{(provided they already have this feat in Advanced)}.
\end{itemize}
\textbf{Effects:}
\begin{itemize}
	\item \textbf{Untrained:} The character cannot learn any Dark Magic spells.
	\item \textbf{Basic:} The character can learn Basic-level Dark Magic spells, and also gains a bonus Basic-level Dark Magic spell of their chosing.
	\item \textbf{Advanced:} The character can learn Advanced-level Dark Magic spells, and gets a \Bonus{10\%} to the effectiveness of their Basic-tier Dark Magic spells. The character also gains a bonus Advanced-level Dark Magic spell of their chosing.
	\item \textbf{Expert:} The character can learn Expert-level Dark Magic spells, and gets a \Bonus{30\%} to the effectiveness of their Basic-tier Dark Magic spells and \Bonus{15\%} to the effectiveness of their Advanced-tier Dark Magic spells. The character also gains a bonus Expert-level Dark Magic spell of their chosing.
\end{itemize}\newpage
\subsection{Necromancy}
\begin{table}[!ht]
\centering
\FeatIII{Basic}{Advanced}{Expert}{Necromancy}{{\import{images/Feats/}{images/Feats/DeathMagic1.pdf_tex}}}{{\import{images/Feats/}{images/Feats/DeathMagic2.pdf_tex}}}{{\import{images/Feats/}{images/Feats/DeathMagic3.pdf_tex}}}
\end{table}
\textbf{Requirements:}
\begin{itemize}
	\item \textbf{Basic:} The character already having Basic Sciomancy.
	\item \textbf{Advanced:} None
	\item \textbf{Expert:} The character already having Advanced Sciomancy.
\end{itemize}
\textbf{Effects:}
\begin{itemize}
	\item \textbf{Untrained:} The character cannot revive deceased people and animals as Undead.
	\item \textbf{Basic:} The character can revive dead people and animals as Skeletons and Zombies. The character may control only 5 undead creatures at once.
	\item \textbf{Advanced:} The character can entrap the recently deceased as Ghosts/Wraiths/Wights in their service. The character may control 15 undead creatures at once.
	\item \textbf{Expert:} The character can revive the dead in pristine condition, albeit without their souls. They can also control an unlimited amount of undead creatures at once, potentially an army!
\end{itemize}\newpage
\subsection{Olethromancy}
\begin{table}[!ht]
\centering
\FeatIII{Basic}{Advanced}{Expert}{Olethromancy}{{\import{images/Feats/}{images/Feats/DestructionMagic1.pdf_tex}}}{{\import{images/Feats/}{images/Feats/DestructionMagic2.pdf_tex}}}{{\import{images/Feats/}{images/Feats/DestructionMagic3.pdf_tex}}}
\end{table}
\textbf{Requirements:}
\begin{itemize}
	\item \textbf{Basic:} The character already possessing the feat \textbf{Magica Profana}.
	\item \textbf{Advanced:} None \textit{(provided they already have this feat in Basic)}.
	\item \textbf{Expert:} None \textit{(provided they already have this feat in Advanced)}.
\end{itemize}
\textbf{Effects:}
\begin{itemize}
	\item \textbf{Untrained:} The character cannot cast Destruction Magic spells, unless he/she has the feat necessary for the corresponding element. The majority of Destruction Magic spells also belong to a school of Elemental Magic \Parentheses{Fire Magic, Water Magic, Air Maqgic or Earth Magic}, thus, a spell that belongs to both schools requires only one of the two feats. Only a small minority of Destruction Magic spells don't also belong to a school of Elemental Magic.
	\item \textbf{Basic:} The character can learn Basic-level Destruction Magic spells, and also gains a bonus Basic-level Destruction Magic spell of their chosing.
	\item \textbf{Advanced:} The character can learn Advanced-level Destruction Magic spells, and gets a \Bonus{10\%} to the effectiveness of their Basic-Destruction Destruction Magic spells. The character also gains a bonus Advanced-level Destruction Magic spell of their chosing. For Destruction Magic spells that also belong to one of the schools of Elemental Magic, this bonus also stacks up with the feat relevant to the corresponding element.
	\item \textbf{Expert:} The character can learn Expert-level Destruction Magic spells, and gets a \Bonus{30\%} to the effectiveness of their Basic-tier Destruction Magic spells and \Bonus{15\%} to the effectiveness of their Advanced-tier Destruction Magic spells. The character also gains a bonus Expert-level Destruction Magic spell of their chosing. For Destruction Magic spells that also belong to one of the schools of Elemental Magic, this bonus also stacks up with the feat relevant to the corresponding element.
\end{itemize}\newpage
\subsection{Pyromancy}
\begin{table}[!ht]
\centering
\FeatIII{Basic}{Advanced}{Expert}{Pyromancy}{{\import{images/Feats/}{images/Feats/FireMagic1.pdf_tex}}}{{\import{images/Feats/}{images/Feats/FireMagic2.pdf_tex}}}{{\import{images/Feats/}{images/Feats/FireMagic3.pdf_tex}}}
\end{table}
\textbf{Requirements:}
\begin{itemize}
	\item \textbf{Basic:} The character already possessing the feat \textbf{Magica Profana} \textit{(enables Arcane Magic)}, \textbf{Magica Divinitatis} \textit{(enables Clerical Magic)}, or \textbf{Magica Naturae} \textit{(enables Nature Magic)}. For users of Clerical Magic, the ability to take up this feat depends on their religion.
	\item \textbf{Advanced:} None \textit{(provided they already have this feat in Basic)}.
	\item \textbf{Expert:} None \textit{(provided they already have this feat in Advanced)}.
\end{itemize}
\textbf{Effects:}
\begin{itemize}
	\item \textbf{Untrained:} The character cannot learn any Fire Magic spells.
	\item \textbf{Basic:} The character can learn Basic-level Fire Magic spells, and also gains a bonus Basic-level Fire Magic spell of their chosing.
	\item \textbf{Advanced:} The character can learn Advanced-level Fire Magic spells, and gets a \Bonus{10\%} to the effectiveness of their Basic-tier Fire Magic spells. The character also gains a bonus Advanced-level Fire Magic spell of their chosing.
	\item \textbf{Expert:} The character can learn Expert-level Fire Magic spells, and gets a \Bonus{30\%} to the effectiveness of their Basic-tier Fire Magic spells and \Bonus{15\%} to the effectiveness of their Advanced-tier Fire Magic spells. The character also gains a bonus Expert-level Fire Magic spell of their chosing.
\end{itemize}\newpage
\subsection{Hydromancy}
\begin{table}[!ht]
\centering
\FeatIII{Basic}{Advanced}{Expert}{Hydromancy}{{\import{images/Feats/}{images/Feats/WaterMagic1.pdf_tex}}}{{\import{images/Feats/}{images/Feats/WaterMagic2.pdf_tex}}}{{\import{images/Feats/}{images/Feats/WaterMagic3.pdf_tex}}}
\end{table}
\textbf{Requirements:}
\begin{itemize}
	\item \textbf{Basic:} The character already possessing the feat \textbf{Magica Profana} \textit{(enables Arcane Magic)}, \textbf{Magica Divinitatis} \textit{(enables Clerical Magic)}, or \textbf{Magica Naturae} \textit{(enables Nature Magic)}. For users of Clerical Magic, the ability to take up this feat depends on their religion.
	\item \textbf{Advanced:} None \textit{(provided they already have this feat in Basic)}.
	\item \textbf{Expert:} None \textit{(provided they already have this feat in Advanced)}.
\end{itemize}
\textbf{Effects:}
\begin{itemize}
	\item \textbf{Untrained:} The character cannot learn any Water Magic spells.
	\item \textbf{Basic:} The character can learn Basic-level Water Magic spells, and also gains a bonus Basic-level Water Magic spell of their chosing.
	\item \textbf{Advanced:} The character can learn Advanced-level Water Magic spells, and gets a \Bonus{10\%} to the effectiveness of their Basic-tier Water Magic spells. The character also gains a bonus Advanced-level Water Magic spell of their chosing.
	\item \textbf{Expert:} The character can learn Expert-level Water Magic spells, and gets a \Bonus{30\%} to the effectiveness of their Basic-tier Water Magic spells and \Bonus{15\%} to the effectiveness of their Advanced-tier Water Magic spells. The character also gains a bonus Expert-level Water Magic spell of their chosing.
\end{itemize}\newpage
\subsection{Geomancy}
\begin{table}[!ht]
\centering
\FeatIII{Basic}{Advanced}{Expert}{Geomancy}{{\import{images/Feats/}{images/Feats/EarthMagic1.pdf_tex}}}{{\import{images/Feats/}{images/Feats/EarthMagic2.pdf_tex}}}{{\import{images/Feats/}{images/Feats/EarthMagic3.pdf_tex}}}
\end{table}
\textbf{Requirements:}
\begin{itemize}
	\item \textbf{Basic:} The character already possessing the feat \textbf{Magica Profana} \textit{(enables Arcane Magic)}, \textbf{Magica Divinitatis} \textit{(enables Clerical Magic)}, or \textbf{Magica Naturae} \textit{(enables Nature Magic)}. For users of Clerical Magic, the ability to take up this feat depends on their religion.
	\item \textbf{Advanced:} None \textit{(provided they already have this feat in Basic)}.
	\item \textbf{Expert:} None \textit{(provided they already have this feat in Advanced)}.
\end{itemize}
\textbf{Effects:}
\begin{itemize}
	\item \textbf{Untrained:} The character cannot learn any Earth Magic spells.
	\item \textbf{Basic:} The character can learn Basic-level Earth Magic spells, and also gains a bonus Basic-level Earth Magic spell of their chosing.
	\item \textbf{Advanced:} The character can learn Advanced-level Earth Magic spells, and gets a \Bonus{10\%} to the effectiveness of their Basic-tier Earth Magic spells. The character also gains a bonus Advanced-level Earth Magic spell of their chosing.
	\item \textbf{Expert:} The character can learn Expert-level Earth Magic spells, and gets a \Bonus{30\%} to the effectiveness of their Basic-tier Earth Magic spells and \Bonus{15\%} to the effectiveness of their Advanced-tier Earth Magic spells. The character also gains a bonus Expert-level Earth Magic spell of their chosing.
\end{itemize}\newpage
\subsection{Aeromancy}
\begin{table}[!ht]
\centering
\FeatIII{Basic}{Advanced}{Expert}{Aeromancy}{{\import{images/Feats/}{images/Feats/AirMagic1.pdf_tex}}}{{\import{images/Feats/}{images/Feats/AirMagic2.pdf_tex}}}{{\import{images/Feats/}{images/Feats/AirMagic3.pdf_tex}}}
\end{table}
\textbf{Requirements:}
\begin{itemize}
	\item \textbf{Basic:} The character already possessing the feat \textbf{Magica Profana} \textit{(enables Arcane Magic)}, \textbf{Magica Divinitatis} \textit{(enables Clerical Magic)}, or \textbf{Magica Naturae} \textit{(enables Nature Magic)}. For users of Clerical Magic, the ability to take up this feat depends on their religion.
	\item \textbf{Advanced:} None \textit{(provided they already have this feat in Basic)}.
	\item \textbf{Expert:} None \textit{(provided they already have this feat in Advanced)}.
\end{itemize}
\textbf{Effects:}
\begin{itemize}
	\item \textbf{Untrained:} The character cannot learn any Air Magic spells.
	\item \textbf{Basic:} The character can learn Basic-level Air Magic spells, and also gains a bonus Basic-level Air Magic spell of their chosing.
	\item \textbf{Advanced:} The character can learn Advanced-level Air Magic spells, and gets a \Bonus{10\%} to the effectiveness of their Basic-tier Air Magic spells. The character also gains a bonus Advanced-level Air Magic spell of their chosing.
	\item \textbf{Expert:} The character can learn Expert-level Air Magic spells, and gets a \Bonus{30\%} to the effectiveness of their Basic-tier Air Magic spells and \Bonus{15\%} to the effectiveness of their Advanced-tier Air Magic spells. The character also gains a bonus Expert-level Air Magic spell of their chosing.
\end{itemize}\newpage
\section{Social Feats}
\subsection{Etiquette}
\begin{table}[!ht]
\centering
\FeatIII{Basic}{Advanced}{Expert}{Etiquette}{{\import{images/Feats/}{images/Feats/Etiquette1.pdf_tex}}}{{\import{images/Feats/}{images/Feats/Etiquette2.pdf_tex}}}{{\import{images/Feats/}{images/Feats/Etiquette3.pdf_tex}}}
\end{table}
\textbf{Requirements:}
\begin{itemize}
	\item \textbf{Basic:} None, unless your background forbids it.
	\item \textbf{Advanced:} None.
	\item \textbf{Expert:} None.
\end{itemize}
\textbf{Effects:}
\begin{itemize}
	\item \textbf{Untrained:} Whether the character can read and write or not, he/she acts like an illiterate barbarian or an uncultured peasant. Heavy penalty at conversing with aristocrats - effects up to DM discression, or a \Malus{4} to Charisma when talking to upper-class people.
	\item \textbf{Basic:} The character knows enough about the mannerisms and rules - both written and unwritten - of polite society to avoid publicly embarassing himself/herself, but not enough to impress those belonging to the upper echelons - effects up to DM discression, or a \Malus{2} to Charisma when talking to upper-class people.
	\item \textbf{Advanced:} The character can fit into more cultured company and can talk to nobles without having a serious disadvantage at persuasion. No Charisma malus.
	\item \textbf{Expert:} The character feels right at home among people of the upper echelons, and can have his/her way with the tongue of those upper-class peons - so much so that, the character gains an advantage at persuading them - effects up to DM discression, or a \Bonus{2} to Charisma when talking to upper-class people.
\end{itemize}\newpage
\subsection{Street Smarts}
\begin{table}[!ht]
\centering
\FeatIII{Basic}{Advanced}{Expert}{Street Smarts}{{\import{images/Feats/}{images/Feats/StreetSmarts1.pdf_tex}}}{{\import{images/Feats/}{images/Feats/StreetSmarts2.pdf_tex}}}{{\import{images/Feats/}{images/Feats/StreetSmarts3.pdf_tex}}}
\end{table}
\textbf{Requirements:}
\begin{itemize}
	\item \textbf{Basic:} None, unless your background forbids it.
	\item \textbf{Advanced:} None.
	\item \textbf{Expert:} None.
\end{itemize}
\textbf{Effects:}
\begin{itemize}
	\item \textbf{Untrained:} The slang of the street thugs and other deliquents is downright unintelligible and incomprehensible to the character. Even if you two supposedly speak the same language, it almost feels like there's a language barrier between the two of you - effects up to DM discression, or a \Malus{4} to Charisma when talking to underground-class people.
	\item \textbf{Basic:} The character can understand street slang, but isn't fluent enough in it to show off their social skills to street deliquents - effects up to DM discression, or a \Malus{2} to Charisma when talking to underground-class people.
	\item \textbf{Advanced:} The character can fit into more caddish company and can talk to slang-slinging criminals without having a serious disadvantage at persuasion. No Charisma malus.
	\item \textbf{Expert:} The character feels right at home in the underground, and can have his/her way with the tongue of those street urchins - so much so that, the character gains an advantage at persuading them - effects up to DM discression, or a \Bonus{2} to Charisma when talking to underground-class people.
\end{itemize}\newpage
\subsection{Diaesthese}
\textbf{Requirements:}
\begin{itemize}
	\item \textbf{Basic:} None, unless your background forbids it.
	\item \textbf{Advanced:} None.
	\item \textbf{Expert:} None.
\end{itemize}
\textbf{Effects:}
\begin{itemize}
	\item \textbf{Untrained:} Unless the character knows Etiquette or Street Smarts, the character lacks people-skills altogether, and also lacks the intuition necessary to depect lies, emotional manipulation, thus lacks the ability to deflect persuasion attempts, unless the other person is trying to convince them to do something they absolutely do not want to do. \textit{(Enforcement largely up to GM - unsure how to implement in CRPG adaptation)}
	\item \textbf{Basic:} The character has some awareness, intuition and people-skills that allow them to detect lies to a degree and intuit other people's motivations \textit{(both to detect persuasion attempts from them and also to better persuade them as well)}. Basically, the character knows human nature enough to have some people-skills. \textit{(Enforcement largely up to GM - unsure how to implement in CRPG adaptation)}
	\item \textbf{Advanced:} The character can read emotions and intuit other people's motivations well enough to suspect that something's not right when other people lie to them or attempt to persuade them. They also know enough about human nature to meaningfully persuade and manipulate others. \textit{(Enforcement largely up to GM - unsure how to implement in CRPG adaptation)}
	\item \textbf{Expert:} The character can detect lies from all but the most masterful liars, and can profile people rather easily. The character has the necessary skill to easily build rapport and manipulate others, as well as deflect manipulation attempts from all but the most skilled manipulators. \textit{(Enforcement largely up to GM - unsure how to implement in CRPG adaptation)}
\end{itemize}\newpage
\subsection{Smuggling}
\begin{table}[!ht]
\centering
\FeatIII{Basic}{Advanced}{Expert}{Smuggling}{{\import{images/Feats/}{images/Feats/Smuggling1.pdf_tex}}}{{\import{images/Feats/}{images/Feats/Smuggling2.pdf_tex}}}{{\import{images/Feats/}{images/Feats/Smuggling3.pdf_tex}}}
\end{table}
\textbf{Requirements:}
\begin{itemize}
	\item \textbf{Basic:} The character must already have Basic Diaesthese.
	\item \textbf{Advanced:} None, so long as the character has this skill in Basic.
	\item \textbf{Expert:} None, so long as the character has this skill in Advanced.
\end{itemize}
\textbf{Effects:}
\begin{itemize}
	\item \textbf{Untrained:} The character is immediately caught carrying illegal goods by guards, if they have anything to hide.
	\item \textbf{Basic:} The character can fool less intelligent and less vigilant guards and convince them to let him/her pass through.
	\item \textbf{Advanced:} The character can hide illegal merchandise well enough to avoid arousing suspicion, as well as convincing the authorities to look the other ways.
	\item \textbf{Expert:} The character can practically smuggle high explosives into the royal palace without anyone so much as raising an eyebrow.
\end{itemize}\newpage
\subsection{Literacy}
The literacy skill is special in that it is not a single feat, but rather a group of feats that covers a character's fluency with a specific writing system.\newline
\textbf{Requirements:}
\begin{itemize}
	\item \textbf{Basic:} None.
	\item \textbf{Advanced:} None.
	\item \textbf{Expert:} None.
\end{itemize}
\textbf{Effects:}
\begin{itemize}
	\item \textbf{Untrained:} The character is illiterate and cannot read and write - at least not with the writing system in question.
	\item \textbf{Basic:} In case of an alphabet, abjad, abugida or syllabary, the character more or less knows which letters map to which sounds and phonemes, albeit makes constant reading and spelling errors if the script isn't particularly phonemic, with historical and etymological spelling just straight-up confusing the character. In case of a logographic script, the character may know perhaps less than a hundred characters. Either way, the character both reads and writes slowly, and might not even be able to read at all without having to sound out the written words out loud.
	\item \textbf{Advanced:} In case of an alphabet, abjad, abugida or syllabary, character has memorized enough rules and words to know about the perils of etymological and historical spelling, thus knows how to spell most commonly used words correctly, though they may still struggle with some fancy names - or in case of a logographic system, the character knows at least five hundred characters. Either way, if the character didn't already possess the ability of silent reading, now they are guaranteed to do.
	\item \textbf{Expert:} The character can both read and write perfectly fluently. If the writing system is an alphabet, then the character knows how to spell the vast majority of words in their language correctly, and isn't confused by historical or etymological spellings - if the writing system is logographic, the character knows 3000-5000 symbols.
\end{itemize}\newpage
\subsection{Foreign Language}
The language skill is special in that it is not a single feat, but rather a group of feats that covers a character's fluency in a language.\newline
\textbf{Requirements:}
\begin{itemize}
	\item \textbf{Basic:} The language in question not being the character's native tongue.
	\item \textbf{Advanced:} None.
	\item \textbf{Expert:} None.
\end{itemize}
\textbf{Effects:}
\begin{itemize}
	\item \textbf{Untrained:} The character cannot speak that language, and cannot understand anything from it, unless their own native language is closely related to \textit{(and preferably mutually intelligible with)} the language in question.
	\item \textbf{Basic:} The character has some level of fluency at the language in question, but makes a lot of errors in grammar and pronounciation alike, has a very noticeable accent and a limited vocabulary. Overall, the character knows enough to get them in and out of trouble, but not much else.
	\item \textbf{Advanced:} The character is fluent at the language. While still having an accent, and the occasional grammatical hiccup, they can both understand and speak the language confidently.
	\item \textbf{Expert:} The character is as fluent at the language as a native speaker - if not more. They not only speak it with perfect grammar and pronounciation, but can also utilize it to a degree that would make even some poets green with envy. Unless the character has intentionally made it their trademark, the character no longer has an accent.
\end{itemize}\newpage
\subsection{Classical Language}
The classical language skill is special in that it is not a single feat, but rather a group of feats that covers a character's fluency in a dead language.\newline
\textbf{Requirements:}
\begin{itemize}
	\item \textbf{Basic:} Basic Literacy at the language's script.
	\item \textbf{Advanced:} Advanced Literacy at the language's script.
	\item \textbf{Expert:} Expert Literacy at the language's script.
\end{itemize}
\textbf{Effects:}
\begin{itemize}
	\item \textbf{Untrained:} The character knows next to nothing about the classical language in question, except maybe a few loanwords their native tongue has borrowed from it.
	\item \textbf{Basic:} The character knows a lot of words, but cannot really speak the language or write in it, only sort of understand written text via slow and sluggish reading, not even understanding all the words, just figuring out what the sentence as a whole means.
	\item \textbf{Advanced:} The character knows how to read the language, and also knows how to write in it, albeit with many errors, and an unorthodox syntax that would be foreign to the now-extinct native speakers of the tongue. They can also speak the language, albeit only slowly, mostly using simple when words forced to improvise - nevertheless, it will get the point accross to anyone who shares their level of fluency at the ancient tongue.
	\item \textbf{Expert:} The character can read and write the ancient tongue fluently without any errors, and can speak it understanibly, albeit with a strong accent coloured by their native tongue - understandable, as the language in question has no native speakers left, and no one is exactly sure how to pronounce things in it anymore \textit{(for example, think of someone using Italian or Spanish pronounciation when reading Classical Latin - in other words, Eccelestial Latin in a nutshell - or Modern Greek pronounciation when reading Ancient Greek text, Classical Japanese with modern pronounciation, etc.)}.
\end{itemize}\newpage
\subsection{Music}
\begin{table}[!ht]
\centering
\FeatIII{Basic}{Advanced}{Expert}{Music}{{\import{images/Feats/}{images/Feats/Music1.pdf_tex}}}{{\import{images/Feats/}{images/Feats/Music2.pdf_tex}}}{{\import{images/Feats/}{images/Feats/Music3.pdf_tex}}}
\end{table}
\textbf{Requirements:}
\begin{itemize}
	\item \textbf{Basic:} Intact hands \textit{(for an instrument)} or the character not being a mute \textit{(for vocals)}
	\item \textbf{Advanced:} None, if Basic Music is already possed.
	\item \textbf{Expert:} None, if Advanced Music is already possed.
\end{itemize}
\textbf{Effects:}
\begin{itemize}
	\item \textbf{Untrained:} The character knows nothing about making music - they may not be tonedeaf, they may appreciate good music, but they are utterly clueless when it comes to making music. They can't play any instruments, sing out of tune, etc.
	\item \textbf{Basic:} The character can make acceptable quality music, and can play back existing songs with only a few mistakes. If formally trained in music, they know a lot about musical theory and can read sheet music - if not, then they can still intuit musical theory and memorize songs. Learning a new instrument is a hard thing to do.
	\item \textbf{Advanced:} The character's musical skills are what can be expected from a typical bard - at least, one that's financially secure and isn't in the risk of running out of business.
	\item \textbf{Expert:} The character can compose fine songs, and can be considered a virtuoso on their instrument of choice - they also learn new instruments relatively easily. If the character has opted for vocals instead or, or in addition to instruments, they also know quite a few vocal techniques, being able to squeeze out as much as possible from their vocal chords.
\end{itemize}\newpage
\subsection{Amorous Culture}
\textbf{Requirements:}
\begin{itemize}
	\item \textbf{Basic:} The character can't be a virgin. \textit{(Backstory requirement)}
	\item \textbf{Advanced:} None, if Basic Amorous Culture is already possed.
	\item \textbf{Expert:} None, if Advanced Amorous Culture is already possed.
\end{itemize}
\textbf{Effects:}
\begin{itemize}
	\item \textbf{Untrained:} When making love, the character is driven purely by their instincts - whether that is a good thing or a bad thing is subjective. One's taste might coincidentally correspond to what the character has instinctively produced, but others may want something more refined: whether that means softer and more considerate, or rougher and more aggressive, varies from partner to partner. Additionally, the character likely doesn't know much about protection, STDs, etc.
	\item \textbf{Basic:} The character knows how to intuit their partner's desires, and can more or less attempt to fufill them - to what degree of success, remains questioanble.
	\item \textbf{Advanced:} The character knows the art of lovemaking, and knows how to please any partner whose desire is a technique of sorts, rather than a passive attribute of their partner. The character leaves most partners satisfied.
	\item \textbf{Expert:} The character has mastered the art of lovemaking to the point of effectively making their partners crazy from ecstasy during the act, constantly leaving them yearning for even more.
\end{itemize}\newpage
\subsection{Logic}
\textbf{Requirements:}
\begin{itemize}
	\item \textbf{Basic:} An intelligence of 10.
	\item \textbf{Advanced:} An intelligence of 12.
	\item \textbf{Expert:} An intelligence of 14.
\end{itemize}
\textbf{Effects:}
\begin{itemize}
	\item \textbf{Untrained:} The character hasn't been formally schooled in Logic - which doesn't necessarily mean that they are devoid of logic or incapable of logical reasoning, potentially, it's quite the contrary: being able to put two and two together and discover the relationships between things is just being intelligent. \textit{"Logic"} in this context refers to more abstract fields of study and philosophy, something formally studied, beyond regular intuition.
	\item \textbf{Basic:} The character has received some formal schooling in Logic. Besides being skilled at logical reasoning - whether deductive, inductive or abductive, up to their preference - they also have knowledge about syllogistic logic, as well as Boolean logic.
	\item \textbf{Advanced:} The character possesses a comprehensive understanding of classical logic and philosophy, and thus can use this knowledge in philosophical debates, as well as when answering complex questions.
	\item \textbf{Expert:} The character is a philosopher in the making.
\end{itemize}\newpage
\subsection{Profession}
The Profession skill is special in that it is not a single feat, but rather a group of feats that covers a character's skill at a mundane profession that are typically pursued by normal people who aren't adventurers - some of these might still come useful to adventurers on special occasions though, like when an obelisk asks an adventurer a question only an Expert Shoemaker could answer. Some professions may also allow the character to craft certain items.\newline
\textbf{Requirements:}
\begin{itemize}
	\item \textbf{Basic:} Nothing.
	\item \textbf{Advanced:} Nothing.
	\item \textbf{Expert:} Nothing.
\end{itemize}
\textbf{Effects:}
\begin{itemize}
	\item \textbf{Untrained:} The character has no experience in or any real knowledge about that profession. Even if someone attempts to teach it to them, they will inevitably forget it in a day.
	\item \textbf{Basic:} The character knows the basics of the trade in question, and could make a fine assisstant to an actual master, but nothing more.
	\item \textbf{Advanced:} The character knows far more than what can be expected from an assistant, but not enough to make them a master of their trade. If the character was to retire from their life of adventure, they could make a decent living out of pursuing that profession, but would be unable to fufill the desires of more refined customers.
	\item \textbf{Expert:} The character is an expert of the trade in question, and if they were to encoutner an oracle or genie who makes a riddle only solvable by someone with considerable knowledge in that field, they could solve it. Their skill effectively gives them a safety net, a Plan B if the whole adventuring business turns out to be less lucrative than initially expected, as they know the profession enough to make a good living out of it.
\end{itemize}\newpage
%\section{Mundane Professions}
%\subsection{Smithing}
%\subsection{Weaponsmithing}
%\subsection{Armoursmithing}
%\subsection{Wheelcrafting}
%\subsection{Woodworking}
%\subsection{Carpentry}
%\subsection{Shoemaking}
%\subsection{Needlework}
%\subsection{Herbalism}
%\subsection{Farming}
%\subsection{Hunting}
%\subsection{Cooking}
\chapter{Backgrounds}
\section{Silver Spoon}
You were born into and raised as a member of a family belonging to the upper echelons of society. This means you have connections, you know people of importance, and you are known by people of importance. All well and good, but your pampered and sheltered childhood has spoiled you, reducing your ability to learn.\newline
\textbf{Bonuses:}
\begin{itemize}
	\item \Bonus{2} to Charisma
	\item You start with the Advanced Etiquette feat
	\item Higher-class people will have a higher opinion of you than they otherwise would. \textit{(enforcement up to the GM)}
	\item High amount of starting money.
\end{itemize}
\textbf{Maluses:}
\begin{itemize}
	\item You can only choose 3 feats at the beginning, instead of the usual 6.
	\item \Malus{1} to Willpower
	\item Lower-class people will have a lower opinion of you than they otherwise would. \textit{(enforcement up to the GM)}
	\item Slower learning and experience \textit{(enforcement up to the GM - in a video game, this should be a malus to experience gain)}
\end{itemize}
\section{The Negotiator}
Ever since an early stage of your life, you have discovered a unique talent of yours: you always had a way with words. Whether you were speaking the slang of the lower-class cad or the flowery poetry of the upper-class peon, you always had a way with words, which saved your butt several times in your life. Of course, being able to talk your way out of every uncomfortable situation had the negative effect of stunting the your growth in other areas - after all, why would you bother learning how to fight, if you can just talk the monster to death?\newline
\textbf{Bonuses:}
\begin{itemize}
	\item \Bonus{2} to Charisma
	\item You start with the Advanced Diaesthese feat
	\item You start with the Basic Etiquette feat
	\item You start with the Basic Street Smarts feat
\end{itemize}
\textbf{Maluses:}
\begin{itemize}
	\item You can only choose 3 feats at the beginning, instead of the usual 6.
	\item \Malus{1} to Strength
	\item \Malus{1} to Endurance
\end{itemize}
\section{Pirate}
Avast, ye scurvy dog! Never feeling quite at home among all those landlubbers, you made the seven seas your home, constantly on the hunt for poorly defended ships full of loot and plunder! You know the open waters like the palm of your own hand, and it is more than obvious that an encounter with you won't be plasant for any merchant ships. Yarrrr!\newline
\textbf{Bonuses:}
\begin{itemize}
	\item Basic Seamanship feat for free.
	\item \Bonus{1} to Strength
	\item \Bonus{1} to Dexterity
\end{itemize}
\textbf{Maluses:}
\begin{itemize}
	\item \Malus{2} to Intelligence
	\item You are a wanted criminal. You can't socialize in public without a disguise in a place where law enforcement exists. \textit{(enforcement up to the GM - shouldn't be active in Shanty Towns, only in more respectable neighbours, where the laws are actually enforced)}
\end{itemize}
\section{Amnesiac Diviner of the Sea}
Some say that every great man's story begins at their birth. Others prefer starting the story at an important event that set them on the path they would go on. If we were to go with the latter school of thought, we'd say that your story truly begins during a sea battle. You were essentially a sailor with Cassandra's curse - blessed with the ability to see the future, cursed with the fate of not being believed by anyone. \textit{"It's a trap!"} you said, but the captain didn't listen to you. Your ship was ambushed, your ship was captured, your crew was massacred, and you were thrown into the sea to feed the sharks - yet, somhehow, you lived, washed ashore a nameless island. Whether it was because you hit your head extra hard some time along the line, or because of traumatic experience, you now have amnesia, barely remember anything about the past before the battle. How tragic to be able to tell others' future, but not your own past, isn't it?\newline
\textbf{Bonuses:}
\begin{itemize}
	\item Basic Seamanship for free
	\item Your character has premotions of sorts, and can tell if something is wrong in an \textit{"I have a bad feeling about this"} way. The GM has to take this into account, albeit not to the point of metagaming.
\end{itemize}
\textbf{Maluses:}
\begin{itemize}
	\item You can only choose 5 feats at the beginning, instead of the usual 6.
\end{itemize}
\section{Disgraced Aristocrat}
Born into a noble family, good with weapons and destined to be noble protector of the realm, you were once the pride of your father - emphasis on was. You did something that upset your family so terribly, that they disowned you, depriving you of your inheritance and much of your wealth, effectively sending you off to walk the earth. Now you travel the lands as a disgruntled mercenary looking for lucrative jobs that might help you become as rich as you once used to be.\newline
\textbf{Bonuses:}
\begin{itemize}
	\item Dummy
\end{itemize}
\textbf{Maluses:}
\begin{itemize}
	\item dummy
\end{itemize}
\section{Discredited Academic}
You tried to prove some study that other academicians didn't like, or tried to perform unorthodox experiments that made you get expelled from whatever institute you were working or studying at. Now you walk the earth in hopes that one day, you will find someone who will appreciate your discoveries and theories, either by finding a sponsor to finance your continued studies, or by earning enough money to continue those studies privately.\newline
\textbf{Bonuses:}
\begin{itemize}
	\item \Bonus{2} to Intelligence
\end{itemize}
\textbf{Maluses:}
\begin{itemize}
	\item \Malus{1} to Charisma
	\item Particularly religious characters will have a lower opinion of this character. Up to the GM to enforce.
\end{itemize}
\section{Street Urchin}
\textbf{Bonuses:}
\begin{itemize}
	\item Dummy
\end{itemize}
\textbf{Maluses:}
\begin{itemize}
	\item Dummy
\end{itemize}
\section{Runaway Slave}
\textbf{Bonuses:}
\begin{itemize}
	\item Dummy
\end{itemize}
\textbf{Maluses:}
\begin{itemize}
	\item Dummy
\end{itemize}
\chapter{Weapons n' Armour}
\chapter{Magicks}
\end{document}
