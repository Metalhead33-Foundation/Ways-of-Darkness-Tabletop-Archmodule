\documentclass[openany,10pt,a4paper]{book}
\usepackage{graphicx}
\usepackage[utf8x]{inputenc}
\usepackage{multirow}
\usepackage{blindtext}
\usepackage{graphicx}
\usepackage{numprint}
\usepackage{listings}
\usepackage{xcolor}
\usepackage{tabularx,booktabs}
\usepackage{amsmath}
\usepackage{mathtools}
\usepackage{float}
\usepackage{caption}
\usepackage{enumitem}
\usepackage{newtxtext,newtxmath}
\usepackage{makecell}
\usepackage{import}
\usepackage[square,sort,comma,numbers]{natbib}
\usepackage[a4paper, margin=1.5cm]{geometry}
%\setcounter{tocdepth}{2}
\newcolumntype{Y}{>{\centering\arraybackslash}X}
\graphicspath{ {./images/} }
\author{Metalhead33}
\title{Ways of Darkness\\
   \normalsize A module for the World of Artograch RPG system}
\definecolor{mGreen}{rgb}{0,0.6,0}
\definecolor{mGray}{rgb}{0.5,0.5,0.5}
\definecolor{mPurple}{rgb}{0.3,0,0.3}
\definecolor{backgroundColour}{rgb}{0.95,0.95,0.92}
\usepackage{hyperref}
\hypersetup{
    colorlinks=true,
    linkcolor=blue,
    filecolor=magenta,      
    urlcolor=cyan,
}

\begin{document}
\newcommand{\BipedalTwoArms}[3]{Being bipedal {#1}, {#3} have two hands and two legs, meaning that they can wield only two one-handed weapons \textit{(or a one-handed weapon and a shield)} or a single two-handed weapon at the same time. Just like with other bipedal {#2} races, reaching zero hitpoints on the arms or legs renders those limbs unusable \textit{(and disables the corresponding item slots)}, while reaching zero hitpoints at the torso or head means death - or unconsciousness, if house rules rule out death.}
\newcommand{\MammalRace}[1]{\BipedalTwoArms{mammals without tails}{mammalian}{{#1}} As mammals, {#1} cannot breathe under water without magic or specifial \textit{(read: modern, therefore nonexistent in this setting)} equipment, with prolonged presence underweater leading to drowning.}
\newcommand{\LizardRace}[1]{\BipedalTwoArms{reptoids with tails}{reptilian}{{#1}} Their tails are not a vital body part - their loss may be very painful \textit{(and cause potentially lethal bleeding, if the wound is not treated fast enough)}, the body part itself isn't vital, making its loss non-lethal. As we're talking about reptiles rather than mammmals, this species breeds by eggs rather than giving live birth; has a critical weakness towards extreme temperatures and cannot be turned into a secondary race specific to mammals. Reptiles are different from amphibians - they are \textbf{not} able to breathe underwater, and thus drown just like mammals. They cannot be infected with Vampirism or Theriantropy, because those are mammal-specific diseases.}
\newcommand{\UndeadRace}[1]{Being undead creatures, {#1} have no need for food or water, thus the vitals for proteon, carbohydrates, fat, calories, alcohol and water are inactive for them.}
\newcommand{\Bonus}[1]{\textcolor{green}{\textbf{+{#1} bonus}}}
\newcommand{\BonusS}[1]{\textcolor{green}{\textbf{+{#1}}}}
\newcommand{\Malus}[1]{\textcolor{red}{\textbf{-{#1} malus}}}
\newcommand{\MalusS}[1]{\textcolor{red}{\textbf{-{#1}}}}
\newcommand{\MalusP}[1]{\textcolor{red}{\textbf{+{#1} malus}}}
\newcommand{\MalusPS}[1]{\textcolor{red}{\textbf{+{#1}}}}
\newcommand{\DimorphismBB}[5]{Male {#1} get a \Bonus{{#2}} to {#3}, while Female {#1} get a \Bonus{{#4}} to {#5}.}
\newcommand{\SoCalled}[1]{\textit{``{#1}''}}
\newcommand{\Parentheses}[1]{\textit{({#1})}}
\newcommand{\FeatI}[2]{\begin{tabular}{|c|}
\hline
{#1} \\ \hline
{\resizebox{.2\linewidth}{!}{{#2}}} \\ \hline
\end{tabular}}
\newcommand{\FeatII}[5]{\begin{tabularx}{0.75\textwidth}{c @{\extracolsep{\fill}} c}
\hline
\multicolumn{2}{|c|}{{#3}} \\ \hline
\multicolumn{1}{|c}{\resizebox{.2\linewidth}{!}{{#4}}} & \multicolumn{1}{c}{\resizebox{.2\linewidth}{!}{{#5}}} \\ \hline
\multicolumn{1}{|c}{{#1}}        & \multicolumn{1}{c}{{#2}}  \\ \hline
\end{tabularx}}
\newcommand{\FeatIII}[7]{\begin{tabularx}{0.75\textwidth}{c @{\extracolsep{\fill}} cc}
\hline
\multicolumn{3}{|c|}{{#4}} \\ \hline
\multicolumn{1}{|c}{\resizebox{.2\linewidth}{!}{{#5}}} & \multicolumn{1}{c}{\resizebox{.2\linewidth}{!}{{#6}}} & \multicolumn{1}{c|}{\resizebox{.2\linewidth}{!}{{#7}}} \\ \hline
\multicolumn{1}{|c}{{#1}}        & \multicolumn{1}{c}{{#2}}       & \multicolumn{1}{c|}{{#3}}       \\ \hline
\end{tabularx}}

\maketitle
\tableofcontents
\chapter*{Preface}
You are currently reading the documentation of the \textbf{Ways of Darkness module for} the \textbf{World of Artograch RPG system}. The author of this document presents you all the information within this document with the assumption that you have read the \textbf{World of Artograch Ruleset}, and thus are familiar with the rules it presented. Contents of the aforementioned document are expected to be referenced in this document.
\addcontentsline{toc}{chapter}{Preface}
\chapter{Introduction to the Occident}
\includegraphics[width=\textwidth,height=\textheight,keepaspectratio]{Occident_Borders}
\newpage
The \textbf{Occident} is the part of \textbf{Artograch} with perhaps the most dynamic history. While even the Occident shows a similar tendency as the Orient to have large countries dominated by a single race, it is slightly more nuanced in the Occident, which has had a history of territories changing hands. While the Orient has always been characterized by the rule of centralized and homogenous kingdoms, and highly bureaucratic and usually equally homogenous, often quasi-despotic empires, the Occident has always been a place where central authorities held only limited amount of power, with provinces enjoying high autonomy.\newline
The various races of the Occident are divided into two major groups: the Elven races \textit{(Humans, High Elves, Wood Elves, Dark Elves, Orcs)} being descended from Oriental invaders; and the indigenous races \textit{(Goblins, Ogres, Lizardmen, Halflings, Gnomes and Dwarves)} whose population was decimated by the arrival of the earlier. Out of the two, the earlier are clearly the dominant force on the continent, with three out of the four dominant powers in the Occident being dominated by an Elven race - Etrand by Humans, Froturn by High Elves, Dragoc by Wood Elves.\newline
Currently, as of 831 AEKE \textit{(831 years after the establishment of the Kingdom of Etrand)}, the Occident is at a tipping point: the three aforementioned great powers are at a three-way cold war with each other, with tensions being at an all-time high, while the previously unmentioned fourth, Gabyr, is sitting in the shadows watching it all with popcorn in their hands, while slowly expanding their trade empire in the Orient. Unseen for roughly five and a half centuries, Demons have been sighted lurking around, no doubt planting their own secret agents among the authorities of the aforementioned states. It is said that even the Orcs of Brutang are getting quite unruly, while the Empire of Neressa continues its long-practiced isolationist policies and the Principality of Artaburro tries to wiggle between the Kingdoms of Froturn, Dragoc and Etrand.
\chapter{Playable Primary Races}
\input{races/Humans.tex}
\input{races/HighElves.tex}
\input{races/WoodElves.tex}
\section{Dark Elves}
More \textit{(up-to-date)} information at: \url{https://ways-of-darkness.sonck.nl/Dark_Elves}\newline
\includegraphics{Dark_Elves}\newline
\textbf{Dark Elves} are a race without a unified country, living in underground clans in dungeon-cities below the the ground, though also being present in several surface states - namely the Kingdoms of Etrand and Froturn - in high numbers as diaspora. They have rather long lifespans, theoretically ten times that of a human \textit{(and they're also typically taller than humans, not to mention their sharp ears and blue-ish grey skin)}, ageing at one tenth a human's rate after reaching the age of eighteen. They descendants of Wood Elves and High Elves who were banished after experimenting with the dark arts, creating their own exodus, where the environment - and the new religion - has altered their appearence. The average adult Dark Elf is about 180 centimetres \textit{(5 feet and 11 inches)} tall, regardless of gender \textit{(albeit men to tend to be slightly taller on average)}.\newline
\begin{tabular}{|c|c|c|}
\hline
 & \textbf{Min} & \textbf{Max} \\ \hline
\textbf{Strength} & 8 & 18 \\ \hline
\textbf{Endurance} & 7 & 17 \\ \hline
\textbf{Dexterity} & 10 & 20 \\ \hline
\textbf{Intelligence} & 8 & 18 \\ \hline
\textbf{Willpower} & 8 & 18 \\ \hline
\textbf{Charisma} & 8 & 18 \\ \hline
\end{tabular}\newline
\DimorphismBB{Dark Elves}{1}{Strength}{1}{Charisma} \MammalRace{Dark Elves}\newpage

\input{races/HalfElves.tex}
\input{races/HalfOrcs.tex}
\input{races/Orcs.tex}
\input{races/Ogres.tex}
\input{races/Goblins.tex}
\section{Halflings}
More \textit{(up-to-date)} information at: \url{https://ways-of-darkness.sonck.nl/Halflings}\newline
\includegraphics{Hobbits}\newline
\textbf{Halflings} are a race indigenous to Artograch, with an average lifespan of 150 years, and average height of 130 centimetres or 4 feet and 3.1 inches. Due to the fact that they are rather short-statued humanoid mammals and speak a language related to Dwarven, they are sometimes grouped in with the Dwarves and Gnomes as the \SoCalled{Norlokian races}, even though they are incapable of interbreeding with each other, and thus may not even be biologically related, despite the superficial resemblance.\newline
Halflings are a race of smart, inventive survivors and adventurers, known for their curiousity and boldness. They are easily seduced by riches, but they prefer spending money, for they are a race definitely not made famous by their frugality. Their skin is ruddy, their hair is typically either black, brown or red, and either straight or curly \textit{(never wavy)}, and their eyes are usually brown or black. Halfling men very often grow out their sideburns, but seldom do they also grow a beard or mustache. They prefer dressing for comfort and practicality, and in spite of their love of riches, they typically shun jewellery. They love to enjoy the little things in life, such as smoking their pipes.\newline
\begin{tabular}{|c|c|c|}
\hline
 & \textbf{Min} & \textbf{Max} \\ \hline
\textbf{Strength} 7 8 & 17 \\ \hline
\textbf{Endurance} & 8 & 18 \\ \hline
\textbf{Dexterity} & 9 & 19 \\ \hline
\textbf{Intelligence} & 8 & 18 \\ \hline
\textbf{Willpower} & 8 & 18 \\ \hline
\textbf{Charisma} & 8 & 18 \\ \hline
\end{tabular}\newline
\DimorphismBB{Halflings}{1}{Strength and Endurance}{1}{Dexterity and Charisma} \MammalRace{Halflings}\newpage

\input{races/Gnomes.tex}
\section{Dwarves}
More \textit{(up-to-date)} information at: \url{https://ways-of-darkness.sonck.nl/Dwarves}\newline
\includegraphics{Dwarves}\newline
\textbf{Dwarves} are a race indigenous to Artograch, with an average lifespan of 200 years, and average height of 120 centimetres or 3 feet and 11.2 inches. Due to the fact that they are rather short-statued humanoid mammals and speak a language related to Halfling, they are sometimes grouped in with the Halflings and Gnomes as the \SoCalled{Norlokian races}, even though they are incapable of interbreeding with each other, and thus may not even be biologically related, despite the superficial resemblance.\newline
The Dwarves are one of the most easily recognizable races of Artograch. Short but muscular statue, crude humour, mild xenophobia, fondness for beer and women, you will almost instantly recognize the dwarf. Having been a closed people for millennia, dwarves are not very trusting with foreigners \textit{(other than the gnomes whom they don’t really consider foreigners at all)}, but they are generous with the ones who earn their trust. They are also known for their talent at blacksmithing, mining and melee weapons. Dwarven steel is universally considered the best material for making both weapons and armour, and no one can deny the beauty of dwarven-made ornaments that decorate armour and weapons that would be already considered nicely-made even without them. Dwarves’ skin colour can vary from yellowish brown to pale white. A dwarf's hair colour can be blond, red, brown or black. In Dwarven society, if a male does not have a beard, he isn't considered male. Traditionally, the dwarves lived in clans that constantly bickered amongst each other, but historical events six centuries ago changed all of that.\newline
\begin{tabular}{|c|c|c|}
\hline
 & \textbf{Min} & \textbf{Max} \\ \hline
\textbf{Strength} & 8 & 18 \\ \hline
\textbf{Endurance} & 10 & 20 \\ \hline
\textbf{Dexterity} & 8 & 18 \\ \hline
\textbf{Intelligence} & 8 & 18 \\ \hline
\textbf{Willpower} & 8 & 18 \\ \hline
\textbf{Charisma} & 7 & 17 \\ \hline
\end{tabular}\newline
\DimorphismBB{Dwarves}{1}{Strength}{1}{Charisma} \MammalRace{Dwarves}\newpage

\input{races/Lizardmen.tex}
\chapter{Non-Playable Primary Races}
\textbf{Non-PLayable races} are races that, while not explicitly forbidden, are not recommended for usage as player characters for various reasons. They will not be playable in any of the video game adaptations either, and they are not recommended for tabletop usage either, other than as NPCs controlled by the game master. The reasons for this vary a lot, but boil down to these so-called \SoCalled{non-playable races} being...
\begin{itemize}
	\item ...unballanced, thus giving players who control them an unfair \Parentheses{dis}advantage.
	\item ...not as fleshed out as the so-called so-called \SoCalled{playable races}
	\item ...not fitting into the traditional formula of characters having attributes, classes and equipment. As a rule of thumb, non-bipedal races are automatically non-playable in the Occident.
\end{itemize}
As previously stated, while playing as characters from these races is not explicitly forbidden per se, it is genuinely not recommended for game masters to let players create characters from these races. These races will also be non-playable in any future video game adaptations.
\input{races/Nereids.tex}
\section{Angels}
More \textit{(up-to-date)} information at: \url{https://ways-of-darkness.sonck.nl/Angels}\newline

\input{races/Dragons.tex}
\input{races/Griffins.tex}
\input{races/WingedCobras.tex}
\chapter{Secondary Races}
\textbf{Secondary races} are races that people aren't usually born into, but instead are turned into, usually via some sort of magic. Most secondary races are undead, and their condition is caused by a symbiotic relationship between their own animated corpse and a tiny creature that is responsible for the animation: the only cure to such a condition is a final death.
\input{races/Liches.tex}
\input{races/Vampires.tex}
\section{Theriantropes}
More \textit{(up-to-date)} information at: \url{https://ways-of-darkness.sonck.nl/Theriantropy}\newline
\textbf{Theriantropes} are sentient undead creatures who - unlike Liches - largely retain their original mortal looks, albeit they take on certain animal motiffs, such as the smell of a wild animal and character tics associated with said animal. The condition of theriantropy shares a lot with Vampirism, such as the fact that it is caused by a parasite that kills its host and reanimates the host's corpse as an undead creature, allowing them retain free will, but still influencing them, giving them powers in exchange for blood. On every full moon, young theriantropes go through an involuntary transformation that makes them lose their free will, turning them into berserkers driven by lust for blood - they also go through a physical transformation, from their original form into a bipedal animal \Parentheses{wolf, bear, boar, rat, lion, tiger, etc.}. As theriantropes grow older and stronger, they gradually learn to control their condition, transforming at will, and retaining their free will even when transformed.\newline
Just like how vampirism comes in various forms, theriantropy does to come in many forms, with werewolves \Parentheses{also known as lycantropes} being far by the most common type of theriantrope. Other kinds of theriantropes include werebears, wereboars, wererats, werelions and weretigers. The differences between these were-animals is largely cosmetic, with the common denominator being the fact that they are undead beings stronger than regular mortals, go through involuntary transformations when young, and obstain the smell and character tics of the associated animal.\newline
Just like vampires, theriantropes are a special kind of undead that - through their symbiosis with the parasite - can still get to mingle with mortals, appreciate the taste of mortal food and beverages, reproduce sexually and grow into adulthood when turned prematurely when fed with sufficient amount of blood at regular intervals. However, lack of blood causes involuntary transformations and loss of control first, then the decaying of their body.\newline
\begin{tabular}{|c|c|c|}
\hline
 & \textbf{Bonus/Malus} \\ \hline
\textbf{Strength} & \BonusS{4} \\ \hline
\textbf{Endurance} & \BonusS{4}  \\ \hline
\textbf{Dexterity} & \textit{unchanged}  \\ \hline
\textbf{Intelligence} & \textit{unchanged} \\ \hline
\textbf{Willpower} & \textit{unchanged} \\ \hline
\textbf{Charisma} & \textit{unchanged} \\ \hline
\end{tabular}\newline
\UndeadRace{Theriantropes} Instead, they activate a need for Blood, their own unique vital shared with Vampires.\newpage


\chapter{Feats}
\section{Combat Ability Feats}
\input{feats/CombatFeats/MartialArts.tex}
\subsection{Flurry}
\begin{table}[!ht]
\centering
\FeatIII{Basic}{Advanced}{Expert}{Flurry}{{\import{images/Feats/}{images/Feats/Flurry1.pdf_tex}}}{{\import{images/Feats/}{images/Feats/Flurry2.pdf_tex}}}{{\import{images/Feats/}{images/Feats/Flurry3.pdf_tex}}}
\end{table}
\textbf{Requirements:}
\begin{itemize}
	\item \textbf{Basic:} None, unless your background forbids it.
	\item \textbf{Advanced:} None.
	\item \textbf{Expert:} None.
\end{itemize}
\textbf{Effects:}
\begin{itemize}
	\item \textbf{Untrained:} The character cannot use the combat ability \textit{"Quick Attack"}.
	\item \textbf{Basic:} The character has unlocked the combat ability \textit{"Quick Attack"}. As stated in the basic ruleset, the Quick Attack move allows the character to attack twice in melee during a turn, but at the cost of a \MalusPS{30\%} increase to chance of miss and a \Malus{30\%} to all damage done in melee in case of landing a hit.
	\item \textbf{Advanced:} The aforementioned maluses are reduced to \MalusPS{15\%} and \MalusS{15\%} respectively.
	\item \textbf{Expert:} The aforementioned maluses are gone, nullified.
\end{itemize}\newpage

\input{feats/CombatFeats/Overswing.tex}
\input{feats/CombatFeats/Chevauchee.tex}
\subsection{Mounted Archery}
\begin{table}[!ht]
\centering
\FeatIII{Basic}{Advanced}{Expert}{Mounted Archery}{{\import{images/Feats/}{images/Feats/HorseArchery1.pdf_tex}}}{{\import{images/Feats/}{images/Feats/HorseArchery2.pdf_tex}}}{{\import{images/Feats/}{images/Feats/HorseArchery3.pdf_tex}}}
\end{table}
\textbf{Requirements:}
\begin{itemize}
	\item \textbf{Basic:} Basic Riding.
	\item \textbf{Advanced:} None.
	\item \textbf{Expert:} None.
\end{itemize}
\textbf{Effects:}
\begin{itemize}
	\item \textbf{Untrained:} The character suffers a \Malus{4} to Dexterity while trying to use Ranged Weapons on horseback, at least when moving.
	\item \textbf{Basic:} The character suffers a \Malus{2} to Dexterity while trying to use Ranged Weapons on horseback, at least when moving.
	\item \textbf{Advanced:} The character suffers a \Malus{1} to Dexterity while trying to use Ranged Weapons on horseback, at least when moving.
	\item \textbf{Expert:} The character suffers no malus to Dexterity while trying to use Ranged Weapons on horseback, at all, shooting just as well as he/she would if stationary.
\end{itemize}\newpage


\section{Combat Equipment Proficiency Feats}
\input{feats/WeaponFeats/ArmourConditioning.tex}
\input{feats/WeaponFeats/Bludgeoners.tex}
\subsection{Swords}
\begin{table}[!ht]
\centering
\FeatIII{Basic}{Advanced}{Expert}{Swords}{{\import{images/Feats/}{images/Feats/Blade1.pdf_tex}}}{{\import{images/Feats/}{images/Feats/Blade2.pdf_tex}}}{{\import{images/Feats/}{images/Feats/Blade3.pdf_tex}}}
\end{table}
\textbf{Requirements:}
\begin{itemize}
	\item \textbf{Basic:} None, unless your background forbids it.
	\item \textbf{Advanced:} None.
	\item \textbf{Expert:} None.
\end{itemize}
\textbf{Effects:}
\begin{itemize}
	\item \textbf{Untrained:} The character always deals minimal damage with swords, no need to roll the dice. They also get a \Malus{2} to Dexterity when it's being counted when using swords.
	\item \textbf{Basic:} The character suffers no bonuses or maluses when using swords.
	\item \textbf{Advanced:} The character reiceves a \Bonus{25\%} to all damage done by swords, and a \Bonus{25\%} to chance to hit or cause critical damage.
	\item \textbf{Expert:} The character reiceves a \Bonus{50\%} to all damage done by swords, and a \Bonus{50\%} to chance to hit or cause critical damage.
\end{itemize}\newpage

\subsection{Polearms}
\begin{table}[!ht]
\centering
\FeatIII{Basic}{Advanced}{Expert}{Polearms}{{\import{images/Feats/}{images/Feats/Stick1.pdf_tex}}}{{\import{images/Feats/}{images/Feats/Stick2.pdf_tex}}}{{\import{images/Feats/}{images/Feats/Stick3.pdf_tex}}}
\end{table}
\textbf{Requirements:}
\begin{itemize}
	\item \textbf{Basic:} None, unless your background forbids it.
	\item \textbf{Advanced:} None.
	\item \textbf{Expert:} None.
\end{itemize}
\textbf{Effects:}
\begin{itemize}
	\item \textbf{Untrained:} The character always deals minimal damage with so-called \SoCalled{polearms} \Parentheses{quarterstaffs, spears, pikes, halberds, pollaxes, glaives, voulges and bills}, no need to roll the dice. They also get a \Malus{2} to Dexterity when it's being counted when using polearms.
	\item \textbf{Basic:} The character suffers no bonuses or maluses when using so-called \SoCalled{polearms} \Parentheses{quarterstaffs, spears, pikes, halberds, pollaxes, glaives, voulges and bills}.
	\item \textbf{Advanced:} The character reiceves a \Bonus{25\%} to all damage done by \SoCalled{polearms} \Parentheses{quarterstaffs, spears, pikes, halberds, pollaxes, glaives, voulges and bills}, and a \Bonus{25\%} to chance to hit or cause critical damage.
	\item \textbf{Expert:} The character reiceves a \Bonus{50\%} to all damage done by \SoCalled{polearms} \Parentheses{quarterstaffs, spears, pikes, halberds, pollaxes, glaives, voulges and bills}, and a \Bonus{50\%} to chance to hit or cause critical damage.
\end{itemize}\newpage

\input{feats/WeaponFeats/Bows.tex}
subsection{Crossbows}
\begin{table}[!ht]
\centering
\FeatIII{Basic}{Advanced}{Expert}{Crossbows}{{\import{images/Feats/}{images/Feats/Crossbow1.pdf_tex}}}{{\import{images/Feats/}{images/Feats/Crossbow2.pdf_tex}}}{{\import{images/Feats/}{images/Feats/Crossbow3.pdf_tex}}}
\end{table}
\textbf{Requirements:}
\begin{itemize}
	\item \textbf{Basic:} None, unless your background forbids it.
	\item \textbf{Advanced:} None.
	\item \textbf{Expert:} None.
\end{itemize}
\textbf{Effects:}
\begin{itemize}
	\item \textbf{Untrained:} The character needs to skip a turn before using a crossbow to reload every single time they want to shoot. They also get a \Malus{4} to Dexterity when it's being counted when using crossbows. In video game adaptations, this should halve the effective range of all bows.
	\item \textbf{Basic:} The character suffers no bonuses or maluses when using crossbows, and can reload fast enough to not to skip a turn doing it.
	\item \textbf{Advanced:} The character reiceves a \Bonus{25\%} to all damage done by crossbows, and a \Bonus{25\%} to chance to hit or cause critical damage. In video game adaptations, this should also increase range.
	\item \textbf{Expert:} The character reiceves a \Bonus{50\%} to all damage done by crossbows, and a \Bonus{50\%} to chance to hit or cause critical damage. In video game adaptations, this should also increase range.
\end{itemize}\newpage

\input{feats/WeaponFeats/Guns.tex}
\input{feats/WeaponFeats/ExoticWeapons.tex}

\section{Logistic Feats}
\input{feats/LogisticFeats/Sneaking.tex}
\input{feats/LogisticFeats/Swimming.tex}
\input{feats/LogisticFeats/Riding.tex}
\input{feats/LogisticFeats/Horsemanship.tex}
\input{feats/LogisticFeats/Cameleering.tex}
\input{feats/LogisticFeats/Raptoring.tex}
\input{feats/LogisticFeats/Seamanship.tex}
\input{feats/LogisticFeats/Arithmetics.tex}
\input{feats/LogisticFeats/Forgery.tex}
\input{feats/LogisticFeats/Tracking.tex}
\input{feats/LogisticFeats/Naturalism.tex}

\section{Magic Feats}
\subsection{Alchemy}
\begin{table}[!ht]
\centering
\FeatIII{Basic}{Advanced}{Expert}{Alchemy}{{\import{images/Feats/}{images/Feats/Alchemy1.pdf_tex}}}{{\import{images/Feats/}{images/Feats/Alchemy2.pdf_tex}}}{{\import{images/Feats/}{images/Feats/Alchemy3.pdf_tex}}}
\end{table}\newpage

\input{feats/MagicFeats/Enchantment.tex}
\input{feats/MagicFeats/ArcaneMagic.tex}
\subsection{Magica Divinitatis}
\begin{table}[!ht]
\centering
\FeatI{Magica Divinitatis}{{\import{images/Feats/}{images/Feats/Clericalism.pdf_tex}}}
\end{table}
\textbf{Requirements:}
\begin{itemize}
	\item \textbf{Trained:} Character must not have the feat \textbf{Magica Profana} or \textbf{Magica Naturae} in order to take up this feat. Additionally, in a serious and realistic tabletop setting - where no one can just learn magic overnight - it should be up to the GM's discression to decide whether characters who didn't already have this feat in the first place should be allowed to take it up during an adventure \textit{(and even so, only religious characters should be allowed to take this up)}.
\end{itemize}
\textbf{Effects:}
\begin{itemize}
	\item \textbf{Untrained:} The character cannot use Clerical Magic.
	\item \textbf{Trained:} The character can use Clerical Magic, and can learn spells and other magic-related feats - albeit he/she is limited by what his/her religion allows. The character also gains the Basic Spells \textit{(Telekinessis, Lighting, Energy Bolt)}.
\end{itemize}\newpage

\input{feats/MagicFeats/NatureMagic.tex}
\input{feats/MagicFeats/DestructionMagic.tex}
\input{feats/MagicFeats/AirMagic.tex}
\input{feats/MagicFeats/EarthMagic.tex}
\input{feats/MagicFeats/FireMagic.tex}
\input{feats/MagicFeats/WaterMagic.tex}
\input{feats/MagicFeats/LightMagic.tex}
\subsection{Sciomancy}
\begin{table}[!ht]
\centering
\FeatIII{Basic}{Advanced}{Expert}{Sciomancy}{{\import{images/Feats/}{images/Feats/DarkMagic1.pdf_tex}}}{{\import{images/Feats/}{images/Feats/DarkMagic2.pdf_tex}}}{{\import{images/Feats/}{images/Feats/DarkMagic3.pdf_tex}}}
\end{table}
\textbf{Requirements:}
\begin{itemize}
	\item \textbf{Basic:} The character already possessing the feat \textbf{Magica Profana} \textit{(enables Arcane Magic)}, \textbf{Magica Divinitatis} \textit{(enables Clerical Magic)}, or \textbf{Magica Naturae} \textit{(enables Nature Magic)}. For users of Clerical Magic, the ability to take up this feat depends on their religion.
	\item \textbf{Advanced:} None \textit{(provided they already have this feat in Basic)}.
	\item \textbf{Expert:} None \textit{(provided they already have this feat in Advanced)}.
\end{itemize}
\textbf{Effects:}
\begin{itemize}
	\item \textbf{Untrained:} The character cannot learn any Dark Magic spells.
	\item \textbf{Basic:} The character can learn Basic-level Dark Magic spells, and also gains a bonus Basic-level Dark Magic spell of their chosing.
	\item \textbf{Advanced:} The character can learn Advanced-level Dark Magic spells, and gets a \Bonus{10\%} to the effectiveness of their Basic-tier Dark Magic spells. The character also gains a bonus Advanced-level Dark Magic spell of their chosing.
	\item \textbf{Expert:} The character can learn Expert-level Dark Magic spells, and gets a \Bonus{30\%} to the effectiveness of their Basic-tier Dark Magic spells and \Bonus{15\%} to the effectiveness of their Advanced-tier Dark Magic spells. The character also gains a bonus Expert-level Dark Magic spell of their chosing.
\end{itemize}\newpage

\input{feats/MagicFeats/Necromancy.tex}

\section{Social Feats}
\input{feats/SocialFeats/Literacy.tex}
\input{feats/SocialFeats/ForeignLanguage.tex}
\input{feats/SocialFeats/ClassicalLanguage.tex}
\input{feats/SocialFeats/Diaesthese.tex}
\input{feats/SocialFeats/Etiquette.tex}
\subsection{Street Smarts}
\begin{table}[!ht]
\centering
\FeatIII{Basic}{Advanced}{Expert}{Street Smarts}{{\import{images/Feats/}{images/Feats/StreetSmarts1.pdf_tex}}}{{\import{images/Feats/}{images/Feats/StreetSmarts2.pdf_tex}}}{{\import{images/Feats/}{images/Feats/StreetSmarts3.pdf_tex}}}
\end{table}
\textbf{Requirements:}
\begin{itemize}
	\item \textbf{Basic:} None, unless your background forbids it.
	\item \textbf{Advanced:} None.
	\item \textbf{Expert:} None.
\end{itemize}
\textbf{Effects:}
\begin{itemize}
	\item \textbf{Untrained:} The slang of the street thugs and other deliquents is downright unintelligible and incomprehensible to the character. Even if you two supposedly speak the same language, it almost feels like there's a language barrier between the two of you - effects up to DM discression, or a \Malus{4} to Charisma when talking to underground-class people.
	\item \textbf{Basic:} The character can understand street slang, but isn't fluent enough in it to show off their social skills to street deliquents - effects up to DM discression, or a \Malus{2} to Charisma when talking to underground-class people.
	\item \textbf{Advanced:} The character can fit into more caddish company and can talk to slang-slinging criminals without having a serious disadvantage at persuasion. No Charisma malus.
	\item \textbf{Expert:} The character feels right at home in the underground, and can have his/her way with the tongue of those street urchins - so much so that, the character gains an advantage at persuading them - effects up to DM discression, or a \Bonus{2} to Charisma when talking to underground-class people.
\end{itemize}\newpage

\input{feats/SocialFeats/Smuggling.tex}
\input{feats/SocialFeats/Logic.tex}
\input{feats/SocialFeats/Music.tex}
\input{feats/SocialFeats/Sex.tex}
\input{feats/SocialFeats/Profession.tex}


\chapter{Backgrounds}
\section{Amnesiac Diviner of the Sea}
Some say that every great man's story begins at their birth. Others prefer starting the story at an important event that set them on the path they would go on. If we were to go with the latter school of thought, we'd say that your story truly begins during a sea battle. You were essentially a sailor with Cassandra's curse - blessed with the ability to see the future, cursed with the fate of not being believed by anyone. \textit{"It's a trap!"} you said, but the captain didn't listen to you. Your ship was ambushed, your ship was captured, your crew was massacred, and you were thrown into the sea to feed the sharks - yet, somhehow, you lived, washed ashore a nameless island. Whether it was because you hit your head extra hard some time along the line, or because of traumatic experience, you now have amnesia, barely remember anything about the past before the battle. How tragic to be able to tell others' future, but not your own past, isn't it?\newline
\textbf{Bonuses:}
\begin{itemize}
	\item Basic Seamanship for free
	\item Your character has premotions of sorts, and can tell if something is wrong in an \textit{"I have a bad feeling about this"} way. The GM has to take this into account, albeit not to the point of metagaming. The character can also vaguely tell the fate of others via some form of diviniation.
\end{itemize}
\textbf{Maluses:}
\begin{itemize}
	\item You can only choose 14 feats at the beginning, instead of the usual 16.
	\item Low amount of starting money.
\end{itemize}

\section{Discredited Academic}
You tried to prove some study that other academicians didn't like, or tried to perform unorthodox experiments that made you get expelled from whatever institute you were working or studying at. Now you walk the earth in hopes that one day, you will find someone who will appreciate your discoveries and theories, either by finding a sponsor to finance your continued studies, or by earning enough money to continue those studies privately.


\textbf{Bonuses:}
\begin{itemize}
	\item \Bonus{2} to Intelligence
	\item You start with the Advanced Literacy feat at your native langauge's script
	\item You start with basic knowledge (Basic Classical Language feat) at a language of your choosing
	\item You start with the Basic Arithmetics and Basic Logic feats for free
	\item You start with either Basic Alchemy or a Basic Profession feat of your choosing
\end{itemize}


\textbf{Maluses:}
\begin{itemize}
	\item \Malus{1} to Charisma
	\item Particularly religious characters will have a lower opinion of this character. Up to the GM to enforce.
	\item Low amount of starting money.
\end{itemize}

\section{Disgraced Aristocrat}
Born into a noble family, good with weapons and destined to be noble protector of the realm, you were once the pride of your father - emphasis on was. You did something that upset your family so terribly, that they disowned you, depriving you of your inheritance and much of your wealth, effectively sending you off to walk the earth. Now you travel the lands as a disgruntled mercenary looking for lucrative jobs that might help you become as rich as you once used to be.\newline
\textbf{Bonuses:}
\begin{itemize}
	\item You start with the Advanced Literacy feat at your native langauge's script
	\item You start with the Basic Etiquette feat
	\item You start with the Basic Riding feat
	\item You start with the Basic Arithmetic feat
	\item You start with the Medium Armour conditioning feat
	\item You start with the Basic Swords feat
	\item You start with either the Advanced Swords feat, or a Basic feat from the usage of any weapon of your choosing.
\end{itemize}
\textbf{Maluses:}
\begin{itemize}
	\item You can only choose 8 feats at the beginning, instead of the usual 16.
	\item Due to your status as a disgraced noble turned cutthroat, upper-class people will have a lower opinion of you than they otherwise would. \textit{(enforcement up to the GM)}
\end{itemize}

\section{Mage's Apprentice}
There are many ways to get acquinted with magic from an early age: both of your parents were magicians, one of your parents was a magician, or you were orphaned and taken in by a magician. Either way, you were raised to be a magician, with your caretaker teaching you the basics at a relatively early age. Your open-minded child brains were ready to suck all that knowledge in, putting you decades ahead of any adult trying to learn magic. On the negative side, your strict caretaker didn't leave much time for you to play, or develop any other skills.


\textbf{Bonuses:}
\begin{itemize}
	\item Magica Profana for free
	\item Free choice between Basic Alchemy or Basic Enchantment for free
	\item Advanced feat in any Magic feat \Parentheses{save for Alchemy and Enchantment} for free
	\item Basic feat in two schools of Magic for free, chosen by player
	\item Advanced Literacy with your native language's script
	\item Basic Literacy with the Despotanmagi script and basic knowledge of Despotanmagi \Parentheses{Basic Classical Language feat}
\end{itemize}
\textbf{Maluses:}
\begin{itemize}
	\item \Malus{2} to Strength
	\item You can only choose 10 feats at the beginning, instead of the usual 16.
	\item Low amount of starting money.
\end{itemize}

\section{The Negotiator}
Ever since an early stage of your life, you have discovered a unique talent of yours: you always had a way with words. Whether you were speaking the slang of the lower-class cad or the flowery poetry of the upper-class peon, you always had a way with words, which saved your butt several times in your life. Of course, being able to talk your way out of every uncomfortable situation had the negative effect of stunting the your growth in other areas - after all, why would you bother learning how to fight, if you can just talk the monster to death?\newline
\textbf{Bonuses:}
\begin{itemize}
	\item \Bonus{3} to Charisma
	\item You start with the Advanced Diaesthese feat
	\item You start with the Basic Etiquette feat
	\item You start with the Basic Street Smarts feat
\end{itemize}
\textbf{Maluses:}
\begin{itemize}
	\item You can only choose 12 feats at the beginning, instead of the usual 16.
	\item \Malus{1} to Strength
	\item \Malus{1} to Endurance
\end{itemize}

\section{Orphaned Paladin}
Most valiant holy knights are of aristocratic blood - but you are not. Truth is, you have no idea who your ancestors were - either they died very early, or they consciously threw you into the trash can as an infant, only for you to be found by wandering knights, who took you in. Through either fate or just luck, once you grew to the apropriate age, they started treating you like a future apprentice, and thus, your life path was set for you from the start. Having trained from a very early age, you are good with weapons and holy magic alike, but the hard years of training didn't really let you explore other areas, thus your skills suffer in just about everything else.\newline
\textbf{Bonuses:}
\begin{itemize}
	\item Medium Armour Conditioning for free.
	\item Basic Flurry for free.
	\item Basic Overswing for free.
	\item Basic Riding for free.
	\item Advanced Horsemanship for free.
	\item Basic Chevauchée for free.
	\item Advanced Swords for free.
	\item Advanced Literacy \Parentheses{at your native language} for free.
	\item Basic knowledge of the Classical High Elven language \Parentheses{Classical Language feat} for free.
	\item Basic Etiquette for free.
	\item Magica Divinitatis for free.
	\item Basic Photomancy for free.
\end{itemize}
\textbf{Maluses:}
\begin{itemize}
	\item You can only pick 4 feats at the start, instead of the usual 16.
	\item You are forbidden from taking on Bows, Crossbows or Firearms as your starting feats. If you stray from the path of the Paladin, you can acquire them later on though.
	\item As long as your character remains a dutiful Paladin, he cannot take on the Bows, Crossbows, Firearms, Sciomancy and Necromancy feats. This is a soft rule meant to be enforced by the GM, as long as the character doesn't go through a significant alignment shift \Parentheses{that is, straying away from Lawful Good}.
	\item Mediocre amount of starting money.
\end{itemize}

\section{Pirate}
Avast, ye scurvy dog! Never feeling quite at home among all those landlubbers, you made the seven seas your home, constantly on the hunt for poorly defended ships full of loot and plunder! You know the open waters like the palm of your own hand, and it is more than obvious that an encounter with you won't be plasant for any merchant ships. Yarrrr!


\textbf{Bonuses:}
\begin{itemize}
	\item Basic Seamanship feat for free.
	\item \Bonus{1} to Strength
	\item \Bonus{1} to Dexterity
\end{itemize}


\textbf{Maluses:}
\begin{itemize}
	\item \Malus{2} to Intelligence
	\item You are a wanted criminal. You can't socialize in public without a disguise in a place where law enforcement exists. \textit{(enforcement up to the GM - shouldn't be active in Shanty Towns, only in more respectable neighbours, where the laws are actually enforced)}
	\item Low amount of starting money.
\end{itemize}

\section{Prodigal Child}
You were born into a family of affluent farmers, and while your other siblings were all content with that, but not you - like the prodigal child you are, you asked for your inheritance and left early for the nearest city, wasting your inherited wealth on booze. Once out of coin though, unlike the Biblical Prodigal Son, you didn't return to your father's house - being too embarassed, you spent your last coins on enough booze to get really drunk - and the next day, you woke up as a member of a group of adventurers. The rest is history - but what is for certain, you've became an adventurer to earn back the money you wasted, so that you can finally face your father again one day.\newline
\textbf{Bonuses:}
\begin{itemize}
	\item Extra alcohol resistance!
\end{itemize}
\textbf{Maluses:}
\begin{itemize}
	\item Low starting money.
\end{itemize}

\section{Runaway Slave}
You are free, free to choose your fate, free to be an adventurer - but that wasn't always so. There was a time in your earlier life when you were in chains, and were someone else's property. Whether you were born into slavery or made a slave at some time in your childhood has little relevance - all that matters is, that your years in chains are behind you, and you couldn't be happier for it. While being a slave effectively forced you to develop a skill against your will, and trained your body, it also prohibited you from becoming accultured or developing any real skills fit for an adventurer.\newline
\textbf{Bonuses:}
\begin{itemize}
	\item \Bonus{2} to Endurance for male characters with this background.
	\item Free to pick an Advanced Profession feat.
	\item Female characters with this background can also pick either a Basic Profession feat, Basic Amorous Culture or Basic Music.
\end{itemize}
\textbf{Maluses:}
\begin{itemize}
	\item You cannot take on any Literacy or Classical language feats as your starting feats. You may acquire them later though.
	\item You cannot take on any Magic feats as your starting feats. You may acquire them later though.
	\item Very low starting money.
\end{itemize}

\section{Silver Spoon}
You were born into and raised as a member of a family belonging to the upper echelons of society. This means you have connections, you know people of importance, and you are known by people of importance. All well and good, but your pampered and sheltered childhood has spoiled you, reducing your ability to learn.\newline
\textbf{Bonuses:}
\begin{itemize}
	\item \Bonus{2} to Charisma
	\item You start with the Advanced Literacy feat at your native langauge's script
	\item You start with the Advanced Etiquette feat
	\item You start with the Basic Riding feat
	\item You start with the Basic Arithmetic feat
	\item You start with basic knowledge (Basic Classical Language feat) at a language of your choosing
	\item Higher-class people will have a higher opinion of you than they otherwise would. \textit{(enforcement up to the GM)}
	\item High amount of starting money.
\end{itemize}
\textbf{Maluses:}
\begin{itemize}
	\item You can only choose 8 feats at the beginning, instead of the usual 16.
	\item \Malus{1} to Willpower
	\item Lower-class people will have a lower opinion of you than they otherwise would. \textit{(enforcement up to the GM)}
	\item Slower learning and experience \textit{(enforcement up to the GM - in a video game, this should be a malus to experience gain)}
\end{itemize}

\section{Street Urchin}
\textbf{Bonuses:}
\begin{itemize}
	\item Dummy
\end{itemize}
\textbf{Maluses:}
\begin{itemize}
	\item Dummy
\end{itemize}

\section{Wandering Monk}
Not all monks spend their time in their monasteries hugging books and praying all day long - some monks wander the lands and walk the earth, for various reasons: to spread their faith and proselytize, to collect ancient artifacts in their church's name \Parentheses{so that their head of religion can hoard them}, or to just act out their religion's commandements in the wild: which, for Light-aligned monks, would mean being a good samaritan, healing the sick, helping those in need and hunting down the demons and the undead - and for Dark-aligned monks, it would be spreading pain and suffering, one can presume. 


\textbf{Bonuses:}
\begin{itemize}
	\item The following feats for free: Advanced Martial Arts, Basic Logic, Advanced Literacy \Parentheses{at your native language}, Basic Etiquette, Magica Divinitatis.
	\item Light-aligned Monks start with Advanced Photomancy. Dark-aligned Monks start with Advanced Sciomancy. Neutral-aligned Monks either start with bonus magic feats defined by their religion, or alternatively, if undefined, start with Basic Geomancy and Basic Aeromancy.
	\item Light-aligned Monks - Monks of Titanius - start with Advanced knowledge of the Classical High Elven language \Parentheses{Classical Language feat}. Other monks instead start with Expert Literacy at their native language's script, instead of Advanced.
\end{itemize}


\textbf{Maluses:}
\begin{itemize}
	\item You can only pick 6 feats at the start, instead of the usual 16.
	\item Light-aligned Monks are forbidden from starting with - and later taking on - any weapon or armour feats, save for Polearms.
	\item Light-aligned Monks can never learn Sciomancy as long as they remain on the path of light - likewise, Dark-aligned monks can never learn Photomancy as long as they remain on the Path of Dark. For other monks, this is defined for their religion.
	\item The character can never learn Necromancy - not even Dark-aligned Monks, as every religion in existence \Parentheses{even Dark-aligned ones} abhors the practice and agree that desecrating the dead is a serious sin that should not be tolerated.
	\item Low amount of starting money.
\end{itemize}

\input{backgrounds/WanderingPriest.tex}

\chapter{Weapons n' Armour}
\chapter{Magicks}
\end{document}
