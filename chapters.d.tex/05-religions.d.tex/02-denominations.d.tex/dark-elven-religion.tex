\subsection{Dark Elven Religion}
\MoreInfo{Dark_Elven_Religion}

\resizebox{.2\linewidth}{!}{\import{images/Religions/}{Dark_Elven_Religion.pdf_tex}}

The \textbf{religion of the Dark Elves} is just as new as the people associated with them. Despite being influenced by them, the religion is partially based off a rejection of Titanism and Naturalism and their dogmatism, instead embracing \SoCalled{the darkness} and \SoCalled{the void}, an abstract concept that transcends both libertarianism and authoritarianism.

Transcendence of traditional concepts is considered a very important part of the Dark Elven religion, allowing the religion and its practicioners to support seemingly contradicting viewpoints, such as libertarianism \Parentheses{sexual freedom, curiousity, innovation, artistic expression, deviance} and authoritarianism \Parentheses{strong central authority, slavery, hierarchial society} at the same time. 

The Dark Elves worship two deities:
\begin{itemize}
  \item \textbf{Braa'darh, the faceless:} According to the legend, when the Dark Elves were expelled they couldn’t survive first, but then Braa'darh gave them mercy and let them survive for a price. No one has ever seen Braa'darh, but he is usually depicted with red eyes and a black smoke substituting for his \SoCalled{face}. The Dark Elves pay tribute and sacrifice to Braa'darh in forms of slave-sacrifice. He is considered the embodiment and incarnation of \SoCalled{the void}, an abstract yet central concept of the Dark Elven religion: \SoCalled{the void} is both freedom and slavery at the same time, life and death, wisdom and ignorance, creation and destruction. 
  \item \textbf{Lolth, the guide:} According to Dark Elven mythology, she is Braa'darh's wife. The Dark Elves claim that Lloth has shown them the path to Darkness and embedded in them enough wisdom to reject the dogmatism and superstition of both Titanism and Naturalism, embracing "the void" instead. She is usually depicted as a beautiful woman.
\end{itemize}

\textbf{Virtues:} Occultism, Hemolatry, Syncretism

\begin{xltabular}{\textwidth}{|c|X|}
\toprule
\multicolumn{2}{|c|}{\textbf{Hierarchy of Sins}} \\
 \bottomrule
 \toprule
\SinElement{10}{\textbf{Showing fear of the dark or the void}}{You are a creature of the dark and of the void. You must embrace it - otherwise, Lolth has guided your ancestors in vain.}
\SinElement{9}{\textbf{Lacking pragmatism}}{Braa'darh has authorized you to use every tool in the box. Do so.}
\SinElement{8}{\textbf{Rejecting an opportunity to learn}}{The void wants you to absorb knowledge.}
\SinElement{7}{\textbf{Rejecting \Parentheses{carnal} pleasure}}{Lolth showed you the way to freedom and pleasure - do not reject it.}
\SinElement{6}{\textbf{Neglecting to sacrifice, to honour Braa'darh and Lolth}}{Braa'darh and Lolth expect regular tribute - in the form of ritual dances, ritual fires, sacrifices - in exchange for helping you through your earthly journey.}
\SinElement{5}{\textbf{Murder without a cause}}{Murder without a valid cause - such as self-defense, revenge or liberation of your folk - is an affront to the void.}
\SinElement{4}{\textbf{Defilement of beauty}}{All that is beautiful in this world was made in Lolth's image - do not spit on her.}
\SinElement{3}{\textbf{Submitting to unworthy authority}}{A race led by weaklings is doomed to failure and extinction.}
\SinElement{2}{\textbf{Disrespecting or assaulting priest or priestess of Darkness}}{The heralds of Braa'darh and Lolth are to be respected.}
\SinElement{1}{\textbf{Betrayal of your kind}}{A traitor to one's own race is a traitor to Braa'darh and Lolth, and is damned according to them.}
 \bottomrule
\end{xltabular}

\newpage
