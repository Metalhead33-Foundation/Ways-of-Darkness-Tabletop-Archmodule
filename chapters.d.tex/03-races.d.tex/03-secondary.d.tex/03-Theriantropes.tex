\section{Theriantropes}
More \textit{(up-to-date)} information at: \url{https://ways-of-darkness.sonck.nl/Theriantropy}


\textbf{Theriantropes} are sentient undead creatures who - unlike Liches - largely retain their original mortal looks, albeit they take on certain animal motiffs, such as the smell of a wild animal and character tics associated with said animal. The condition of theriantropy shares a lot with Vampirism, such as the fact that it is caused by a parasite that kills its host and reanimates the host's corpse as an undead creature, allowing them retain free will, but still influencing them, giving them powers in exchange for blood. On every full moon, young theriantropes go through an involuntary transformation that makes them lose their free will, turning them into berserkers driven by lust for blood - they also go through a physical transformation, from their original form into a bipedal animal \Parentheses{wolf, bear, boar, rat, lion, tiger, etc.}. As theriantropes grow older and stronger, they gradually learn to control their condition, transforming at will, and retaining their free will even when transformed.


Just like how vampirism comes in various forms, theriantropy does to come in many forms, with werewolves \Parentheses{also known as lycantropes} being far by the most common type of theriantrope. Other kinds of theriantropes include werebears, wereboars, wererats, werelions and weretigers. The differences between these were-animals is largely cosmetic, with the common denominator being the fact that they are undead beings stronger than regular mortals, go through involuntary transformations when young, and obstain the smell and character tics of the associated animal.


Just like vampires, theriantropes are a special kind of undead that - through their symbiosis with the parasite - can still get to mingle with mortals, appreciate the taste of mortal food and beverages, reproduce sexually and grow into adulthood when turned prematurely when fed with sufficient amount of blood at regular intervals. However, lack of blood causes involuntary transformations and loss of control first, then the decaying of their body.


\begin{tabular}{|c|c|c|}
\hline
 & \textbf{Bonus/Malus} \\ \hline
\textbf{Strength} & \BonusS{4} \\ \hline
\textbf{Endurance} & \BonusS{4}  \\ \hline
\textbf{Dexterity} & \textit{unchanged}  \\ \hline
\textbf{Intelligence} & \textit{unchanged} \\ \hline
\textbf{Willpower} & \textit{unchanged} \\ \hline
\textbf{Charisma} & \textit{unchanged} \\ \hline
\end{tabular}


\UndeadRace{Theriantropes} Instead, they activate a need for Blood, their own unique vital shared with Vampires.\newpage
