\subsection{Crossbows}
\begin{table}[!ht]
\centering
\FeatIII{Basic}{Advanced}{Expert}{Crossbows}{{\includegraphics[width=0.25\textwidth]{Feats/Crossbow1.pdf}}}{{\includegraphics[width=0.25\textwidth]{Feats/Crossbow2.pdf}}}{{\includegraphics[width=0.25\textwidth]{Feats/Crossbow3.pdf}}}
\end{table}
\textbf{Requirements:}
\begin{itemize}
	\item \textbf{Basic:} None, unless your background forbids it.
	\item \textbf{Advanced:} None.
	\item \textbf{Expert:} None.
\end{itemize}
\textbf{Effects:}
\begin{itemize}
	\item \textbf{Untrained:} The character needs to skip a turn before using a crossbow to reload every single time they want to shoot. They also get a \Malus{4} to Dexterity when it's being counted when using crossbows. In video game adaptations, this should halve the effective range of all bows.
	\item \textbf{Basic:} The character suffers no bonuses or maluses when using crossbows, and can reload fast enough to not to skip a turn doing it.
	\item \textbf{Advanced:} The character reiceves a \Bonus{25\%} to all damage done by crossbows, and a \Bonus{25\%} to chance to hit or cause critical damage. In video game adaptations, this should also increase range.
	\item \textbf{Expert:} The character reiceves a \Bonus{50\%} to all damage done by crossbows, and a \Bonus{50\%} to chance to hit or cause critical damage. In video game adaptations, this should also increase range.
\end{itemize}\newpage
