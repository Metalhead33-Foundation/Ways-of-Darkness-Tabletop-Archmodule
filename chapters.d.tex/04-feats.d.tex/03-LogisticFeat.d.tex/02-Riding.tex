\subsection{Riding}
\begin{table}[!ht]
\centering
\FeatI{Basic Riding}{{\import{images/Feats/}{Riding.pdf_tex}}}
\end{table}
\textbf{Requirements:}
\begin{itemize}
	\item \textbf{Basic:} None, unless your background forbids it.
\end{itemize}
\textbf{Effects:}
\begin{itemize}
	\item \textbf{Untrained:} The character has probably never ridden a mount before, and even if they did, they quickly forget all the lessons learned. They don't really understand how to communicate with their mount. When they say "Yah!" to the animal, it won't move one inch. 
	\item \textbf{Basic:} A character with basic riding skills has a broad understanding of how to mount most mounts and get them to move in the direction the character wants. Some animals are harder to control than others, but as long as the animal in question is already tame to begin with, the character should have no problems mounting it and riding it after some familiarization - of course, mounted combat is a different matter altogether, and a different skill.
\end{itemize}\newpage
