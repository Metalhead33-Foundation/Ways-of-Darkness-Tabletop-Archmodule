\subsection{Arithmetics}
\begin{table}[!ht]
\centering
\FeatIII{Basic}{Advanced}{Expert}{Arithmetics}{{\import{images/Feats/}{images/Feats/Arithmetic1.pdf_tex}}}{{\import{images/Feats/}{images/Feats/Arithmetic2.pdf_tex}}}{{\import{images/Feats/}{images/Feats/Arithmetic3.pdf_tex}}}
\end{table}
\textbf{Requirements:}
\begin{itemize}
	\item \textbf{Basic:} None.
	\item \textbf{Advanced:} None.
	\item \textbf{Expert:} None.
\end{itemize}
\textbf{Effects:}
\begin{itemize}
	\item \textbf{Untrained:} The character can count up to maybe ten, and can literally put two and two together, but not much beyond that.
	\item \textbf{Basic:} The character is acquited with mathematics well enough to count up to hundreds, add, negate, multiply and divide with relative ease.
	\item \textbf{Advanced:} The character isn't intimidated by big numbers, can perform just about all mathematical operations that are can be broken down to primitive functions \textit{(addition, negation, multiplication, division, modulo, even possibly integer exponentiaton)}, albeit possibly rather slowly. With the aid of a lookup table, even trigonometry is possible. Nevertheless, complex equations still confuse the character. The character is also good with boolean algebra.
	\item \textbf{Expert:} The character can not only perform calculations by hand at a reasonable speed, but is also good at comprehending more complex equations - albeit, trigonometry is still going to be slow and difficult without the aid of a lookup table. The character might also know about some of the more obscure sides of mathematics, such as complex numbers and differential algebra \textit{(derivation and integration)}.
\end{itemize}\newpage
