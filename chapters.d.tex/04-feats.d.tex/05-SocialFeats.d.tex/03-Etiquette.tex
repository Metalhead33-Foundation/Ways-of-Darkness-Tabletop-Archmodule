\subsection{Etiquette}
\begin{table}[!ht]
\centering
\FeatIII{Basic}{Advanced}{Expert}{Etiquette}{{\import{images/Feats/}{images/Feats/Etiquette1.pdf_tex}}}{{\import{images/Feats/}{images/Feats/Etiquette2.pdf_tex}}}{{\import{images/Feats/}{images/Feats/Etiquette3.pdf_tex}}}
\end{table}
\textbf{Requirements:}
\begin{itemize}
	\item \textbf{Basic:} None, unless your background forbids it.
	\item \textbf{Advanced:} None.
	\item \textbf{Expert:} None.
\end{itemize}
\textbf{Effects:}
\begin{itemize}
	\item \textbf{Untrained:} Whether the character can read and write or not, he/she acts like an illiterate barbarian or an uncultured peasant. Heavy penalty at conversing with aristocrats - effects up to DM discression, or a \Malus{4} to Charisma when talking to upper-class people.
	\item \textbf{Basic:} The character knows enough about the mannerisms and rules - both written and unwritten - of polite society to avoid publicly embarassing himself/herself, but not enough to impress those belonging to the upper echelons - effects up to DM discression, or a \Malus{2} to Charisma when talking to upper-class people.
	\item \textbf{Advanced:} The character can fit into more cultured company and can talk to nobles without having a serious disadvantage at persuasion. No Charisma malus.
	\item \textbf{Expert:} The character feels right at home among people of the upper echelons, and can have his/her way with the tongue of those upper-class peons - so much so that, the character gains an advantage at persuading them - effects up to DM discression, or a \Bonus{2} to Charisma when talking to upper-class people.
\end{itemize}\newpage
