\subsection{Classical Language}
The classical language skill is special in that it is not a single feat, but rather a group of feats that covers a character's fluency in a dead language.


\textbf{Requirements:}
\begin{itemize}
	\item \textbf{Basic:} Basic Literacy at the language's script.
	\item \textbf{Advanced:} Advanced Literacy at the language's script.
	\item \textbf{Expert:} Expert Literacy at the language's script.
\end{itemize}
\textbf{Effects:}
\begin{itemize}
	\item \textbf{Untrained:} The character knows next to nothing about the classical language in question, except maybe a few loanwords their native tongue has borrowed from it.
	\item \textbf{Basic:} The character knows a lot of words, but cannot really speak the language or write in it, only sort of understand written text via slow and sluggish reading, not even understanding all the words, just figuring out what the sentence as a whole means.
	\item \textbf{Advanced:} The character knows how to read the language, and also knows how to write in it, albeit with many errors, and an unorthodox syntax that would be foreign to the now-extinct native speakers of the tongue. They can also speak the language, albeit only slowly, mostly using simple when words forced to improvise - nevertheless, it will get the point accross to anyone who shares their level of fluency at the ancient tongue.
	\item \textbf{Expert:} The character can read and write the ancient tongue fluently without any errors, and can speak it understanibly, albeit with a strong accent coloured by their native tongue - understandable, as the language in question has no native speakers left, and no one is exactly sure how to pronounce things in it anymore \textit{(for example, think of someone using Italian or Spanish pronounciation when reading Classical Latin - in other words, Eccelestial Latin in a nutshell - or Modern Greek pronounciation when reading Ancient Greek text, Classical Japanese with modern pronounciation, etc.)}.
\end{itemize}\newpage
