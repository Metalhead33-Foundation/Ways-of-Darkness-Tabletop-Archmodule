\subsection{Street Smarts}
\begin{table}[!ht]
\centering
\FeatIII{Basic}{Advanced}{Expert}{Street Smarts}{{\includegraphics[width=0.25\textwidth]{Feats/StreetSmarts1.pdf}}}{{\includegraphics[width=0.25\textwidth]{Feats/StreetSmarts2.pdf}}}{{\includegraphics[width=0.25\textwidth]{Feats/StreetSmarts3.pdf}}}
\end{table}
\textbf{Requirements:}
\begin{itemize}
	\item \textbf{Basic:} None, unless your background forbids it.
	\item \textbf{Advanced:} None.
	\item \textbf{Expert:} None.
\end{itemize}
\textbf{Effects:}
\begin{itemize}
	\item \textbf{Untrained:} The slang of the street thugs and other deliquents is downright unintelligible and incomprehensible to the character. Even if you two supposedly speak the same language, it almost feels like there's a language barrier between the two of you - effects up to DM discression, or a \Malus{4} to Charisma when talking to underground-class people.
	\item \textbf{Basic:} The character can understand street slang, but isn't fluent enough in it to show off their social skills to street deliquents - effects up to DM discression, or a \Malus{2} to Charisma when talking to underground-class people.
	\item \textbf{Advanced:} The character can fit into more caddish company and can talk to slang-slinging criminals without having a serious disadvantage at persuasion. No Charisma malus.
	\item \textbf{Expert:} The character feels right at home in the underground, and can have his/her way with the tongue of those street urchins - so much so that, the character gains an advantage at persuading them - effects up to DM discression, or a \Bonus{2} to Charisma when talking to underground-class people.
\end{itemize}\newpage
