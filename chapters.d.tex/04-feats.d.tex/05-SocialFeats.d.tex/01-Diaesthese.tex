\subsection{Diaesthese}
\begin{table}[!ht]
\centering
\FeatIII{Basic}{Advanced}{Expert}{Diaesthese}{{\import{images/Feats/}{Diaesthese1.pdf_tex}}}{{\import{images/Feats/}{Diaesthese2.pdf_tex}}}{{\import{images/Feats/}{Diaesthese3.pdf_tex}}}
\end{table}
\textbf{Requirements:}
\begin{itemize}
	\item \textbf{Basic:} None, unless your background forbids it.
	\item \textbf{Advanced:} None.
	\item \textbf{Expert:} None.
\end{itemize}
\textbf{Effects:}
\begin{itemize}
	\item \textbf{Untrained:} Unless the character knows Etiquette or Street Smarts, the character lacks people-skills altogether, and also lacks the intuition necessary to detect lies, emotional manipulation, thus lacks the ability to deflect persuasion attempts, unless the other person is trying to convince them to do something they absolutely do not want to do. \textit{(Enforcement largely up to GM - unsure how to implement in CRPG adaptation)}
	\item \textbf{Basic:} The character has some awareness, intuition and people-skills that allow them to detect lies to a degree and intuit other people's motivations \textit{(both to detect and deflect persuasion attempts from them and also to better persuade them as well)}. Basically, the character knows human nature enough to have some people-skills. \textit{(Enforcement largely up to GM - unsure how to implement in CRPG adaptation)}
	\item \textbf{Advanced:} The character can read emotions and intuit other people's motivations well enough to suspect that something's not right when other people lie to them or attempt to persuade them. They also know enough about human nature to meaningfully persuade and manipulate others. \textit{(Enforcement largely up to GM - unsure how to implement in CRPG adaptation)}
	\item \textbf{Expert:} The character can detect lies from all but the most masterful liars, and can profile people rather easily. The character has the necessary skill to easily build rapport and manipulate others, as well as deflect manipulation attempts from all but the most skilled manipulators. \textit{(Enforcement largely up to GM - unsure how to implement in CRPG adaptation)}
\end{itemize}\newpage
