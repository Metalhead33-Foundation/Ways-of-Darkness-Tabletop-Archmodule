\subsection{Literacy}
The literacy skill is special in that it is not a single feat, but rather a group of feats that covers a character's fluency with a specific writing system.
\begin{table}[h]
\center
\begin{threeparttable}
\caption{Literacy}
\begin{tabular*}{\textwidth}{c@{\extracolsep{\fill}}ccc}
\hline
%\multicolumn{4}{|c|}{Literacy} \\ \hline
\LangFeatContent{High Elven}{{\import{images/Feats/}{images/Feats/Literacy-Helven1.pdf_tex}}}{{\import{images/Feats/}{images/Feats/Literacy-Helven2.pdf_tex}}}{{\import{images/Feats/}{images/Feats/Literacy-Helven3.pdf_tex}}}
\LangFeatContent{Wood Elven}{{\import{images/Feats/}{images/Feats/Literacy-Welven1.pdf_tex}}}{{\import{images/Feats/}{images/Feats/Literacy-Welven2.pdf_tex}}}{{\import{images/Feats/}{images/Feats/Literacy-Welven3.pdf_tex}}}
\LangFeatContent{Neressan}{{\import{images/Feats/}{images/Feats/Literacy-Neressan1.pdf_tex}}}{{\import{images/Feats/}{images/Feats/Literacy-Neressan2.pdf_tex}}}{{\import{images/Feats/}{images/Feats/Literacy-Neressan3.pdf_tex}}}
\LangFeatContent{Dwarven}{{\import{images/Feats/}{images/Feats/Literacy-Dwarven1.pdf_tex}}}{{\import{images/Feats/}{images/Feats/Literacy-Dwarven2.pdf_tex}}}{{\import{images/Feats/}{images/Feats/Literacy-Dwarven3.pdf_tex}}}
\LangFeatContent{Gabyrian}{{\import{images/Feats/}{images/Feats/Literacy-Gabyrian1.pdf_tex}}}{{\import{images/Feats/}{images/Feats/Literacy-Gabyrian2.pdf_tex}}}{{\import{images/Feats/}{images/Feats/Literacy-Gabyrian3.pdf_tex}}}
\end{tabular*}
\begin{tablenotes}
      \small
      \item \textbf{High Elven:} Used for the High Elven, Classical High Elven, Artaburran Wood Elven, Dark Elven, Halfling, Etrandish and Etrancoasti languages.
      \item \textbf{Wood Elven:} Used for the Dragoci Wood Elven language.
      \item \textbf{Neressan:} Used for the Neressan language.
      \item \textbf{Dwarven:} Used for the Dwarven language.
      \item \textbf{Gabyrian:} Used for the Gabyrian language.
    \end{tablenotes}
\end{threeparttable}
\end{table}
\textbf{Requirements:}
\begin{itemize}
	\item \textbf{Basic:} None.
	\item \textbf{Advanced:} None.
	\item \textbf{Expert:} None.
\end{itemize}
\textbf{Effects:}
\begin{itemize}
	\item \textbf{Untrained:} The character is illiterate and cannot read and write - at least not with the writing system in question.
	\item \textbf{Basic:} In case of an alphabet, abjad, abugida or syllabary, the character more or less knows which letters map to which sounds and phonemes, albeit makes constant reading and spelling errors if the script isn't particularly phonemic, with historical and etymological spelling just straight-up confusing the character. In case of a logographic script, the character may know perhaps less than a hundred characters. Either way, the character both reads and writes slowly, and might not even be able to read at all without having to sound out the written words out loud.
	\item \textbf{Advanced:} In case of an alphabet, abjad, abugida or syllabary, character has memorized enough rules and words to know about the perils of etymological and historical spelling, thus knows how to spell most commonly used words correctly, though they may still struggle with some fancy names - or in case of a logographic system, the character knows at least five hundred characters. Either way, if the character didn't already possess the ability of silent reading, now they are guaranteed to do.
	\item \textbf{Expert:} The character can both read and write perfectly fluently. If the writing system is an alphabet, then the character knows how to spell the vast majority of words in their language correctly, and isn't confused by historical or etymological spellings - if the writing system is logographic, the character knows 3000-5000 symbols.
\end{itemize}\newpage
