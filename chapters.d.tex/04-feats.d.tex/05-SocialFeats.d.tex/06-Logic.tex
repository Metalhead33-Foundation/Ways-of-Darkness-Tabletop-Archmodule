\subsection{Logic}
\begin{table}[!ht]
\centering
\FeatIII{Basic}{Advanced}{Expert}{Logic}{{\import{images/Feats/}{images/Feats/Logic1.pdf_tex}}}{{\import{images/Feats/}{images/Feats/Logic2.pdf_tex}}}{{\import{images/Feats/}{images/Feats/Logic3.pdf_tex}}}
\end{table}
\textbf{Requirements:}
\begin{itemize}
	\item \textbf{Basic:} An intelligence of 10.
	\item \textbf{Advanced:} An intelligence of 12.
	\item \textbf{Expert:} An intelligence of 14.
\end{itemize}
\textbf{Effects:}
\begin{itemize}
	\item \textbf{Untrained:} The character hasn't been formally schooled in Logic - which doesn't necessarily mean that they are devoid of logic or incapable of logical reasoning, potentially, it's quite the contrary: being able to put two and two together and discover the relationships between things is just being intelligent. \textit{"Logic"} in this context refers to more abstract fields of study and philosophy, something formally studied, beyond regular intuition.
	\item \textbf{Basic:} The character has received some formal schooling in Logic. Besides being skilled at logical reasoning - whether deductive, inductive or abductive, up to their preference - they also have knowledge about syllogistic logic, as well as Boolean logic.
	\item \textbf{Advanced:} The character possesses a comprehensive understanding of classical logic and philosophy, and thus can use this knowledge in philosophical debates, as well as when answering complex questions.
	\item \textbf{Expert:} The character is a philosopher in the making.
\end{itemize}\newpage
