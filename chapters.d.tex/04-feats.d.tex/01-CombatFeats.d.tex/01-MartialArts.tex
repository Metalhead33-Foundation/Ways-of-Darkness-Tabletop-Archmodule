\subsection{Martial Arts}
\begin{table}[!ht]
\centering
\FeatIII{Basic}{Advanced}{Expert}{Martial Arts}{{\includegraphics{Feats/Unarmed1}}}{{\includegraphics{Feats/Unarmed2}}}{{\includegraphics{Feats/Unarmed3}}}
\end{table}
\textbf{Requirements:}
\begin{itemize}
	\item \textbf{Basic:} None, unless your background forbids it.
	\item \textbf{Advanced:} None.
	\item \textbf{Expert:} None.
\end{itemize}
\textbf{Effects:}
\begin{itemize}
	\item \textbf{Untrained:} The character doesn't know any martial arts at all, only basic moves \Parentheses{such as punching, slapping, kicking, tripping, biting, headbutting, etc.} - in unarmed combat, the character is forced to rely purely on their physical strength, which isn't necessarily a bad thing, if their Strength attribute is high.
	\item \textbf{Basic:} The character knows some martial arts at a beginner level - in other words the character's dexterity counts towards their unarmed damage - if the character's dexterity is higher than their strength, strength is ignored, and dexterity is used in calculations instead. Additionally, the character gains a \Bonus{10\%} to all unarmed melee damage.
	\item \textbf{Advanced:} The character is rather good at martial arts - in other words, when the character does unarmed damage, their actual strength is substituted with 75\% of a combination of their Strength and Dexterity. Additionally, the character gains a \Bonus{20\%} to all unarmed melee damage.
	\item \textbf{Expert:} The character is a master at martial arts - in other words, when the character does unarmed damage, during the calculation, their actual strength is substituted with a sum of their Strength and Dexterity. Additionally, the character gains a \Bonus{30\%} to all unarmed melee damage.
\end{itemize}\newpage
