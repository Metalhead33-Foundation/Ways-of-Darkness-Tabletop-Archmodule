\subsection{Lockpicking}
\begin{table}[!ht]
\centering
\FeatIII{Basic}{Advanced}{Expert}{Lockpicking}{{\includegraphics{Feats/Lockpicking1}}}{{\includegraphics{Feats/Lockpicking2}}}{{\includegraphics{Feats/Lockpicking3}}}
\end{table}
\textbf{Requirements:}
\begin{itemize}
	\item \textbf{Basic:} None, unless your background forbids it.
	\item \textbf{Advanced:} None.
	\item \textbf{Expert:} None.
\end{itemize}
\textbf{Effects:}
\begin{itemize}
	\item \textbf{Untrained:} The character cannot pick any locks.
	\item \textbf{Basic:} When the character attempts to pick a lock, a d20 dice \Parentheses{or the sum of four d5 dices, or five d4 dices} must be rolled, The lockpicking succeeds if the number is smaller than the character's Dexterity. \Parentheses{some locks are more difficult to pick than others - at DM/GM discretion, constant numbers can be added to the rolled dice, it can be multiplied, etc. to simulate harder-to-pick locks. At DM discretion, some locks could be straight-up declared too difficult for a character with Basic Lockpicking to pick}
	\item \textbf{Advanced:} The rolled dice has to be multiplied by \( \frac{3}{4} \).
	\item \textbf{Expert:} The rolled dice has to be multiplied by \( \frac{1}{4} \).
\end{itemize}\newpage
