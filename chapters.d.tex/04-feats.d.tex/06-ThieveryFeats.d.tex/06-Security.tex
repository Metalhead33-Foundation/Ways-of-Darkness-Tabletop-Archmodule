\subsection{Security}
\begin{table}[!ht]
\centering
\FeatIII{Basic}{Advanced}{Expert}{Security}{{\import{images/Feats/}{images/Feats/Security1.pdf_tex}}}{{\import{images/Feats/}{images/Feats/Security2.pdf_tex}}}{{\import{images/Feats/}{images/Feats/Security3.pdf_tex}}}
\end{table}
\textbf{Requirements:}
\begin{itemize}
	\item \textbf{Basic:} None, unless your background forbids it.
	\item \textbf{Advanced:} None.
	\item \textbf{Expert:} None.
\end{itemize}
\textbf{Effects:}
\begin{itemize}
	\item \textbf{Untrained:} The character cannot disarm traps, and can only detect the most obvious ones \Parentheses{up to DM discretion}.
	\item \textbf{Basic:} Character can immediately notice more advanced traps \Parentheses{up to DM discretion}. Additionally, when the character attmepts to disarm a trap, they must roll a d20 dice \Parentheses{or the sum of four d5 dices, or five d4 dices}. When the received number is smaller than the character's dexterity, the disarming attempt is succesful. Otherwise, it is a fail, and the trap deals damage to the character.
	\item \textbf{Advanced:} The rolled dice has to be multiplied by \( \frac{3}{4} \). Additionally, the character is instinctively notified of almost all traps, save for the most masterfully crafted ones \Parentheses{up to DM discretion}.
	\item \textbf{Expert:} The rolled dice has to be multiplied by \( \frac{1}{4} \). The character can also notice all traps immediately.
\end{itemize}\newpage
