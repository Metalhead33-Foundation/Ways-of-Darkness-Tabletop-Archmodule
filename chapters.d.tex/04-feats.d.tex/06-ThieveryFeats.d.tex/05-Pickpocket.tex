\subsection{Pickpocket}
\begin{table}[!ht]
\centering
\FeatIII{Basic}{Advanced}{Expert}{Lockpicking}{{\includegraphics[width=0.25\textwidth]{Feats/Pickpocket1.pdf}}}{{\includegraphics[width=0.25\textwidth]{Feats/Pickpocket2.pdf}}}{{\includegraphics[width=0.25\textwidth]{Feats/Pickpocket3.pdf}}}
\end{table}
\textbf{Requirements:}
\begin{itemize}
	\item \textbf{Basic:} None, unless your background forbids it.
	\item \textbf{Advanced:} None.
	\item \textbf{Expert:} None.
\end{itemize}
\textbf{Effects:}
\begin{itemize}
	\item \textbf{Untrained:} The character cannot pickpocket.
	\item \textbf{Basic:} When the character attmepts to pickpocket, they must roll a d20 dice \Parentheses{or the sum of four d5 dices, or five d4 dices}. When the received number is smaller than the character's dexterity, the pickpocket attempt is succesful. Otherwise, it is a fail, and the target is alerted.
	\item \textbf{Advanced:} The rolled dice has to be multiplied by \( \frac{3}{4} \).
	\item \textbf{Expert:} The rolled dice has to be multiplied by \( \frac{1}{4} \).
\end{itemize}\newpage
